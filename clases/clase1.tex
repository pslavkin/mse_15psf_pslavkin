%-------------------------------------------------------------------------------
\subtitle{Clase 1 - Introducción}
\begin{frame}[c]
\maketitle
\begin{tikzpicture}[overlay,remember picture]
    \node[anchor=south east,xshift=-30pt,yshift=45pt]
      at (current page.south east) {
        \includegraphics[width=45mm]{1_clase/python_continuo_vs_discreto}
      };
  \end{tikzpicture}
\end{frame}
%-------------------------------------------------------------------------------
\AtBeginSection[]{% Print an outline at the beginning of sections
\begin{frame}
   \frametitle{Resumen de seccion \thesection}
   \tiny{\tableofcontents[currentsection]}
 \end{frame}}
%-------------------------------------------------------------------------------
 \section{Señales}
 \subsection{Plan de vuelo}
 \begin{frame}{Plan de vuelo}{Ud. esta aqui}
    \begin{columns}[onlytextwidth]
       \column{.7\textwidth}
       \center\includegraphics[width=1.0\textwidth]{1_clase/Esquema_MSE}
       \column{.3\textwidth}
       \tiny{
          Colaboardores
          \begin{itemize}
             \item{Gonzalo Lavigna             \\ <gonzalolavigna@gmail.com>}
             \item{Guillermo Guichal           \\ <guillermo.guichal@gmail.com>}
             \item{Federico Giordano Zacchigna \\ <federico.zacchigna@gmail.com>}
          \end{itemize}
        }
    \end{columns}
    \vfill
 \end{frame}
%-------------------------------------------------------------------------------
 \begin{frame}{Plan de vuelo}{Ud. esta aqui}
    \begin{columns}[onlytextwidth]
       \column{.5\textwidth}
       \center\includegraphics[width=0.7\textwidth]{1_clase/Esquema_MSE2}
       \column{.5\textwidth}
       \begin{enumerate}
          \item{
                \begin{itemize}
                      \scriptsize{
                      \item {Sampleo}
                      \item {Fourier, Z}
                      \item {Filtrado basico}
                      \item {CIAA}
                      \item {Python}
                      }
                \end{itemize}
             }
          \item{
                \begin{itemize}
                      \scriptsize{
                      \item {Estadistica}
                      \item {Canales de comunicacion}
                      \item {Filtrado y ventaneo}
                      \item {FPGA?}
                      \item {Python}
                      }
                \end{itemize}
             }
          \item{
                \begin{itemize}
                      \scriptsize{
                      \item{Implementacion}
                      \item{Hi-Speed}
                      \item{Comunicaciones}
                      \item{FPGA}
                      \item{Python}
                      }
                \end{itemize}
             }
       \end{enumerate}
    \end{columns}
 \end{frame}
%-------------------------------------------------------------------------------
 \begin{frame}{Bibliografia}
    \framesubtitle{Libros, links y otro material}
    \tiny{
       \begin{thebibliography}{1}
             \bibitem{smith}
             Steven W. Smith. \\
             \textcolor{green}{\emph{The Scientist and Engineer's Guide to Digital Signal Processing}} \\
             Second Edition, 1999.
             \bibitem{thinkdsp}
             Allen B. Downey \\
             \textcolor{green}{\emph{Think DSP - Digital Signal Processing in Python}}
             \bibitem{lyons}
             Richard Lyons. \\
             \emph {Understanding digital signal processing}. \\
             Third edition.
             \bibitem{porat}
             Boaz Porat. \\
            \emph{Digital Processing of Random Signals: Theory and Methods}.
             \emph{Digital Processing of Random Signals: Theory and Methods}.
             \bibitem{thinkpython}
             Allen B. Downey \\
             \emph{ Think Python, 2nd Edition, - How to Think Like a Computer Scientist}
             \bibitem{digitalsignalprocessing}
             Emmanuel C Ifeachor, Barrie W Jervis \\
             \emph{ Digital Signal Processing. A practial approach.}
             \bibitem{ntroductiontopython}
             NW. Taylor, Francis Group, LLC.\\
             \emph{ Introduction to Python Programming.}
             \bibitem{digitalsignalprocessing}
             Matt Harrison\\
             \emph{ Illustrated guide to python 3}
       \end{thebibliography}
   }
    \end{frame}
%-------------------------------------------------------------------------------
 \begin{frame}{Evaluacion}{Proyecto final}
    \protoboardicon
    \begin{columns}[onlytextwidth]
       \column{.5\textwidth}
       \begin{itemize}
          \item Debera incluir algun tipo de procesamiento en hardware. ej. fft, fir, iir
          \item Puede utilizar el ADC para samplear, DAC para reconstruir y/o canales de comunicacion para adquirir datos previamente digitalizados
          \item presentacion de 10 minutos.
          \item debera funcionar!
       \end{itemize}
       \column{.5\textwidth}
       Ejemplos:
       \begin{itemize}
          \item Filtrado y/o procesamiento de audio, señales biomedicas, etc.
          \item Tecnicas de compresion en dominio de la frecuencia
          \item Aplicaciones con acelerometro, magnetometro, T+H
       \end{itemize}
    \end{columns}
    \vfill
 \end{frame}
%-------------------------------------------------------------------------------
 \subsection{porque digital?}
 \begin{frame}{porque digital?}{digital vs analogico}
    \begin{columns}[onlytextwidth]
       \column{.5\textwidth}
       \begin{itemize}
          \item{digital}
             \begin{itemize}
                \item{reproducibilidad}
                \item{tolerancia de componentes}
                \item{partidas todas iguales}
                \item{componentes no envejecen}
                \item{facil de actualizar}
                \item{soluciones de un solo chip}
             \end{itemize}
          \item{analogico}
             \begin{itemize}
                \item{alto ancho de banda}
                \item{alta potencia}
                \item{baja latencia}
             \end{itemize}
       \end{itemize}
       \column{.5\textwidth}
       \includegraphics[width=30mm]{1_clase/fpga}
       \newline
       \includegraphics[width=35mm]{1_clase/transistor_amp}
    \end{columns}
 \end{frame}
%-------------------------------------------------------------------------------
 \subsection{Señales}
 \begin{frame}{Señales y sistemas}{Que son?}
    \begin{block}{Señal}
       Una señal, en función de una o más variables, puede definirse como un cambio observable en una entidad cuantificable
    \end{block}
    \begin{block}{Sistema}
       Un sistema es cualquier conjunto físico de componentes que actúan en una señal, tomando una o más señales de entrada, y produciendo una o más señales de salida.
    \end{block}
 \end{frame}
%-------------------------------------------------------------------------------
 \begin{frame}{Señales y sistemas}{Tipos de señales}
    \begin{columns}[onlytextwidth]
       \column{.5\textwidth}
       \begin{itemize}
          \item{De tiempo continuo}
          \item{Pares}
          \item{Periódicas}
          \item{De energía}
          \item{Reales}
       \end{itemize}
       \column{.5\textwidth}
       \begin{itemize}
          \item{De tiempo discreto}
          \item{No deterministas}
          \item{Impares}
          \item{Aperiódicas}
          \item{De potencia}
          \item{Imaginarias}
       \end{itemize}
    \end{columns}
 \end{frame}
%-------------------------------------------------------------------------------
 \begin{frame}{Señales y sistemas}{Tipos de señales}
    \begin{columns}[onlytextwidth]
       \column{.45\textwidth}
       \begin{itemize}
          \item{De tiempo continuo}
       \end{itemize}
       Tiene valores para todos los puntos en el tiempo en algún intervalo (posiblemente infinito)
       \column{.45\textwidth}
       \begin{itemize}
          \item{De tiempo discreto}
       \end{itemize}
       Tiene valores solo para puntos discretos en el tiempo
    \end{columns}
    \vfill
    \includegraphics[width=\textwidth]{1_clase/continuo_vs_discreto}
 \end{frame}
%-------------------------------------------------------------------------------
 \begin{frame}{Generacion de señales en Python}{Continuo? vs discreto}
    \lstset{ basicstyle=\fontsize{ 8}{ 2}\selectfont\ttfamily }
    \begin{columns}[onlytextwidth]
       \column{.6\textwidth}
       \lstinputlisting[language=Python,tabsize=4]{1_clase/sine.py}
       \column{.4\textwidth}
       \includegraphics[width=\textwidth]{1_clase/python_continuo_vs_discreto}
    \end{columns}
    Podrian pensarse como muestras de una señal de tiempo continuo $x[n] = x (nT)$ donde n es un número entero y \textbf{T} es el período de muestreo.
    \vfill
 \end{frame}
%-------------------------------------------------------------------------------
 \begin{frame}{Señales periodicas}
    \begin{block}{Continua periodica}
       si existe un $T_0>0$, tal que $x(t+T_0)=x(t)$, para todo $t$\\
       $T_0$ es el período de $x(t)$ medido en tiempo, y $f_0=1/T_0$ es la frecuencia fundamental de $x(t)$
    \end{block}
    \begin{block}{Discreta periodica}
       si existe un entero $N_0>0$ tal que $x[n+N_0]=x[n]$ para
       todo $n$ \\
       $N_0$ es el período fundamental de $x[n]$ medido en espacio entre muestras
       y  $F_0=\Delta t/N_0$ es la frecuencia fundamental de $x[n]$
    \end{block}
    \center\includegraphics[width=0.5\textwidth]{1_clase/periodica}
    \vfill
 \end{frame}
%-------------------------------------------------------------------------------
 \subsection{Sistemas}
 \begin{frame}{Sistemas}
    \begin{block}{Sistema}
       Un sistema es cualquier conjunto físico de componentes que actúan en una señal, tomando una o más señales de entrada, y produciendo una o más señales de salida.
    \end{block}
    En ingeniería, a menudo la entrada y la salida son señales eléctricas.\\
    \centering{\includegraphics[width=0.5\textwidth]{1_clase/sistema}}
    \vfill
 \end{frame}
   %-------------------------------------------------------------------------------
 \begin{frame}{Sistemas}
    \begin{block}{Linealidad}
       Un sistema es lineal cuando su salida depende linealmente de la entrada.
       Satisface el principio de superposicion.
    \end{block}
    \begin{centering}
       \begin{table}[h]
          \begin{tabular}{cm{6cm}cm{6cm}}
             escalado      & \includegraphics[width=0.4\textwidth]{1_clase/superposicion1}\\
             adicion       & \includegraphics[width=0.4\textwidth]{1_clase/superposicion2}\\
             superposicion & \includegraphics[width=0.4\textwidth]{1_clase/superposicion3}\\
          \end{tabular}
       \end{table}
    \end{centering}
    \centering{\alert{$y(t)=e^{x(t)}$}\hspace{1cm}\textcolor{green}{$y(t)=\frac{1}{2}x(t)$}}
    \vfill
 \end{frame}
   %-------------------------------------------------------------------------------
 \begin{frame}{Sistemas}{Invariantes en el tiempo}
    \begin{block}{Invariantes en el tiempo}
       Un sistema es invariante en el tiempo cuando la salida para una determinada entrada es la misma sin importar el tiempo en el cual se aplica la entrada
    \end{block}
    \center\includegraphics[width=1\textwidth]{1_clase/invariante_en_tiempo} \\
    \centering{\alert{$y(t)=x(t)*\cos{(t)}$}\hspace{1cm}\textcolor{green}{$y(t)=\cos(x(t))$}}
    \vfill
 \end{frame}
 %-------------------------------------------------------------------------------
 \begin{frame}{Sistemas}{Causalidad}
    \begin{block}{Sistema causal}
       Un sistema es causal cuando la salida depende solo de los valores presentes y pasados de la entrada
    \end{block}
    \center\includegraphics[width=0.4\textwidth]{1_clase/causalidad} \\
    \centering{\alert{$y(t)=x(t+1)$}\hspace{1cm}\textcolor{green}{$y(t)=x(t-2)$}}
    \vfill
 \end{frame}
%-------------------------------------------------------------------------------
 \begin{frame}{Sistemas}{}
    \begin{block}{Lineales invariantes en el tiempo}
       Un sistema es LTI cuando satisface las 2 condiciones anteriores, de linealidad y de invariancia en el tiempo.
    \end{block}
    \center\includegraphics[width=1\textwidth]{1_clase/lti}
    \vfill
    \begin{alertblock}{*** LTI ***}
       En este curso, \alert{solo} estudiaremos sistemas lineales invariantes en el tiempo.
    \end{alertblock}
 \end{frame}
%-------------------------------------------------------------------------------
 \begin{frame}{Sistemas}{}
    \begin{block}{Linealidad estatica}
       En todo sistema LTI para una entrada constante (DC) la salida es \textcolor{green}{siempre} la entrada multiplicada por una constante.
    \end{block}
    \begin{block}{Fidelidad senoidal}
       En todo sistema LTI para una entrada senoidal la salida es \textcolor{green}{siempre} senoidal.
    \end{block}
    \vfill
 \end{frame}
%------------------------------------------------------------------------------
 \section{ADC}
 \begin{frame}{ADC}{Bloque incompleto de procesamiento}
    \center\includegraphics[width=1\textwidth]{1_clase/adc_dac1}
    \begin{alertblock} {Que falta?}
    \end{alertblock}
    \vfill
 \end{frame}
%-------------------------------------------------------------------------------
 \subsection{Aliasing}
 \begin{frame}{Aliasing}{Simulando en Python}
    \handsonicon
    Diferentes frecuencias de sampleo para capturar una señal de 50hz
    \lstset{ basicstyle=\fontsize{10}{ 2}\selectfont\ttfamily }
    \lstinputlisting[language=Python,tabsize=4]{1_clase/teorema_sampleo.py}
    \vfill
 \end{frame}
%-------------------------------------------------------------------------------
 \begin{frame}{Aliasing}{Simulando en Python}
    Diferentes frecuencias de sampleo para capturar una señal de 50hz
    \center\includegraphics[width=1.0\textwidth]{1_clase/teorema_sampleo}
    \vfill
 \end{frame}
%-------------------------------------------------------------------------------
 \begin{frame}{Aliasing}{Simulando en Python}
    Que pasa si se suma ruido de alta frecuencia?
    \center\includegraphics[width=1.0\textwidth]{1_clase/teorema_sampleo2}
    \vfill
 \end{frame}
%-------------------------------------------------------------------------------
 \begin{frame}{Aliasing}{Disco Giratorio}
    \center{
       \href{run:./1_clase/disco_aliasing.mp4}{
          \includegraphics[width=0.8\textwidth]{1_clase/disco_aliasing}
       }
    }
    \vfill
 \end{frame}
%------------------------------------------------------------------------------
 \section{ADC}
 \begin{frame}{ADC}{Bloque generico de procesamiento}
    \center\includegraphics[width=1\textwidth]{1_clase/adc_dac2}
    \begin{alertblock} {Agregamos el filtro antialising}
    \end{alertblock}
    \vfill
 \end{frame}
%-------------------------------------------------------------------------------
 \subsection{Teorema de Shannon}
 \begin{frame}{Teorema de sampleo}{Teorema de Shannon}
    \begin{teorema}
       La reconstrucción exacta de una señal periódica continua en banda base a partir de sus muestras, es matemáticamente posible si la señal está \textcolor{green}{limitada en banda} y la tasa de muestreo es \textcolor{green}{superior al doble} de su ancho de banda
    \end{teorema}
    \begin{columns}[onlytextwidth]
       \column{.6\textwidth}
       \center\includegraphics[width=0.7\textwidth]{1_clase/shannon} \\
       \center\includegraphics[width=0.7\textwidth]{1_clase/sinc_limpia}
       \column{.4\textwidth}
       \center\includegraphics[width=0.6\textwidth]{1_clase/claude_shannon}
    \end{columns}
    \vfill
 \end{frame}
%-------------------------------------------------------------------------------
 \begin{frame}{Teorema de sampleo}{Teorema de Shannon}
    \handsonicon
    Sampleo e interpolado
    \lstset{ basicstyle=\fontsize{ 7}{ 1}\selectfont\ttfamily }
       \lstinputlisting[language=Python,tabsize=4]{1_clase/teorema_sampleo_interpolado.py}
    \vfill
 \end{frame}
%-------------------------------------------------------------------------------
 \begin{frame}{Teorema de sampleo}{Teorema de Shannon}
    \handsonicon
    Sampleo e interpolado
    \center\includegraphics[width=1.0\textwidth]{1_clase/teorema_sampleo_interpolado}
    \vfill
 \end{frame}
%-------------------------------------------------------------------------------
 \begin{frame}{Sampleo}{Filtro antialias}
    \begin{block}{FAA}
       Filtro \alert{analogico} pasabajos que elimina o al menos mitiga el efecto de aliasing
    \end{block}
    \center\includegraphics[width=0.8\textwidth]{1_clase/filtro_anti_aliasing}
    \vfill
 \end{frame}
%-------------------------------------------------------------------------------
 \section{Quantizacion}
 \subsection{Ejemplos}
 \begin{frame}{Ruido de cuantizacion}{Ejemplo de cuantizacion}
    Diferentes formas de onda cuantizadas
    \center\includegraphics[width=1\textwidth]{1_clase/noise_examples}
    \vfill
 \end{frame}
%-------------------------------------------------------------------------------
 \begin{frame}{Ruido de cuantizacion}{Cuantizacion en python}
    \handsonicon
    \lstset{ basicstyle=\fontsize{ 9}{ 2}\selectfont\ttfamily }
    \lstinputlisting[language=Python,tabsize=4]{1_clase/noise_model.py}
    \vfill
 \end{frame}
%-------------------------------------------------------------------------------
 \begin{frame}{Ruido de cuantizacion}{Histogramas}
    Histogramas de ruido para cada señal
    \center\includegraphics[width=1\textwidth]{1_clase/noise_histogram}
    \vfill
 \end{frame}
%-------------------------------------------------------------------------------
 \begin{frame}{Ruido de cuantizacion}{Histogramas}
    Histogramas en Python
    \handsonicon
    \lstset{ basicstyle=\fontsize{ 8}{ 2}\selectfont\ttfamily }
    \lstinputlisting[language=Python,tabsize=4]{1_clase/noise_histogram.py}
    \vfill
 \end{frame}
%-------------------------------------------------------------------------------
 \subsection{Modelo estadistico}
 \begin{frame}{Ruido de cuantizacion}{Modelo estadistico}
    En el caso de que se cumplan las siguientes premisas:
    \begin{itemize}
       \item La entrada se distancia de los diferentes niveles de cuantizacion con igual probabilidad
       \item El error de cuantizacion NO esta correlacionado con la entrada
       \item El cuantizador cuanta con un numero relativamente largo de niveles
       \item Los niveles de cuantizacion son uniformes
    \end{itemize}
    Se puede considerar la cuantizacion como un ruido aditivo a la señal segun el siguiente esquema:
    \center\includegraphics[width=0.5\textwidth]{1_clase/noise_model}
    \vfill
 \end{frame}
%-------------------------------------------------------------------------------
 \begin{frame}{Ruido de cuantizacion}{Funcion densidad de probabilidad}
    \begin{columns}[onlytextwidth]
       \column{.5\textwidth}
       \center\includegraphics[width=0.7\textwidth]{1_clase/noise_funcion_probabilidad}
       \column{.5\textwidth}
       \begin{align*}
          \int^\frac{lsb}{2}_{-\frac{lsb}{2}} p(e) de = 1 \\
       \end{align*}
    \end{columns}
    \vfill
 \end{frame}
%-------------------------------------------------------------------------------
 \begin{frame}{Ruido de cuantizacion}{Potencia de ruido de cuantizacion}
    \begin{columns}[onlytextwidth]
       \column{.5\textwidth}
       \begin{align*}
          P_q &= \int^\frac{lsb}{2}_{-\frac{lsb}{2}} e^2 p(e) de \\
          P_q &= \int^\frac{lsb}{2}_{-\frac{lsb}{2}} e^2 \frac{1}{lsb} de \\
          P_q &= \frac{1}{lsb}\left(\frac{e^3}{3} \Big\rvert^{\frac{lsb}{2}}_{-\frac{lsb}{2}}\right)
       \end{align*}
       \column{.5\textwidth}
       \begin{align*}
          P_q &= \frac{1}{lsb}\left(\frac{(\frac{lsb}{2})^3}{3} - \frac{(\frac{-lsb}{2})^3}{3}\right)\\
          P_q &= \frac{1}{lsb} \left(\frac{lsb^3}{24} + \frac{lsb^3}{24}\right) \\
       \end{align*}
    \end{columns}
    \begin{block}{Potencia de ruido de cuantizacion}
       \begin{equation}
          P_q = \frac{lsb^2}{12}
       \end{equation}
    \end{block}
    \vfill
 \end{frame}
%-------------------------------------------------------------------------------
 \subsection{SNR}
 \begin{frame}{Ruido de cuantizacion}{Relacion señal a ruido}
    \begin{columns}[onlytextwidth]
       \column{.5\textwidth}
       \begin{align*}
          input&=\frac{Amp}{2}\sin(t) \\
          P_{input} &= \frac{1}{T} \int^T_0 \left(\frac{Amp}{2}\sin(t)\right)^2 dt \\
          P_{input} &= \frac{1}{T} \left(\frac{Amp}{2}\right)^2* \left( \frac{t}{2}-\frac{\sin(2t)}{4}\right)\Big\rvert^T_0 \\
          P_{input} &= \frac{Amp^2}{4T} \frac{T}{2}\\
          P_{input} &= \frac{Amp^2}{8} \\
       \end{align*}
       \column{.5\textwidth}
       \begin{align*}
          lsb       &= \frac{Amp}{2^N} \\
          P_{ruido} &= \frac{lsb^2}{12}\\
          P_{ruido} &= \frac{\left(\frac{Amp}{2^N}\right)^2}{12}\\
          P_{ruido} &= \frac{Amp^2}{12*2^{2N}}\\
       \end{align*}
    \end{columns}
    \vfill
 \end{frame}
%-------------------------------------------------------------------------------
 \begin{frame}{Ruido de cuantizacion}{Relacion señal a ruido}
    \begin{columns}[onlytextwidth]
       \column{.5\textwidth}
       \begin{align*}
          SNR&=10 \log_{10} \left(\frac{P_{input}}{P_{ruido}} \right)\\
          SNR&=10 \log_{10}\left(\frac{\frac{Amp^2}{8}}{\frac{Amp^2}{12*2^{2N}}} \right) \\
       \end{align*}
       \column{.5\textwidth}
       \begin{align*}
          SNR&=10 \log_{10}\left(\frac{3*2^{2N}}{2} \right)\\
          SNR&=10\log_{10}\left(\frac{3}{2}\right)-10\log_{10}\left(2^{2N}\right)
       \end{align*}
    \end{columns}
    \begin{block}{SNR}
       \begin{equation}
          SNR = 1.76 + 6.02 * N
       \end{equation}
    \end{block}
    \begin{centering}
       \resizebox{0.2\textwidth}{!}{$SNR_{N=10} \approx 60dB$} \\
       \resizebox{0.2\textwidth}{!}{$SNR_{N=11} \approx 66dB$} \\
    \end{centering}
 \end{frame}
%-------------------------------------------------------------------------------
 \begin{frame}{Ruido de cuantizacion}{Densidad espectral de potencia de ruido}
    Si condideramos la potencia de ruido uniformemente distribuido en todo el espectro desde $-Fs$ hasta $+Fs$, nos queda que:
    \begin{block}{Densidad espectral de potencia de ruido}
       \begin{align*}
          S_{espectral}(f) = \frac{P_q}{Fs}
       \end{align*}
    \end{block}
    Entonces como puedo mejorar la SNR de un sistema?
    \vfill
 \end{frame}
%-------------------------------------------------------------------------------
 \subsection{oversampling}
 \begin{frame}{Sobremuestreo}{Densidad espectral de potencia de ruido}
    \begin{columns}[onlytextwidth]
       \column{.5\textwidth}
       Oversampling x4
       \begin{align*}
          S_{espectral}(f) = \frac{P_q}{4*Fs}
       \end{align*}
       \column{.5\textwidth}
       \center\includegraphics[width=0.8\textwidth]{1_clase/oversampling}
    \end{columns}
    \vfill
    Que hago si tengo un AD de 10bits y deseo una SNR de 72dB?
    $SNR_{10}\approx 66dB$
    Pero si sobremuestreo a 4x obtengo \textcolor{green}{6dB} extras
 \end{frame}
%-------------------------------------------------------------------------------
 \section{CIAA}
 \subsection{Acondicionamiento de señal}
 \begin{frame}{Sampleo}{Propuesta para acondicionamiento de señal}
    Acondicionar la señal de salida del dispositivo de sonido (en PC ronda $\pm1V$) al rango del ADC del hardware. En el caso de la CIAA sera de 0-3.3V. \\ 
    Se propone el siguiente circuito, que minimiza los componentes sacrificando calidad y agrega en filtro anti alias de 1er orden.
    \protoboardicon
    \center\includegraphics[width=0.8\textwidth]{1_clase/circuito}
    \vfill
 \end{frame}
%-------------------------------------------------------------------------------
 \begin{frame}{Sampleo}{Propuesta para acondicionamiento de señal}
    Pinout de la CIAA para conectar el ADC/DAC
    \protoboardicon
    \center\includegraphics[width=0.7\textwidth]{1_clase/adc_dac_pins}
    \vfill
 \end{frame}
%-------------------------------------------------------------------------------
 \section{CIAA}
 \subsection{Generacion de audio con Python}
 \begin{frame}{Generacion de audio con Python}{simpleaudio lib}
    \handsonicon
    Como herramienta para la generacion de audio para capturar con la CIAA se propone el uso de simpleaudio con el siguiente template de codigo como base
    \begin{columns}[onlytextwidth]
       \column{.6\textwidth}
       \lstset{ basicstyle=\fontsize{ 8}{ 2}\selectfont\ttfamily }
       \lstinputlisting[language=Python,tabsize=4]{1_clase/audio_gen1.py}
       \column{.4\textwidth}
       \includegraphics[width=\textwidth]{1_clase/audio_gen1.png}
    \end{columns}
    \vfill
 \end{frame}
%-------------------------------------------------------------------------------
 \subsection{Captura con la CIAA}
 \begin{frame}{Captura de audio con la CIAA}{Ciaa->Uart->picocom->log.bin}
    \handsonicon
      Se detalla codigo simple de sampleo y envio por UART
    \begin{columns}[onlytextwidth]
       \column{.6\textwidth}
       \lstset{ basicstyle=\fontsize{ 7}{ 3}\selectfont\ttfamily }
       \lstinputlisting[language=Python,tabsize=4]{1_clase/ciaa/psf1/src/psf.c}
       \column{.4\textwidth}
       \includegraphics[width=\textwidth]{1_clase/ciaa/psf1/log.png}
       \begin{block}{\tiny{picocom /dev/ttyUSB1 -b 460800 --logfile=log.bin}}
       \end{block}
    \end{columns}
    \vfill
 \end{frame}
%-------------------------------------------------------------------------------
 \begin{frame}{Captura de audio con la CIAA}{Uart->Python}
    \handsonicon
    Se detalla la lectura de un log y visualizacion en tiempo real de los datos
       \lstset{ basicstyle=\fontsize{ 5}{ 2}\selectfont\ttfamily }
    \begin{columns}[onlytextwidth]
       \column{.3\textwidth}
       \lstinputlisting[language=Python,tabsize=4,lastline=25]{1_clase/ciaa/psf1/visualize.py}
       \column{.3\textwidth}
       \lstinputlisting[language=Python,tabsize=4,firstline=26]{1_clase/ciaa/psf1/visualize.py}
       \column{.4\textwidth}
       \includegraphics[width=\textwidth]{1_clase/ciaa/psf1/visualize.png}
    \end{columns}
    \vfill
 \end{frame}
%-------------------------------------------------------------------------------
