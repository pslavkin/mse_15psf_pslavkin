%-------------------------------------------------------------------------------
\subtitle{Clase 2 - CIAA<>Python}
\begin{frame}[t]
\maketitle
\begin{tikzpicture}[overlay,remember picture]
    \node[anchor=south east,xshift=-30pt,yshift=45pt]
      at (current page.south east) {
         \includegraphics[width=7cm]{2_clase/log.png}
      };
  \end{tikzpicture}
\end{frame}
%-------------------------------------------------------------------------------
%\begin{frame}
%   \frametitle{Resumen de seccion \thesection}
%   \tiny{\tableofcontents[currentsection]}
% \end{frame}}
%-------------------------------------------------------------------------------
 \section{CIAA}
 \subsection{Acondicionamiento de señal}
 \begin{frame}[t]{Sampleo}{Acondicionamiento de señal}
    Acondicionar la señal de salida del dispositivo de sonido (en PC ronda $\pm1V$) al rango del ADC del hardware. En el caso de la CIAA sera de 0-3.3V. \\ 
    Se propone el siguiente circuito, que minimiza los componentes sacrificando calidad y agrega en filtro anti alias de 1er orden.
    \protoboardicon
    \center\includegraphics[width=9cm]{2_clase/circuito}
    \vfill
 \end{frame}
%-------------------------------------------------------------------------------
 \begin{frame}{Sampleo}{Acondicionamiento de señal}
    Pinout de la CIAA para conectar el ADC/DAC
    \protoboardicon
    \center\includegraphics[width=7cm]{2_clase/adc_dac_pins}
    \vfill
 \end{frame}
%-------------------------------------------------------------------------------
 \section{CIAA}
 \subsection{Generación de audio con Python}
 \begin{frame}[t]{Generación de audio con Python}{simpleaudio lib}
    \handsonicon
    Instalar el modulo simpleaudio para generar sonidos con python
    \tiny
       \href{https://simpleaudio.readthedocs.io/en/latest/installation.html}{https://simpleaudio.readthedocs.io/en/latest/installation.html}
    \normalsize
    Y utilizamos el siguiente código como base:
    \begin{columns}[t]
       \begin{column}{0.5\textwidth}
       \lstinputlisting[basicstyle=\fontsize{8}{2}\selectfont\ttfamily ,language=Python,tabsize=4]{2_clase/audio_gen.py}
    \end{column}
       \begin{column}{0.5\textwidth}
       \center\includegraphics[width=6cm,height=2.4cm]{2_clase/audio_gen1.png}
       \center\includegraphics[width=6cm,height=2.4cm]{2_clase/audio_gen2.png}
    \end{column}
    \end{columns}
   \vfill
 \end{frame}
%-------------------------------------------------------------------------------
 \subsection{Captura con la CIAA}
 \begin{frame}{Captura de audio con la CIAA}{CIAA->UART->picocom->log.bin}
    \handsonicon
    Utilizando picocom
    \tiny
    \href{https://github.com/npat-efault/picocom}{https://github.com/npat-efault/picocom}
    \normalsize
    o similar se graba en un archivo la salida de la UART para luego procesar como sigue
    \begin{block}{\tiny{picocom /dev/ttyUSB1 -b 460800 --logfile=log.bin}}
    \end{block}
    \begin{columns}[onlytextwidth]
       \column{.55\textwidth}
       \lstinputlisting[basicstyle=\fontsize{ 6}{ 2}\selectfont\ttfamily,language=c,tabsize=4]{2_clase/ciaa/psf1/src/psf.c}
       \column{.45\textwidth}
       \includegraphics[width=1.0\textwidth]{2_clase/log.png}
    \end{columns}
    \vfill
 \end{frame}
%-------------------------------------------------------------------------------
 \begin{frame}[t]{Ancho de banda}{}
    \handsonicon
    \begin{columns}[onlytextwidth]
       \column{.55\textwidth}
       \begin{align*}
          USB<>UART_{max bps} &= 460800 bps \\
          Eficacia                 &= \frac{10b}{8b} = 0.8\\
          bits_{muestra}           &= 16 \\
          Tasa_{efectiva}          &= \frac{460800_{bps}*0.8}{16} = 23040
       \end{align*}
          \begin{block}{Maxima señal muestreable y reconstruible}
             \center{11520hz}
          \end{block}
       \column{.45\textwidth}
       \center\includegraphics[width=6cm,height=2.8cm]{2_clase/osci1}
       \center\includegraphics[width=6cm,height=2.8cm]{2_clase/osci2}
    \end{columns}
    \vfill
 \end{frame}
%-------------------------------------------------------------------------------
 \begin{frame}{Sampleo}{Calculo del filtro antialias 1er orden R-C}
    \begin{columns}[onlytextwidth]
       \column{.50\textwidth}
       \begin{align*}
          B         &= 10k bps \\
          f_{corte} &= \frac{1}{2*\pi*R*C} \\
          R         &= 1k\Omega \\
          C         &= \frac{1}{f_{corte}*R*2*\pi} \approx 15nF
       \end{align*}
          \begin{block}{Maxima señal muestreable y reconstruible}
             \center{11520hz}
          \end{block}
       \column{.45\textwidth}
       \includegraphics[width=0.8\textwidth]{2_clase/low_pass}
    \end{columns}
    \vfill
 \end{frame}
%-------------------------------------------------------------------------------
 \begin{frame}{Captura de audio con la CIAA}{Uart->Python}
    \handsonicon
    Lectura de un log y visualización en tiempo real de los datos
       \lstset{ basicstyle=\fontsize{ 6}{ 1}\selectfont\ttfamily,language=Python,tabsize=4}
       \begin{columns}[c]
          \hspace{2pt}
          \begin{column}{.3\textwidth}
             \lstinputlisting[lastline=25]{2_clase/ciaa/psf1/visualize.py}
          \end{column}
          \hspace{2pt}
          \vrule
          \hspace{2pt}
          \begin{column}{.3\textwidth}
             \lstinputlisting[firstline=26]{2_clase/ciaa/psf1/visualize.py}
          \end{column}
          \hspace{2pt}
          \vrule
          \hspace{2pt}
          \begin{column}{.4\textwidth}
             \includegraphics[width=6cm,height=4.8cm]{2_clase/ciaa/psf1/visualize.png}
          \end{column}
          \hspace{2pt}
       \end{columns}
       \vfill
    \end{frame}
%-------------------------------------------------------------------------------
 \section{CIAA}
 \subsection{Acondicionamiento de señal}
 \begin{frame}[t]{Recontruccion}{Acondicionamiento de señal}
    Se realiza un loop del DAC al ADC permitiendo sumar a la señal de entrada ya existente
    \protoboardicon
    \center\includegraphics[width=9cm]{2_clase/circuito_dac}
    \vfill
 \end{frame}
%-------------------------------------------------------------------------------
 \subsection{Generación de señales con el DAC}
 \begin{frame}[t]{Generación de audio con el DAC de la CIAA}{ARM CMSIS-DSP lib \tiny
    \href{https://www.keil.com/pack/doc/CMSIS/DSP/html/group\_\_sin.html}{https://www.keil.com/pack/doc/CMSIS/DSP/html/group\_\_sin.html}
    }
    \protoboardicon
    Con arm\_sin\_f32 se genera un tono y se convierte a analogico con el DAC
    \normalsize
       \lstset{ basicstyle=\fontsize{ 7}{ 5}\selectfont\ttfamily,language=c,tabsize=4}
       \begin{columns}[t]
          \hspace{2pt}
          \begin{column}[c]{.2\textwidth}
             \lstinputlisting[lastline=18]{2_clase/ciaa/psf2/src/psf.c}
          \end{column}
          \hspace{2pt}
          \vrule
          \hspace{2pt}
          \begin{column}[c]{.4\textwidth}
             \lstinputlisting[firstline=19]{2_clase/ciaa/psf2/src/psf.c}
          \end{column}
          \hspace{2pt}
          \vrule
          \hspace{2pt}
          \begin{column}[c]{.4\textwidth}
             \begin{center}
                \includegraphics[width=6cm,height=2.8cm]{2_clase/ciaa_audio_gen1.png} \\
                \includegraphics[width=6cm,height=2.8cm]{2_clase/ciaa_audio_gen2.png}
             \end{center}
          \end{column}
          \hspace{2pt}
       \end{columns}
   \vfill
 \end{frame}
%-------------------------------------------------------------------------------
 \section{Números}
 \begin{frame}{Sistemas de números}{Punto fijo vs punto flotante}
    \begin{columns}[onlytextwidth]
       \column{.5\textwidth}
    Punto fijo:
       \scriptsize{
    \begin{itemize}
       \item{Cantidad de patrones de bits= 65536}
       \item{Gap entre números constante }
       \item{Rango dinámico $32767, -32768$}
       \item{Gap ~10 mil veces mas chico que el numero}
    \end{itemize}
 }
       \column{.5\textwidth}
    Punto flotante:
       \scriptsize{
    \begin{itemize}
       \item{Cantidad de patrones de bits= 4,294,967,296}
       \item{Gap entre números variable }
       \item{Rango dinámico $\pm3.4 e10^{38},  \pm1.2 e10^{-38}$}
       \item{Gap ~10 millones de veces mas chico que el numero}
    \end{itemize}
 }
    \end{columns}
       \center\includegraphics[width=8cm]{2_clase/fix_vs_float}
    \vfill
 \end{frame}
%-------------------------------------------------------------------------------
 \begin{frame}[t]{Sistemas de números}{Sistema Q}
       Qm.n:
       \begin{itemize}
          \item{m: cantidad de bits para la parte entera}
          \item{n: cantidad de bits para la parte decimal}
       \end{itemize}
       Q1.15: \\
          1\textcolor{green}{000 0000 0000 0000} =  -1 \\
          0\textcolor{green}{111 1111 1111 1111} = $1/2+1/4+1/8+..+1/2^{15} = 0.99$ \\
       Q2.14: \\
          10\textcolor{green}{10 0000 0000 0000} =  -1.5 \\
          01\textcolor{green}{01 0000 0000 0000} =  1.25
    \vfill
 \end{frame}
%-------------------------------------------------------------------------------
 \begin{frame}[t]{Sistemas de números}{Sistema Q}
       Qm.n:
       \begin{itemize}
          \item{m: cantidad de bits para la parte entera}
          \item{n: cantidad de bits para la parte decimal}
       \end{itemize}
       Q1.15: \\
          1\textcolor{green}{000 0000 0000 0000} =  -1 \\
          0\textcolor{green}{111 1111 1111 1111} = $1/2+1/4+1/8+..+1/2^{15} = 0.99$ \\
       Q2.14: \\
          10\textcolor{green}{10 0000 0000 0000} =  -1.5 \\
          01\textcolor{green}{01 0000 0000 0000} =  1.25
    \vfill
 \end{frame}
%-------------------------------------------------------------------------------
