%-------------------------------------------------------------------------------
\subtitle{Clase 2 - CIAA<>Python}
\begin{frame}[t]
\maketitle
\begin{tikzpicture}[overlay,remember picture]
    \node[anchor=south east,xshift=-30pt,yshift=45pt]
      at (current page.south east) {
         \includegraphics[width=7cm]{2_clase/log.png}
      };
  \end{tikzpicture}
\end{frame}

%-------------------------------------------------------------------------------
%\begin{frame}
%   \frametitle{Resumen de seccion \thesection}
%   \tiny{\tableofcontents[currentsection]}
%\end{frame}
%-------------------------------------------------------------------------------
 \begin{frame}{Enuestas}
    \framesubtitle{Encuesta anónima clase a clase}
    Propiciamos este espacio para compartir sus sugerencias, criticas constructivas, oportunidades de mejora y cualquier tipo de comentario relacionado a la clase.
    \begin{block}{Encuesta anónima}{
       \includegraphics[width=0.1\textwidth]{1_clase/click}
       \href{https://forms.gle/1j5dDTQ7qjVfRwYo8}{https://forms.gle/1j5dDTQ7qjVfRwYo8}
    }
       \end{block}
    \begin{block}{Link al material de la material}{
       \includegraphics[width=0.1\textwidth]{1_clase/click}
       \tiny{\href{https://drive.google.com/drive/u/1/folders/1TlR2cgDPchL\_4v7DxdpS7pZHtjKq38CK}{https://drive.google.com/drive/u/1/folders/1TlR2cgDPchL\_4v7DxdpS7pZHtjKq38CK}
    }
    }
       \end{block}
\end{frame}
%-------------------------------------------------------------------------------
 \section{CIAA}
 \subsection{Acondicionamiento de señal}
 \begin{frame}[t]{Sampleo}{Acondicionamiento de señal}
    Acondicionar la señal de salida del dispositivo de sonido (en PC ronda $\pm1V$) al rango del ADC del hardware. En el caso de la CIAA sera de 0-3.3V. \\ 
    Se propone el siguiente circuito, que minimiza los componentes sacrificando calidad y agrega en filtro anti alias de 1er orden.
    \protoboardicon
    \center\includegraphics[width=9cm]{2_clase/circuito}
    \vfill
 \end{frame}
%-------------------------------------------------------------------------------
 \begin{frame}{Sampleo}{Acondicionamiento de señal}
    Pinout de la CIAA para conectar el ADC/DAC
    \protoboardicon
    \center\includegraphics[width=7cm]{2_clase/adc_dac_pins}
    \vfill
 \end{frame}
%-------------------------------------------------------------------------------
 \section{CIAA}
 \subsection{Generación de audio con Python}
    
 \begin{frame}[t]{Generación de audio con Python}{\tiny{\href{https://simpleaudio.readthedocs.io/en/latest/installation.html}{https://simpleaudio.readthedocs.io/en/latest/installation.html}}}
    \handsonicon
    \normalsize
    Instalar el módulo simpleaudio \tiny{(ver apéndice \ref{appendix:simpleaudio})} \normalsize para generar sonidos con python y utilizamos el siguiente código como base:
    \begin{columns}[t]
       \begin{column}{0.5\textwidth}
       \lstinputlisting[basicstyle=\fontsize{8}{2}\selectfont\ttfamily ,language=Python,tabsize=4]{2_clase/audio_gen.py}
    \end{column}
       \begin{column}{0.5\textwidth}
       \center\includegraphics[width=6cm,height=2.4cm]{2_clase/audio_gen1.png}
       \center\includegraphics[width=6cm,height=2.4cm]{2_clase/audio_gen2.png}
    \end{column}
    \end{columns}
   \vfill
 \end{frame}
%-------------------------------------------------------------------------------
 \subsection{Captura con la CIAA}
 \begin{frame}{Captura de audio con la CIAA}{CIAA->UART->picocom->log.bin}
    \handsonicon
    Utilizando picocom
    \tiny
    \href{https://github.com/npat-efault/picocom}{https://github.com/npat-efault/picocom}
    \normalsize
    o similar se graba en un archivo la salida de la UART para luego procesar como sigue
    \begin{block}{\tiny{picocom /dev/ttyUSB1 -b 460800 --logfile=log.bin}}
    \end{block}
    \begin{columns}[onlytextwidth]
       \column{.55\textwidth}
       \lstinputlisting[basicstyle=\fontsize{ 6}{ 2}\selectfont\ttfamily,language=c,tabsize=4]{2_clase/ciaa/psf1/src/psf.c}
       \column{.45\textwidth}
       \includegraphics[width=1.0\textwidth]{2_clase/log.png}
    \end{columns}
    \vfill
 \end{frame}
%-------------------------------------------------------------------------------
 \begin{frame}[t]{Ancho de banda}{}
    \handsonicon
    \begin{columns}[onlytextwidth]
       \column{.55\textwidth}
       \begin{align*}
          USB<>UART_{max bps} &= 460800 bps \\
          Eficacia                 &= \frac{10b}{8b} = 0.8\\
          bits_{muestra}           &= 16 \\
          Tasa_{efectiva}          &= \frac{460800_{bps}*0.8}{16} = 23040
       \end{align*}
          \begin{block}{Máxima señal muestreable y reconstruible}
             \center{11520hz}
          \end{block}
       \column{.45\textwidth}
       \center\includegraphics[width=6cm,height=2.8cm]{2_clase/osci1}
       \center\includegraphics[width=6cm,height=2.8cm]{2_clase/osci2}
    \end{columns}
    \vfill
 \end{frame}
%-------------------------------------------------------------------------------
 \begin{frame}{Sampleo}{Calculo del filtro antialias 1er orden R-C}
    \begin{columns}[onlytextwidth]
       \column{.50\textwidth}
       \begin{align*}
          B         &= 10k bps \\
          f_{corte} &= \frac{1}{2*\pi*R*C} \\
          R         &= 1k\Omega \\
          C         &= \frac{1}{f_{corte}*R*2*\pi} \approx 15nF
       \end{align*}
          \begin{block}{Máxima señal muestreable y reconstruible}
             \center{11520hz}
          \end{block}
       \column{.45\textwidth}
       \includegraphics[width=0.8\textwidth]{2_clase/low_pass}
    \end{columns}
    \vfill
 \end{frame}
%-------------------------------------------------------------------------------
 \begin{frame}{Captura de audio con la CIAA}{UART->Python}
    \handsonicon
    Lectura de un log y visualización en tiempo real de los datos
       \lstset{ basicstyle=\fontsize{ 6}{ 1}\selectfont\ttfamily,language=Python,tabsize=4}
       \begin{columns}[c]
          \hspace{2pt}
          \begin{column}{.3\textwidth}
             \lstinputlisting[lastline=25]{2_clase/ciaa/psf1/visualize.py}
          \end{column}
          \hspace{2pt}
          \vrule
          \hspace{2pt}
          \begin{column}{.3\textwidth}
             \lstinputlisting[firstline=26]{2_clase/ciaa/psf1/visualize.py}
          \end{column}
          \hspace{2pt}
          \vrule
          \hspace{2pt}
          \begin{column}{.4\textwidth}
             \includegraphics[width=6cm,height=4.8cm]{2_clase/ciaa/psf1/visualize.png}
          \end{column}
          \hspace{2pt}
       \end{columns}
       \vfill
    \end{frame}
%-------------------------------------------------------------------------------
 \section{CIAA}
 \subsection{Acondicionamiento de señal}
 \begin{frame}[t]{Reconstrucción}{Acondicionamiento de señal}
    Se realiza un loop del DAC al ADC permitiendo sumar a la señal de entrada ya existente
    \protoboardicon
    \center\includegraphics[width=9cm]{2_clase/circuito_dac}
    \vfill
 \end{frame}
%-------------------------------------------------------------------------------
 \section{Generación de señales con el DAC}
%-------------------------------------------------------------------------------
 \subsection{Generación de señales con el DAC}
 \begin{frame}[t]{Generación de audio con el DAC de la CIAA}{}
    \protoboardicon
    \videoicon{2}{54m18s}
    \footnotesize
    \begin{itemize}
       \item{Con arm\_sin\_f32 se genera un tono y se convierte a analógico con el DAC}
       \item{Se utiliza visualize.py del lado de la PC para graficar lo recibido}
       \item{Se samplean 512 (LENGTH) datos y luego mientras se envían por la uart se siguen capturando}
       \item{De esta manera no se pierden samples y se logra reconstruir en la PC la señal completa sin cortes}
    \end{itemize}

    \begin{columns}[t]
       \hspace{2pt}
       \begin{column}[c]{.5\textwidth}
          \pythonpic{2\_clase/ciaa/psf2/src/psf.c}
          {https://drive.google.com/open?id=1lGD5CJoPD1GmGyu78g_W9f37QP9JPP6Z}
          {0.7}
          {2_clase/audio_gen1}
       \end{column}
       \hspace{2pt}
       \vrule
       \hspace{2pt}
       \begin{column}[c]{.5\textwidth}
          \pythonpic{2\_clase/ciaa/psf2/visualize.py}
          {https://drive.google.com/open?id=1vgw68wDJ6770iPy_t5rtMYXXovEY7wXw}
          {0.7}
          {2_clase/audio_gen2}
       \end{column}
       \hspace{2pt}
    \end{columns}
    \vfill
 \end{frame}
%-------------------------------------------------------------------------------
 \subsection{CMSIS}
 \begin{frame}[t]{Documentación}{ARM CMSIS-DSP lib \tiny
    \href{https://www.keil.com/pack/doc/CMSIS/DSP/html/group\_\_sin.html}{https://www.keil.com/pack/doc/CMSIS/DSP/html/group\_\_sin.html}
    }
    \videoicon{2}{58m41s}
    \footnotesize
    \begin{columns}[t]
       \hspace{2pt}
       \begin{column}[c]{.4\textwidth}
    \begin{itemize}
       \item{CMSIS-DSP Cuenta con una buena documentation on line de la API}
       \item{Es muy importante consultar la documentación de las funciones que se dispongan a usar para evitar errores}
       \item{Esta biblioteca esta portada para muchas familias de procesadores, permitiendo la reutilización de código}
    \end{itemize}
       \end{column}
       \hspace{2pt}
       \vrule
       \hspace{2pt}
       \begin{column}[c]{.6\textwidth}
          \center\includegraphics[width=0.85\textwidth]{2_clase/cmsis_page}
       \end{column}
       \hspace{2pt}
    \end{columns}
 \end{frame}
%-------------------------------------------------------------------------------
 \section{Números formato Q}
 \begin{frame}{Sistemas de números}{Punto fijo vs punto flotante}
    \begin{columns}[onlytextwidth]
       \column{.5\textwidth}
    Punto fijo:
       \scriptsize{
    \begin{itemize}
       \item{Cantidad de patrones de bits= 65536}
       \item{Gap entre números constante }
       \item{Rango dinámico $32767, -32768$}
       \item{Gap ~10 mil veces mas chico que el numero}
    \end{itemize}
 }
       \column{.5\textwidth}
    Punto flotante:
       \scriptsize{
    \begin{itemize}
       \item{Cantidad de patrones de bits= 4,294,967,296}
       \item{Gap entre números variable }
       \item{Rango dinámico $\pm3.4 e10^{38},  \pm1.2 e10^{-38}$}
       \item{Gap ~10 millones de veces mas chico que el numero}
    \end{itemize}
 }
    \end{columns}
       \center\includegraphics[width=8cm]{2_clase/fix_vs_float}
    \vfill
 \end{frame}
%-------------------------------------------------------------------------------
 \begin{frame}[t]{Sistemas de números}{Sistema Q}
       Qm.n:
       \begin{itemize}
          \item{m: cantidad de bits para la parte entera}
          \item{n: cantidad de bits para la parte decimal}
       \end{itemize}
       Q1.15: \\
          1\textcolor{green}{000 0000 0000 0000} =  -1 \\
          0\textcolor{green}{111 1111 1111 1111} = $1/2+1/4+1/8+..+1/2^{15} = 0.99$ \\
       Q2.14: \\
          10\textcolor{green}{10 0000 0000 0000} =  -1.5 \\
          01\textcolor{green}{01 0000 0000 0000} =  1.25
    \vfill
 \end{frame}
%-------------------------------------------------------------------------------
 \begin{frame}[t]{Sistemas de números}{Tabla de ejemplos Q1.2 y Q2.1 signado y no signado }
    \scriptsize
    \begin{columns}[t]
       \begin{column}[t]{.5\textwidth}
          \begin{table}[]
             \begin{tabular}{|l|l|l|}
             \hline
                UQ3.0   &        UQ2.1      &     UQ1.2             \\
             \hline
                011 = 3 & 01.1 = 1+1/2= 1.5 & 0.11 = 0+1/2+1/4= 0.75\\
                010 = 2 & 01.0 = 1+0/2= 1.0 & 0.10 = 0+1/2+0/4= 0.5 \\
                001 = 1 & 00.1 = 0+1/2= 0.5 & 0.01 = 0+0/2+1/4= 0.25\\
                000 = 0 & 00.0 = 0+0/2= 0.0 & 0.00 = 0+0/2+0/4= 0.0 \\
                111 = 7 & 11.1 = 3+1/2= 3.5 & 1.11 = 1+1/2+1/4= 1.75\\
                110 = 6 & 11.0 = 3+0/2= 3.0 & 1.10 = 1+1/2+0/4= 1.5 \\
                101 = 5 & 10.1 = 2+1/2= 2.5 & 1.01 = 1+0/2+1/4= 1.25\\
                100 = 4 & 10.0 = 2+0/2= 2.0 & 1.00 = 1+0/2+0/4= 1.0 \\
             \hline
             \end{tabular}
          \end{table}
       \end{column}
       \hspace{5pt}
       \hspace{5pt}
       \begin{column}[t]{.5\textwidth}
          \begin{table}[]
             \begin{tabular}{|l|l|l|}
             \hline
                SQ3.0   &        SQ2.1      &     SQ1.2             \\
             \hline
                011 =+3 & 01.1 = 1+1/2=+1.5 & 0.11 = 0+1/2+1/4=+0.75\\
                010 =+2 & 01.0 = 1+0/2=+1.0 & 0.10 = 0+1/2+0/4=+0.5 \\
                001 =+1 & 00.1 = 0+1/2=+0.5 & 0.01 = 0+0/2+1/4=+0.25\\
                000 =+0 & 00.0 = 0+0/2=+0.0 & 0.00 = 0+0/2+0/4=+0.0 \\
                111 =-1 & 11.1 =-1+1/2=-0.5 & 1.11 =-1+1/2+1/4=-0.25\\
                110 =-2 & 11.0 =-1+0/2=-1.0 & 1.10 =-1+1/2+0/4=-0.5 \\
                101 =-3 & 10.1 =-2+1/2=-1.5 & 1.01 =-1+0/2+1/4=-0.75\\
                100 =-4 & 10.0 =-2+0/2=-2.0 & 1.00 =-1+0/2+0/4=-1.0 \\
             \hline
             \end{tabular}
          \end{table}
       \end{column}
    \end{columns}
    \vfill
   \note{
      \begin{itemize}
         \item{explicar mejore esto de los q que trajo problemas, ace dejo un ejemplo tabulado bastante util}
         \item{pasar el link de bibliografia para que lo usen}
      \end{itemize}
   }
 \end{frame}
%-------------------------------------------------------------------------------
 \begin{frame}[t]{Sistemas de números}{Float32 IEEE 754}
       \center\includegraphics[width=0.65\textwidth]{2_clase/float32_example}
    \vfill
 \end{frame}
%-------------------------------------------------------------------------------
 \begin{frame}[t]{Calculo de propiedades temporales}{ARM CMSIS-DSP lib}
    \protoboardicon
    \footnotesize{Calculamos max, min y rms con la CIAA}
       \lstset{ basicstyle=\fontsize{ 5}{ 1}\selectfont\ttfamily,language=c,tabsize=4}
       \begin{columns}[t]
          \begin{column}[t]{.5\textwidth}
             \lstinputlisting{2_clase/ciaa/psf3/src/psf.c}
          \end{column}
          \begin{column}[t]{.5\textwidth}
                \tiny{10 bits}\\
                   \includegraphics[width=5cm]{2_clase/max_min_rms_10b.png}\\
                \tiny{4 bits}\\
                   \includegraphics[width=5cm]{2_clase/max_min_rms_4b.png} \\
          \end{column}
       \end{columns}
 \end{frame}
%-------------------------------------------------------------------------------
 \begin{frame}[t]{Cálculo de propiedades temporales}{ARM CMSIS-DSP lib}
    \handsonicon
    \footnotesize{Visualizamos max, min y rms con Python}
       \lstset{ basicstyle=\fontsize{ 5}{ 1}\selectfont\ttfamily,language=python,tabsize=4}
       \begin{columns}[t]
          \hspace{2pt}
          \begin{column}[t]{.4\textwidth}
             \lstinputlisting[lastline=27]{2_clase/ciaa/psf3/visualize.py}
          \end{column}
          \hspace{5pt}
          \vrule
          \hspace{5pt}
          \begin{column}[t]{.6\textwidth}
             \lstinputlisting[firstline=28]{2_clase/ciaa/psf3/visualize.py}
          \end{column}
       \end{columns}
 \end{frame}
%-------------------------------------------------------------------------------
 \begin{frame}{Bibliografía}
    \framesubtitle{Libros, links y otro material}
    \tiny{
       \begin{thebibliography}{1}
             \bibitem{wikipedia}
             \emph{Numeracion Q}. \\
             \href {https://en.wikipedia.org/wiki/q_(number_format)}{https://en.wikipedia.org/wiki/q\_(number\_format)}
             \bibitem{wikipedia}
             \emph{Calculador float on-line}. \\
             \href {https://www.binaryconvert.com/result\_float.html?decimal=048046053}{https://www.binaryconvert.com/result\_float.html?decimal=048046053}
             \href {https://www.h-schmidt.net/FloatConverter/IEEE754.html}{https://www.h-schmidt.net/FloatConverter/IEEE754.html}
             \bibitem{calculador on-line}
             \emph{Calculando numeros Q}. \\
             \href {https://www.rfwireless-world.com/calculators/floating-vs-fixed-point-converter.html}{https://www.rfwireless-world.com/calculators/floating-vs-fixed-point-converter.html}
       \end{thebibliography}
    }
 \end{frame}
%-------------------------------------------------------------------------------
\appendix
\section{Appendix}
\label{appendix:simpleaudio}
\begin{frame}[c]{Apéndice}{Instrucciones para usar simpleaudio}
   \begin{center}
      \includegraphics[       width=0.3\textwidth]{2_clase/simpleaudio_install}
      \includegraphics[page=1,width=0.3\textwidth]{2_clase/simpleaudio_tutorial}
      \includegraphics[page=2,width=0.3\textwidth]{2_clase/simpleaudio_tutorial}
   \end{center}
\end{frame}
\begin{frame}[c]{Apéndice}{Instrucciones para usar simpleaudio}
   \begin{center}
      \includegraphics[page=3,width=0.3\textwidth]{2_clase/simpleaudio_tutorial}
      \includegraphics[page=4,width=0.3\textwidth]{2_clase/simpleaudio_tutorial}
      \includegraphics[page=5,width=0.3\textwidth]{2_clase/simpleaudio_tutorial}
   \end{center}
\end{frame}
