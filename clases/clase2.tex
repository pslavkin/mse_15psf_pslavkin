\subtitle{Clase 2 - Euler | Fourier}


\begin{frame}[c]
\maketitle
\begin{tikzpicture}[overlay,remember picture]
    \node[anchor=south east,xshift=-30pt,yshift=45pt]
      at (current page.south east) {
        \includegraphics[width=0.2\textwidth]{2_clase/leonhard_euler}
        \includegraphics[width=0.2\textwidth]{2_clase/joseph_fourier}
      };
  \end{tikzpicture}
\end{frame}
%-------------------------------------------------------------------------------
\begin{frame}{2.7182818284590450907955982984276488423347473144}
   \begin{columns}[onlytextwidth]
      \column{.25\textwidth}
      \begin{itemize}
         \item $f(t) =t$
         \item $f(t) =t^2$
         \item $f(t) =sin(t)$
      \end{itemize}
      \column{.25\textwidth}
      \begin{itemize}
         \item $f'(t) = 1$
         \item $f'(t) = 2*t$
         \item $f'(t) = cos(t)$
      \end{itemize}
      \column{.4\textwidth}
      \includegraphics[width=1.0\textwidth]{2_clase/derivadas}
      %codigos en 2_clase/derivadas.py
   \end{columns}
   \begin{block}{La derivada es igual a la funcion}
      \begin{center}
         $f(t) =e^t    \implies f'(t) = e^t$\\
         $f(t) =e^{kt} \implies f'(t) =ke^{kt}$
      \end{center}
   \end{block}
   \vfill
\end{frame}
%-------------------------------------------------------------------------------
\begin{frame}{$e^{j2\pi t}$}{Euler}
   Pero que pasa con $\LARGE\textcolor{green}{e^{jt}}$?
   \begin{columns}[onlytextwidth]
      \column{.5\textwidth}
   \begin{block}{La derivada es igual a la funcion}
      \begin{center}
         $f(t) =e^{jt} \implies f'(t) =je^{jt}$
      \end{center}
   \end{block}
      \begin{align*}
         e^{jt}   &= \cos(t) + j \sin(t)\\
         e^{j\pi} &= -1\\
         e^{\frac{\pi}{2}} &= i\\
         e^{\frac{j3\pi}{2}} &= i\\
      \end{align*}
      \column{.5\textwidth}
      \centering\includegraphics[width=0.8\textwidth]{2_clase/euler_senos_cosenos}
   \end{columns}
   \vfill
\end{frame}

%-------------------------------------------------------------------------------
\begin{frame}{$e^{j2\pi t}$}{$e^{j2\pi t}$ animado }
   \handsonicon
   \begin{columns}[onlytextwidth]
      \column{.6\textwidth}
      \lstset{ basicstyle=\fontsize{ 5}{ 3}\selectfont\ttfamily }
      \lstinputlisting[language=Python,tabsize=4]{2_clase/euler1.py}
      \column{.4\textwidth}
      \centering\includegraphics[width=1.0\textwidth]{2_clase/euler1}
   \end{columns}
   \vfill
\end{frame}
%-------------------------------------------------------------------------------
\begin{frame}{$e^{j2\pi t}$}{$e^{j2\pi t}$ y $\sin(t)$ animados independientemente}
   \handsonicon
   \begin{columns}[onlytextwidth]
      \column{.4\textwidth}
      \lstset{ basicstyle=\fontsize{ 4}{ 1}\selectfont\ttfamily }
      \lstinputlisting[language=Python,tabsize=4]{2_clase/euler2.py}
      \column{.6\textwidth}
      \centering\includegraphics[width=1.0\textwidth]{2_clase/euler2}
   \end{columns}
   \vfill
\end{frame}
%-------------------------------------------------------------------------------
\begin{frame}{$e^{j2\pi t}$}{$e^{j2\pi t}$ modulado por $\sin(t)$ y centro de masas }
   \handsonicon
   \begin{columns}[onlytextwidth]
      \column{.4\textwidth}
      \lstset{ basicstyle=\fontsize{ 3}{ 0}\selectfont\ttfamily }
      \lstinputlisting[language=Python,tabsize=4]{2_clase/euler3.py}
      \column{.6\textwidth}
      \centering\includegraphics[width=1.0\textwidth]{2_clase/euler3}
   \end{columns}
   \vfill
\end{frame}
%-------------------------------------------------------------------------------
\begin{frame}{$e^{j2\pi t}$}{$e^{j2\pi t}$ modulado por $\sin(t)$ y centro de masas en f}
   \handsonicon
   \begin{columns}[onlytextwidth]
      \column{.4\textwidth}
      \lstset{ basicstyle=\fontsize{ 2}{ 0}\selectfont\ttfamily }
      \lstinputlisting[language=Python,tabsize=4]{2_clase/euler4.py}
      \column{.6\textwidth}
      \centering\includegraphics[width=1.0\textwidth]{2_clase/euler4}
   \end{columns}
   \vfill
\end{frame}

%-------------------------------------------------------------------------------
\begin{frame}{$e^{j2\pi t}$}{$e^{j2\pi t}$ del centro de masas de vuelta a $\sin(t)$}
   \handsonicon
   \begin{columns}[onlytextwidth]
      \lstset{ basicstyle=\fontsize{ 2}{ 1}\selectfont\ttfamily }
      \column{.2\textwidth}
      \lstinputlisting[language=Python,tabsize=4,lastline=45]{2_clase/euler5.py}
      \column{.2\textwidth}
      \lstinputlisting[language=Python,tabsize=4,firstline=46]{2_clase/euler5.py}
      \column{.6\textwidth}
      \centering\includegraphics[width=1.0\textwidth]{2_clase/euler5}
   \end{columns}
   \vfill
\end{frame}

%-------------------------------------------------------------------------------
\begin{frame}{$e^{j2\pi t}$}{Señales complejas, conejo}
   \handsonicon
   \begin{columns}[onlytextwidth]
      \lstset{ basicstyle=\fontsize{ 2}{ 1}\selectfont\ttfamily }
      \column{.2\textwidth}
      \lstinputlisting[language=Python,tabsize=4,lastline=50]{2_clase/euler6.py}
      \column{.2\textwidth}
      \lstinputlisting[language=Python,tabsize=4,firstline=51]{2_clase/euler6.py}
      \column{.6\textwidth}
      \centering\includegraphics[width=1.0\textwidth]{2_clase/euler6}
   \end{columns}
   \vfill
\end{frame}
%-------------------------------------------------------------------------------
\begin{frame}{Bibliografia}
   \framesubtitle{Libros, links y otro material}
   \begin{thebibliography}{9}
         \bibitem{dsp}
         Steven W. Smith.
         \emph{The Scientist and Engineer's Guide to Digital Signal Processing}.
         Second Edition, 1999.
         \bibitem{link}
         \emph{Interactive Mathematics Site Info}.
         \bibitem{youtube}
         Grant Sanderson
         \emph{ https://youtu.be/spUNpyF58BY}
   \end{thebibliography}
\end{frame}

