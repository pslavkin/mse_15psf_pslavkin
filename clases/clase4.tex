\subtitle{Clase 4 - Euler | Fourier - IDFT \\
Convolucion}
\begin{frame}[c]
\maketitle
\begin{tikzpicture}[overlay,remember picture]
    \node[anchor=south east,xshift=-30pt,yshift=45pt]
      at (current page.south east) {
        \includegraphics[width=0.2\textwidth]{4_clase/leonhard_euler}
        \includegraphics[width=0.2\textwidth]{4_clase/joseph_fourier}
      };
  \end{tikzpicture}
\end{frame}
%-------------------------------------------------------------------------------
\begin{frame}{repaso DFT}{$e^{j2\pi ft}$ modulado por $\sin(t)$ y centro de masas en f, DFT?}
   \videoicon{4}{04m50s}
   \handsonicon
   \footnotesize
   \begin{columns}[c]
      \hspace{2pt}
      \begin{column}{.4\textwidth}
         \begin{itemize}
            \item{Correr el cofigo y notar como el promedio del circulo que gira multiplicado por la señal para diferentes frecuencias nos da la DFT}
            \item{Las frecuencias van de -Fs/2 inclusive a Fs/2 NO inclusive}
            \item{Para una señal de N puntos separados 1/fs segundos se obtienen N puntos en la DFT separados fs/N hz}
            \item{El modulo de cada bin de la DFT al cuadrado es la potencia de la señal en esa frecuencia}
         \end{itemize}
      \end{column}
      \hspace{2pt}
      \vrule
      \hspace{2pt}
      \begin{column}{.6\textwidth}
         \pythonpic{3\_clase/euler4.py}
         {https://drive.google.com/open?id=1RzddFg9YlSVSb1WGYhcaEzMU2MqXYk2p}
         {0.9}
         {3_clase/euler4}
      \end{column}
      \hspace{2pt}
   \end{columns}
   \vfill
   \note{
      \begin{itemize}
         \item{retomar DFT con el concepto de centro de masas}
         \item{lanzar euler4.py de la clase3 con dft en modo rectangular}
         \item{destacar DFT parecido entre la señal y el circulo y viceversa}
         \item{Por lo tanto lo que tenemos son relojitos girando}
      \end{itemize}
   }
\end{frame}
%-------------------------------------------------------------------------------
\begin{frame}{Repaso DFT}{Analisis, Transformada discreta de Fourier}
   \videoicon{4}{07m48s}
   \begin{itemize}
      \item{Definicion formal de la DFT}
      \item{Notar que hay un faltor de escala que puede ser 1/N, $1/\sqrt{N}$ o 1 dependiendo de cada implementacion}
      \item{Tambien podria optarse por cambiar el signo del exponente, lo importante es mantener luego la dualidad entre el factor de escala y el signo con la IDFT}
   \end{itemize}
   \center\includegraphics[width=1.0\textwidth]{3_clase/dft_eq}
   \vfill
\end{frame}
%-------------------------------------------------------------------------------
\begin{frame}{Síntesis}{Como reconstruyo la señal en el tiempo}
   \videoicon{4}{10m45s}
   \handsonicon
   \footnotesize
   \begin{columns}[c]
      \hspace{2pt}
      \begin{column}{.4\textwidth}
         \begin{itemize}
            \item{La IDFT toma como entrada un vector de N numeros complejos}
            \item{Se le llama sintesis a la reconstruccion de la señal en el tiempo}
            \item{DFT es completamente dual con la IDFT}
            \item{Se puede convertir una señal de t a f y f a t usando DFT e IDFT sin perder informacion}
         \end{itemize}
      \end{column}
      \hspace{2pt}
      \vrule
      \hspace{2pt}
      \begin{column}{.6\textwidth}
         \pythonpic{4\_clase/euler6.py}
         {https://drive.google.com/open?id=1bSF5PvXLA_V17erQx0xE0rOrpLLjlktx}
         {0.9}
         {4_clase/euler6}
      \end{column}
      \hspace{2pt}
   \end{columns}
   \note{
      \begin{itemize}
         \item{lanzar euler6 y explicar que pasa con sin y cos}
         \item{explicar el concepto de relojes}
         \item{explicar el concepto de frecuencia negativa}
         \item{comentar lo de hermitica cuando la salida es real y dft complejo conjugado}
         \item{de nuevo la formula la calculamos a mano y arrivamos a lo que viene en la filmina siguiente}
      \end{itemize}
   }
   \vfill
\end{frame}
%-------------------------------------------------------------------------------
\begin{frame}{Síntesis, Transformada inversa discreta de Fourier}{IDFT}
   \videoicon{4}{23m42s}
   \begin{itemize}
      \item{La IDFT es la suma de todos los bines en f evalueados en cada n}
      \item{La IDFT toma N numeros complejos y devuelve N numeros complejos}
      \item{Si el espectro es hermitico, la salida de la IDFT son N numeros reales}
      \item{El espectro es hermitico si los valores X(f) para f positivos son complejos conjugados de los valores de X(f) para f negativos}
      \item{Notar que para la DFT el exponente es negativo y para la IDFT es positivo}
   \end{itemize}
   \center\includegraphics[width=0.9\textwidth]{4_clase/idft_eq}
   \vfill
   \note{
      \begin{itemize}
         \item{comentar tambien lo del 1/N, 1/$\sqrt[2]{N}$}
      \end{itemize}
   }
 \end{frame}
%-------------------------------------------------------------------------------
\begin{frame}{IDFT}{Señales complejas como entrada?}
   \videoicon{4}{48m40s}
   \handsonicon
   \footnotesize
   \begin{columns}[c]
      \hspace{2pt}
      \begin{column}{.4\textwidth}
         \begin{itemize}
            \item{La DFT puede recibir datos complejos}
            \item{La interpretacion en el tiempo de los datos complejos es arbitraria}
            \item{EL campo real podria ser audio y el imaginario temperatura por ej.}
            \item{La salida de la DFT que recibe N datos complejos, tambien devuelve N datos complejos}
            \item{Podria analizar 2 señales en un mismo calculo}
         \end{itemize}
      \end{column}
      \hspace{2pt}
      \vrule
      \hspace{2pt}
      \begin{column}{.6\textwidth}
         \pythonpic{4\_clase/euler7.py}
         {https://drive.google.com/open?id=1yWHckVDOa2_3th_UE0Gsv2kFNzr96R50}
         {0.9}
         {4_clase/euler7}
      \end{column}
      \hspace{2pt}
   \end{columns}
   \vfill
   \note{
      \begin{itemize}
         \item{lanzar euler7}
         \item{hacer algunas pruebas con diferentes valores.}
         \item{destacar que esto es la DFT compleja de lado a lado}
      \end{itemize}
   }
\end{frame}
%-------------------------------------------------------------------------------
\begin{frame}{DFT<>IDFT}{Transformadas relevantes}
   \videoicon{4}{56m48s}
   \begin{columns}[c]
      \begin{column}{.5\textwidth}
         Delta
         \centering\includegraphics[width=1.0\textwidth]{4_clase/euler_delta}
      \end{column}
      \begin{column}{.5\textwidth}
         Cuadrada
         \centering\includegraphics[width=1.0\textwidth]{4_clase/euler_cuadrada}
      \end{column}
   \end{columns}
   \vfill
   \note{
      \begin{itemize}
         \item{lanzar euler9}
         \item{jugar un poco con la cuadrada y ver como se va formando con los diferentes armonicos}
         \item{explicar la idea de compresion, mp3, etc cortando ancho de banda}
         \item{destacar lo de los armonicos 1f, 3f, 5f para la cuadrada }
         \item{probar la delata y mostrar como tiene todas las frecuencias}
      \end{itemize}
   }
\end{frame}
%-------------------------------------------------------------------------------
\begin{frame}{DFT<>IDFT}{Transformadas relevantes}
   \videoicon{4}{1h08m10s}
   \begin{columns}[c]
      \begin{column}{.5\textwidth}
         Triangular
         \centering\includegraphics[width=1.0\textwidth]{4_clase/euler_triangular}
      \end{column}
      \begin{column}{.5\textwidth}
         Conjugado
         \centering\includegraphics[width=1.0\textwidth]{4_clase/euler_conjugado}
      \end{column}
   \end{columns}
   \vfill
   \note{
      \begin{itemize}
         \item{lanzar el euler4 con los conjugados y ver la dualidad}
         \item{jugar con el conjugado y mostar como sale la senoide}
         \item{hablar del tema de dualidad}
      \end{itemize}
   }
\end{frame}
%-------------------------------------------------------------------------------
\begin{frame}{DFT<>IDFT}{Transformadas relevantes}
   \videoicon{4}{1h22m55s}
   \begin{columns}[c]
      \begin{column}{.33\textwidth}
         Sin
         \centering\includegraphics[width=1.0\textwidth]{4_clase/equivalencias_seno}
      \end{column}
      \begin{column}{.33\textwidth}
         Delta
         \centering\includegraphics[width=1.0\textwidth]{4_clase/equivalencias_delta}
      \end{column}
      \begin{column}{.33\textwidth}
         Cuadrada
         \centering\includegraphics[width=1.0\textwidth]{4_clase/equivalencias_cuadrada}
      \end{column}
   \end{columns}
   \vfill
\end{frame}
%-------------------------------------------------------------------------------
\begin{frame}{IDFT}{Un conejo como entrada?}
   \videoicon{4}{1h16m40s}
   \handsonicon
   \begin{columns}[c]
      \hspace{2pt}
      \begin{column}{.4\textwidth}
         \begin{itemize}
            \item{Ejemplo en donde la entrada a la DFT es compleja y representa los puntos de una figura en 2 dimensiones}
         \end{itemize}
      \end{column}
      \hspace{2pt}
      \vrule
      \hspace{2pt}
      \begin{column}{.6\textwidth}
         \pythonpic{4\_clase/euler8.py}
         {https://drive.google.com/open?id=1WaWuAapTNRhArLByWmb_k3GjKNy6cUbN}
         {0.9}
         {4_clase/euler8}
      \end{column}
      \hspace{2pt}
   \end{columns}
   \vfill
   \note{
      \begin{itemize}
         \item{lanzar euler8.py y hablar de que todo se puede transformar}
         \item{volver a mostara la idea de compresion limitando la idft en ancho de banda}
      \end{itemize}
   }
\end{frame}
%-------------------------------------------------------------------------------
\begin{frame}{FFT-IFFT}{Transformadas usando FFT en Python}
   \videoicon{4}{1h26m00s}
   \handsonicon
   Ejemplo de codigo de como calcular la FFT e IFFT usando la biblioteca numpy
   \lstset{ basicstyle=\fontsize{ 5}{ 0}\selectfont\ttfamily,language=Python,tabsize=4}
   \begin{columns}[c]
      \hspace{2pt}
      \begin{column}{.55\textwidth}
         \lstinputlisting[lastline=33]{4_clase/fft1.py}
      \end{column}
      \hspace{2pt}
      \vrule
      \hspace{2pt}
      \begin{column}{.45\textwidth}
         \centering\includegraphics[width=0.9\textwidth]{4_clase/fft1}
      \end{column}
   \end{columns}
   \vfill
   \note{
      \begin{itemize}
         \item{lanzar fft1}
         \item{verificar los mismos resultados que la maquina animada}
      \end{itemize}
   }
\end{frame}
%-------------------------------------------------------------------------------
\section{Respuesta al impulso}
\begin{frame}{Cambio de tema?}{}
   \videoicon{4}{1h50m40s}
   \begin{enumerate}
      \item{Función delta}
      \item{Respuesta al impulso}
      \item{Convolución}
   \end{enumerate}
   \vfill
   \note{
      \begin{itemize}
         \item{Corte porque freno con DFT y pasamos a tiempo para hablar de estos temas}
      \end{itemize}
   }
\end{frame}
%-------------------------------------------------------------------------------
\begin{frame}{Función delta}{Respuesta al impulso}
   \videoicon{4}{1h50m50s}
   \begin{columns}[c]
      \begin{column}{1\textwidth}
         \centering\includegraphics[width=0.8\textwidth]{4_clase/respuesta_impulso}
      \end{column}
   \end{columns}
   \vfill
   \note{
      \begin{itemize}
         \item{comentar que cada sistema tiene una única h(n)}
         \item{golpe de Homero en la panza}
         \item{si algo cambia en el sistema, h(n) cambia si o si}
         \item{recordar propiedad de shift y linealidad en las 2 filminas siguientes}
      \end{itemize}
   }
\end{frame}
%-------------------------------------------------------------------------------
\begin{frame}{Todo esta en la respuesta al impulso}
   \videoicon{4}{1h53m50s}
    \center{
       \href{run:./4_clase/video_homero_panza.mp4}{
          \includegraphics[width=0.8\textwidth]{4_clase/homero_panza}
       }
    }
    \vfill
   \note{
      \begin{itemize}
         \item{todo esta en la h(n)}
      \end{itemize}
   }
\end{frame}
%-------------------------------------------------------------------------------
\begin{frame}{Repaso Sistemas}
   \videoicon{4}{1h54m58s}
   \begin{block}{Linealidad}
      Un sistema es lineal cuando su salida depende linealmente de la entrada.
      Satisface el principio de superposición.
   \end{block}
   \begin{columns}[onlytextwidth]
      \column{.7\textwidth}
      \begin{centering}
         \begin{table}[h]
            \begin{tabular}{cm{6cm}cm{6cm}}
               escalado      & \includegraphics[width=6cm]{1_clase/superposicion1}\\
               adición       & \includegraphics[width=6cm]{1_clase/superposicion2}\\
               superposición & \includegraphics[width=6cm]{1_clase/superposicion3}\\
            \end{tabular}
         \end{table}
      \end{centering}
      \column{.3\textwidth}
      \centering{\alert{$y(t)=e^{x(t)}$}\hspace{1cm}\textcolor{green}{$y(t)=\frac{1}{2}x(t)$}}
   \end{columns}
   \vfill
   \note{
      \begin{itemize}
         \item{recordar las propiedades fundamentales de LTI}
      \end{itemize}
   }
\end{frame}
%-------------------------------------------------------------------------------
\begin{frame}{Repaso - Sistemas}{Invariantes en el tiempo}
   \videoicon{4}{1h57m52s}
   \begin{block}{Invariantes en el tiempo}
      Un sistema es invariante en el tiempo cuando la salida para una determinada entrada es la misma sin importar el tiempo en el cual se aplica la entrada
   \end{block}
   \begin{columns}[onlytextwidth]
      \column{.7\textwidth}
      \center\includegraphics[width=1\textwidth]{1_clase/invariante_en_tiempo} \\
      \column{.3\textwidth}
      \centering{\alert{$y(t)=x(t)*\cos{(t)}$}\hspace{1cm}\textcolor{green}{$y(t)=\cos(x(t))$}}
   \end{columns}
   \vfill
   \note{
      \begin{itemize}
         \item{recordar las propiedades fundamentales de LTI}
      \end{itemize}
   }
\end{frame}
%-------------------------------------------------------------------------------
\begin{frame}{Convolución}{Señal vs h(n)}
   \videoicon{4}{1h58m57s}
   \begin{columns}[c]
      \begin{column}{1\textwidth}
         \centering\includegraphics[width=0.8\textwidth]{4_clase/signal_vs_h}
      \end{column}
   \end{columns}
   \vfill
   \note{
      \begin{itemize}
         \item{analizar senial y h}
         \item{vamos a hacer las cuentas a mano..}
      \end{itemize}
   }
\end{frame}
%-------------------------------------------------------------------------------
\begin{frame}{Convolución}{Descomposición delta}
   \videoicon{4}{2h00m00s}
   \begin{columns}[c]
      \begin{column}{1\textwidth}
         \centering\includegraphics[width=0.8\textwidth]{4_clase/descomposicion_delta}
      \end{column}
   \end{columns}
   \vfill
   \note{
      \begin{itemize}
         \item{lanzar conv3}
         \item{sumo delta a delta y voy acumulando..}
         \item{la salida es N+M-1}
      \end{itemize}
   }
\end{frame}
%-------------------------------------------------------------------------------
\section{Convolución}
\begin{frame}{Convolución}{Respuesta al impulso}
   \videoicon{4}{2h18m00s}
   \lstset{ basicstyle=\fontsize{ 5}{ 2}\selectfont\ttfamily,language=python,tabsize=4}
   \begin{columns}[c]
      \hspace{5pt}
      \begin{column}{.3\textwidth}
         \lstinputlisting[lastline=29]{4_clase/conv1.py}
      \end{column}
      \hspace{2pt}
      \vrule
      \hspace{2pt}
      \begin{column}{.3\textwidth}
         \lstinputlisting[firstline=30]{4_clase/conv1.py}
      \end{column}
      \hspace{2pt}
      \vrule
      \hspace{2pt}
      \begin{column}{.38\textwidth}
         \pythonpic{4\_clase/conv1.py}
         {https://drive.google.com/open?id=1Bi0Lz0GcDI8teTLNse-S4px0tQTr1sVx}
         {0.9}
         {4_clase/conv1}
      \end{column}
      \hspace{2pt}
   \end{columns}
   \vfill
   \note{
      \begin{itemize}
         \item{lanzar conv1}
         \item{hacer las cuentas a mano}
      \end{itemize}
   }
\end{frame}
%-------------------------------------------------------------------------------
\begin{frame}{Funcion delta}{Respuesta al impulso}
   \videoicon{4}{2h16m40s}
   \centering\includegraphics[width=0.6\textwidth]{4_clase/entrada_conv_h}\\
   \centering\includegraphics[width=0.6\textwidth]{4_clase/convolucion_eq}
   \vfill
\end{frame}
%-------------------------------------------------------------------------------
\section{Filtrado}
\begin{frame}{Filtrado}{Pasa bajos}
   \videoicon{4}{2h23m00s}
   \lstset{ basicstyle=\fontsize{ 5}{ 1}\selectfont\ttfamily,language=python,tabsize=4}
   \begin{columns}[c]
      \hspace{5pt}
      \begin{column}{.3\textwidth}
         \lstinputlisting[lastline=28]{4_clase/conv2.py}
      \end{column}
      \hspace{2pt}
      \vrule
      \hspace{2pt}
      \begin{column}{.3\textwidth}
         \lstinputlisting[firstline=29]{4_clase/conv2.py}
      \end{column}
      \hspace{2pt}
      \vrule
      \hspace{2pt}
      \begin{column}{.38\textwidth}
         \pythonpic{4\_clase/conv2.py}
         {https://drive.google.com/open?id=1SYwy_vsVlnUdArNaxykd93BW14uWSV39}
         {0.9}
         {4_clase/conv2}
      \end{column}
      \hspace{2pt}
   \end{columns}
   \vfill
\end{frame}
%-------------------------------------------------------------------------------
\begin{frame}{Filtrado}{diferenciador}
   \videoicon{4}{2h37m50s}
         \centering\includegraphics[width=0.5\textwidth]{4_clase/diferenciador}
   \vfill
\end{frame}
%-------------------------------------------------------------------------------
\begin{frame}[c]{Convolución}{Análisis en la CIAA}
   \videoicon{4}{2h41m20s}
   \protoboardicon
   \lstset{ basicstyle=\fontsize{ 4}{ 0}\selectfont\ttfamily,language=c,tabsize=4}
   \begin{columns}[c]
      \hspace{2pt}
      \begin{column}{.5\textwidth}
         \lstinputlisting[lastline=30]{4_clase/ciaa/psf2/src/psf.c}
      \end{column}
      \hspace{2pt}
      \vrule
      \hspace{2pt}
      \begin{column}{.5\textwidth}
         \lstinputlisting[firstline=31,lastline=50]{4_clase/ciaa/psf2/src/psf.c}
      \end{column}
   \end{columns}
   \vfill
\end{frame}
%-------------------------------------------------------------------------------
\begin{frame}[c]{Convolución}{Análisis en la CIAA}
   \videoicon{4}{2h21m11s}
   \protoboardicon
   \lstset{ basicstyle=\fontsize{ 4}{ 0}\selectfont\ttfamily,language=c,tabsize=4}
   \begin{columns}[c]
      \hspace{2pt}
      \begin{column}{.5\textwidth}
         \pythonpic{4\_clase/ciaa/psf2/src/psf.c}
         {https://drive.google.com/open?id=1X-i509HnvhYhnTP-5sq4p42WcBr8PMdU}
         {0.9}
         {4_clase/convolucion_ciaa_pasabanda1}
      \end{column}
      \hspace{2pt}
      \vrule
      \hspace{2pt}
      \begin{column}{.5\textwidth}
         \pythonpic{4\_clase/ciaa/psf2/src/psf.c}
         {https://drive.google.com/open?id=1X-i509HnvhYhnTP-5sq4p42WcBr8PMdU}
         {0.9}
         {4_clase/convolucion_ciaa_pasabanda2}
      \end{column}
   \end{columns}
   \vfill
\end{frame}
%-------------------------------------------------------------------------------
\begin{frame}{Bibliografía}
   \framesubtitle{Libros, links y otro material}
   \begin{thebibliography}{9}
         \bibitem{cmsisdsp}
         \emph{ARM CMSIS DSP}. \\
         \href {https://arm-software.github.io/CMSIS_5/DSP/html/index.html}{https://arm-software.github.io/CMSIS\_5/DSP/html/index.html}
         \bibitem{dsp}
         Steven W. Smith.
         \emph{The Scientist and Engineer's Guide to Digital Signal Processing}.
         Second Edition, 1999.
         \bibitem{grants Anderson}
         \emph{Grant Sanderson} \\
         \href{ https://youtu.be/spUNpyF58BY}{ https://youtu.be/spUNpyF58BY}
         \bibitem{interactivemath}
         \emph{Interactive Mathematics Site Info}. \\
         \href {https://www.intmath.com/fourier-series/fourier-intro.php}{https://www.intmath.com/fourier-series/fourier-intro.php}
   \end{thebibliography}
\end{frame}
%-------------------------------------------------------------------------------

