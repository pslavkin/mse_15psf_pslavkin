\subtitle{Clase 4 - Euler | Fourier - IDFT}

\begin{frame}[c]
\maketitle
\begin{tikzpicture}[overlay,remember picture]
    \node[anchor=south east,xshift=-30pt,yshift=45pt]
      at (current page.south east) {
        \includegraphics[width=0.2\textwidth]{4_clase/leonhard_euler}
        \includegraphics[width=0.2\textwidth]{4_clase/joseph_fourier}
      };
  \end{tikzpicture}
\end{frame}
%-------------------------------------------------------------------------------
\begin{frame}{repaso DFT}{$e^{j2\pi ft}$ modulado por $\sin(t)$ y centro de masas en f, DFT?}
   \handsonicon
   \lstset{ basicstyle=\fontsize{ 4}{ 1}\selectfont\ttfamily,language=Python,tabsize=4}
   \begin{columns}[c]
      \hspace{2pt}
      \begin{column}{.3\textwidth}
         \lstinputlisting[lastline=40]{3_clase/euler4.py}
      \end{column}
      \hspace{2pt}
      \vrule
      \hspace{2pt}
      \begin{column}{.3\textwidth}
         \lstinputlisting[firstline=40]{3_clase/euler4.py}
      \end{column}
      \hspace{2pt}
      \vrule
      \hspace{2pt}
      \begin{column}{.4\textwidth}
         \centering\includegraphics[width=1.0\textwidth]{3_clase/euler4}
      \end{column}
      \hspace{2pt}
   \end{columns}
   \vfill
   \note{
      \begin{itemize}
         \item{retomar DFT con el concepto de centro de masas}
         \item{lanzar euler4.py de la clase3 con dft en modo rectangular}
         \item{destacar DFT parecido entre la señal y el circulo y viceversa}
         \item{Por lo tanto lo que tenemos son relojitos girando}
      \end{itemize}
   }
\end{frame}
%-------------------------------------------------------------------------------
 \begin{frame}{Repaso DFT}{Analisis, Transformada discreta de Fourier}
    \center\includegraphics[width=0.9\textwidth]{3_clase/dft_eq}
    \vfill
 \end{frame}

%-------------------------------------------------------------------------------
\begin{frame}{Síntesis}{Como reconstruyo la señal en el tiempo}
   \handsonicon
   \lstset{ basicstyle=\fontsize{ 4}{ 1}\selectfont\ttfamily,language=Python,tabsize=4}
   \begin{columns}[c]
      \hspace{2pt}
      \begin{column}{.5\textwidth}
         \lstinputlisting[lastline=33]{4_clase/euler6.py}
      \end{column}
      \hspace{2pt}
      \vrule
      \hspace{2pt}
      \begin{column}{.5\textwidth}
         \lstinputlisting[firstline=33,lastline=66]{4_clase/euler6.py}
      \end{column}
   \end{columns}
   \vfill
   \note{
      \begin{itemize}
         \item{lanzar euler6 y explicar que pasa con sin y cos}
         \item{comentar lo de hermitica cuando la salida es real y dft complejo conjugado}
      \end{itemize}
   }
\end{frame}
%-------------------------------------------------------------------------------
\begin{frame}{Síntesis}{Como reconstruyo la señal en el tiempo}
   \handsonicon
   \lstset{ basicstyle=\fontsize{ 4}{ 1}\selectfont\ttfamily,language=Python,tabsize=4}
   \begin{columns}[c]
      \hspace{2pt}
      \begin{column}{.4\textwidth}
         \lstinputlisting[firstline=66]{4_clase/euler6.py}
      \end{column}
      \hspace{2pt}
      \vrule
      \hspace{2pt}
      \begin{column}{.6\textwidth}
         \centering\includegraphics[width=0.9\textwidth]{4_clase/euler6}
      \end{column}
      \hspace{2pt}
   \end{columns}
   \note{
      \begin{itemize}
         \item{de nuevo la formula la calculamos a mano y arrivamos a lo que viene en la filmina siguiente}
      \end{itemize}
   }
   \vfill
\end{frame}
%-------------------------------------------------------------------------------
 \begin{frame}{Síntesis, Transformada inversa discreta de Fourier}{IDFT}
    \center\includegraphics[width=0.9\textwidth]{4_clase/idft_eq}
    \vfill
   \note{
      \begin{itemize}
         \item{comentar tambien lo del 1/N, 1/$\sqrt[2]{N}$}
      \end{itemize}
   }
 \end{frame}

%-------------------------------------------------------------------------------
\begin{frame}{IDFT}{Señales complejas como entrada?}
   \handsonicon
   \lstset{ basicstyle=\fontsize{ 4}{ 1}\selectfont\ttfamily,language=Python,tabsize=4}
   \begin{columns}[c]
      \hspace{2pt}
      \begin{column}{.5\textwidth}
         \lstinputlisting[lastline=33]{4_clase/euler7.py}
      \end{column}
      \hspace{2pt}
      \vrule
      \hspace{2pt}
      \begin{column}{.5\textwidth}
         \lstinputlisting[firstline=33,lastline=66]{4_clase/euler7.py}
      \end{column}
   \end{columns}
   \vfill
   \note{
      \begin{itemize}
         \item{lanzar euler7}
         \item{hacer algunas pruebas con diferentes valores.}
         \item{destacar que esto es la DFT compleja de lado a lado}
      \end{itemize}
   }
\end{frame}
%-------------------------------------------------------------------------------
\begin{frame}{IDFT}{Señales complejas como entrada?}
   \handsonicon
   \lstset{ basicstyle=\fontsize{ 4}{ 1}\selectfont\ttfamily,language=Python,tabsize=4}
   \begin{columns}[c]
      \hspace{2pt}
      \begin{column}{.5\textwidth}
         \lstinputlisting[firstline=66]{4_clase/euler7.py}
      \end{column}
      \hspace{2pt}
      \vrule
      \hspace{2pt}
      \begin{column}{.4\textwidth}
         \centering\includegraphics[width=1.0\textwidth]{4_clase/euler7}
      \end{column}
      \hspace{2pt}
   \end{columns}
   \vfill
\end{frame}
%-------------------------------------------------------------------------------
\begin{frame}{DFT<>IDFT}{Transformadas relevantes}
   \begin{columns}[c]
      \begin{column}{.5\textwidth}
         Delta
         \centering\includegraphics[width=1.0\textwidth]{4_clase/euler_delta}
      \end{column}
      \begin{column}{.5\textwidth}
         Cuadrada
         \centering\includegraphics[width=1.0\textwidth]{4_clase/euler_cuadrada}
      \end{column}
   \end{columns}
   \vfill
   \note{
      \begin{itemize}
         \item{lanzar euler9}
         \item{jugar un poco con la cuadrada y ver como se va formando con los diferentes armonicos}
         \item{explicar la idea de compresion, mp3, etc cortando ancho de banda}
         \item{destacar lo de los armonicos 1f, 3f, 5f para la cuadrada }
         \item{probar la delata y mostrar como tiene todas las frecuencias}
      \end{itemize}
   }
\end{frame}
%-------------------------------------------------------------------------------
\begin{frame}{DFT<>IDFT}{Transformadas relevantes}
   \begin{columns}[c]
      \begin{column}{.5\textwidth}
         Triangular
         \centering\includegraphics[width=1.0\textwidth]{4_clase/euler_triangular}
      \end{column}
      \begin{column}{.5\textwidth}
         Conjugado
         \centering\includegraphics[width=1.0\textwidth]{4_clase/euler_conjugado}
      \end{column}
   \end{columns}
   \vfill
   \note{
      \begin{itemize}
         \item{jugar con el conjugado y mostar como sale la senoide}
         \item{hablar del tema de dualidad}
      \end{itemize}
   }
\end{frame}
%-------------------------------------------------------------------------------
\begin{frame}{DFT<>IDFT}{Transformadas relevantes}
   \begin{columns}[c]
      \begin{column}{.33\textwidth}
         Sin
         \centering\includegraphics[width=1.0\textwidth]{4_clase/equivalencias_seno}
      \end{column}
      \begin{column}{.33\textwidth}
         Delta
         \centering\includegraphics[width=1.0\textwidth]{4_clase/equivalencias_delta}
      \end{column}
      \begin{column}{.33\textwidth}
         Cuadrada
         \centering\includegraphics[width=1.0\textwidth]{4_clase/equivalencias_cuadrada}
      \end{column}
   \end{columns}
   \vfill
\end{frame}
%-------------------------------------------------------------------------------
\begin{frame}{IDFT}{Un conejo como entrada?}
   \handsonicon
   \lstset{ basicstyle=\fontsize{ 4}{ 1}\selectfont\ttfamily,language=Python,tabsize=4}
   \begin{columns}[c]
      \hspace{2pt}
      \begin{column}{.5\textwidth}
         \lstinputlisting[lastline=33]{4_clase/euler8.py}
      \end{column}
      \hspace{2pt}
      \vrule
      \hspace{2pt}
      \begin{column}{.5\textwidth}
         \lstinputlisting[firstline=33,lastline=66]{4_clase/euler8.py}
      \end{column}
   \end{columns}
   \vfill
\end{frame}
%-------------------------------------------------------------------------------
\begin{frame}{IDFT}{Un conejo como entrada?}
   \handsonicon
   \lstset{ basicstyle=\fontsize{ 4}{ 1}\selectfont\ttfamily,language=python,tabsize=4}
   \begin{columns}[c]
      \hspace{2pt}
      \begin{column}{.5\textwidth}
         \lstinputlisting[firstline=66]{4_clase/euler8.py}
      \end{column}
      \hspace{2pt}
      \vrule
      \hspace{2pt}
      \begin{column}{.4\textwidth}
         \centering\includegraphics[width=1.0\textwidth]{4_clase/euler8}
      \end{column}
      \hspace{2pt}
   \end{columns}
   \vfill
   \note{
      \begin{itemize}
         \item{volver a mostara la idea de compresion limitando la idft en ancho de banda}
      \end{itemize}
   }
\end{frame}
%-------------------------------------------------------------------------------
\begin{frame}{FFT-IFFT}{Transformadas usando FFT en Python}
   \handsonicon
   \lstset{ basicstyle=\fontsize{ 5}{ 0}\selectfont\ttfamily,language=Python,tabsize=4}
   \begin{columns}[c]
      \hspace{2pt}
      \begin{column}{.55\textwidth}
         \lstinputlisting[lastline=33]{4_clase/fft1.py}
      \end{column}
      \hspace{2pt}
      \vrule
      \hspace{2pt}
      \begin{column}{.45\textwidth}
         \centering\includegraphics[width=0.9\textwidth]{4_clase/fft1}
      \end{column}
   \end{columns}
   \vfill
\end{frame}
%-------------------------------------------------------------------------------
\section{Respuesta al impulso}
\begin{frame}{Funcion delta}{Respuesta al impulso}
   \begin{columns}[c]
      \begin{column}{1\textwidth}
         \centering\includegraphics[width=0.8\textwidth]{4_clase/respuesta_impulso}
      \end{column}
   \end{columns}
   \vfill
\end{frame}
%-------------------------------------------------------------------------------
\begin{frame}{Convolucion}{Descomposicion Delta}
   \begin{columns}[c]
      \begin{column}{1\textwidth}
         \centering\includegraphics[width=0.8\textwidth]{4_clase/descomposicion_delta}
      \end{column}
   \end{columns}
   \vfill
\end{frame}
%-------------------------------------------------------------------------------
\begin{frame}{Funcion delta}{Respuesta al impulso}
   \centering\includegraphics[width=0.6\textwidth]{4_clase/entrada_conv_h}\\
   \centering\includegraphics[width=0.6\textwidth]{4_clase/convolucion_eq}
   \vfill
\end{frame}
%-------------------------------------------------------------------------------
\section{Convolucion}
\begin{frame}{Convolucion}{Respuesta al impulso}
   \lstset{ basicstyle=\fontsize{ 5}{ 2}\selectfont\ttfamily,language=python,tabsize=4}
   \begin{columns}[c]
      \hspace{5pt}
      \begin{column}{.3\textwidth}
         \lstinputlisting[lastline=25]{4_clase/conv1.py}
      \end{column}
      \hspace{2pt}
      \vrule
      \hspace{2pt}
      \begin{column}{.3\textwidth}
         \lstinputlisting[firstline=26]{4_clase/conv1.py}
      \end{column}
      \hspace{2pt}
      \vrule
      \hspace{2pt}
      \begin{column}{.38\textwidth}
         \centering\includegraphics[width=1.0\textwidth]{4_clase/conv1}
      \end{column}
      \hspace{2pt}
   \end{columns}
   \vfill
\end{frame}
%-------------------------------------------------------------------------------
\section{Filtrado}
\begin{frame}{Filtrado}{Pasa bajos}
   \lstset{ basicstyle=\fontsize{ 5}{ 1}\selectfont\ttfamily,language=python,tabsize=4}
   \begin{columns}[c]
      \hspace{5pt}
      \begin{column}{.3\textwidth}
         \lstinputlisting[lastline=28]{4_clase/conv2.py}
      \end{column}
      \hspace{2pt}
      \vrule
      \hspace{2pt}
      \begin{column}{.3\textwidth}
         \lstinputlisting[firstline=29]{4_clase/conv2.py}
      \end{column}
      \hspace{2pt}
      \vrule
      \hspace{2pt}
      \begin{column}{.38\textwidth}
         \centering\includegraphics[width=1.0\textwidth]{4_clase/conv2}
      \end{column}
      \hspace{2pt}
   \end{columns}
   \vfill
\end{frame}
%-------------------------------------------------------------------------------
\begin{frame}{Filtrado}{diferenciador}
         \centering\includegraphics[width=0.5\textwidth]{4_clase/diferenciador}
   \vfill
\end{frame}
%-------------------------------------------------------------------------------
\begin{frame}{Bibliografía}
   \framesubtitle{Libros, links y otro material}
   \begin{thebibliography}{9}
         \bibitem{cmsisdsp}
         \emph{ARM CMSIS DSP}. \\
         \href {https://arm-software.github.io/CMSIS_5/DSP/html/index.html}{link}
         \bibitem{dsp}
         Steven W. Smith.
         \emph{The Scientist and Engineer's Guide to Digital Signal Processing}.
         Second Edition, 1999.
         \bibitem{link}
         \emph{Interactive Mathematics Site Info}.
         \bibitem{grants Anderson}
         \emph{Grant Sanderson} \\
         \href{ https://youtu.be/spUNpyF58BY}{link}
        \bibitem{interactivemath}
            \emph{Interactive Mathematics Site Info}. \\
            \href {https://www.intmath.com/fourier-series/fourier-intro.php}{link}
   \end{thebibliography}
\end{frame}
%-------------------------------------------------------------------------------

