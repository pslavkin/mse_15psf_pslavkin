%-------------------------------------------------------------------------------
\subtitle{Trabajo Practico}
\begin{frame}[c]
\maketitle
\begin{tikzpicture}[overlay,remember picture]
    \node[anchor=south east,xshift=-30pt,yshift=45pt]
      at (current page.south east) {
        \includegraphics[width=45mm]{1_clase/python_continuo_vs_discreto}
      };
  \end{tikzpicture}
\end{frame}
%-------------------------------------------------------------------------------
 \section{LTI}
 \subsection{Sistemas LTI}
 \begin{frame}{Sistemas LTI}
 Demuestre si los siguientes sistemas son LTI:\\
 \begin{align*}
    y(t) &= x(t)*\cos{(t)} \\
    y(t) &= \cos(x(t)) \\
    y(t) &= e^{x(t)} \\
    y(t) &= \frac{1}{2}x(t)
 \end{align*}
    \vfill
 \end{frame}
%-------------------------------------------------------------------------------
 \begin{frame}{Generacion de señales con Python}
    Realizar las funciones necesarias para generar las siguientes señales:
    \begin{enumerate}
       \item {Senoidal. (parámetros: fase (radianes))}
       \item {Cuadrada. (parámetros: ciclo de actividad)}
       \item {Triangular. (parámetros: punto de simetría)}
    \end{enumerate}
    Nota: Los parámetros comunes a todas serán:
    \begin{itemize}
       \item {frecuencia de muestreo fs (Hz) }
       \item {frecuencia fundamental de la onda fo (Hz) }
       \item {Amplitud normalizada }
       \item {Cantidad de muestras N. }
    \end{itemize}
        Es decir que se podría invocar la señal que genere la senoidal como: \\
        signal = sin( 1000, 100, 1, 1000, pi/2);
        Grafique al menos una de cada una de las señales
    \vfill
 \end{frame}
%-------------------------------------------------------------------------------
 \section{Fourier}
 \begin{frame}{Transformada discreta de Fourier}
    \begin{enumerate}
       \item{Grafique una senoidal, una cuadrada y una triangular lado a lado con su respectivo espectro en frecuencias indicando los parametros destacados como Fs, N, B, etc.} 
       \item{Generar un LA-440 con python, capturarlo con la CIAA y mostrarlo con pyplot}
       \item{Samplear con 10, 8, 4 y 2 bits y analizar el ruido de cuantizacion en cada caso}
       \item{Sume al LA-440 un LA-442 y utilizando la tecnica de zero padding para aumentar la resolucion expectral intente discriminar ambas}
       \item{Compare los resultados con la simulacion en python}
       \item{Anime el trazado de la inicial de su nombre a partir de la IDFT y explique su funcionamiento}
       \item{}
    \end{enumerate}
 \end{frame}
