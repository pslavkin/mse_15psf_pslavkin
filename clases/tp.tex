%-------------------------------------------------------------------------------
\subtitle{Trabajo Practico}
\begin{frame}[c]
\maketitle
\begin{tikzpicture}[overlay,remember picture]
    \node[anchor=south east,xshift=-30pt,yshift=45pt]
      at (current page.south east) {
        \includegraphics[width=45mm]{1_clase/python_continuo_vs_discreto}
      };
  \end{tikzpicture}
\end{frame}
%-------------------------------------------------------------------------------
 \section{LTI}
 \subsection{Sistemas LTI}
 \begin{frame}{Sistemas LTI}
 Demuestre si los siguientes sistemas son LTI:\\
 \begin{align*}
    y(t) &= x(t)*\cos{(t)} \\
    y(t) &= \cos(x(t)) \\
    y(t) &= e^{x(t)} \\
    y(t) &= \frac{1}{2}x(t)
 \end{align*}
    \vfill
 \end{frame}
%-------------------------------------------------------------------------------
 \begin{frame}{ADC-DAC}
    \begin{itemize}
       \item{Calcule la relación señal a ruido de cuantización de un sistema con ADC de 10 bits}
       \item{Cuantice el LA-440 con 10, 8, 4 y 2 bits y analice el ruido de cuantización en cada caso}
    \end{itemize}
 \end{frame}
%-------------------------------------------------------------------------------
 \begin{frame}{Filtro antialias y reconstrucción}
    \begin{itemize}
       \item{Calcular el filtro antialias que utilizara para su practica y/o trabajo final y justifique su decision}
       \item{Si lo tuviera justifique el filtro de reconstrucción}
    \end{itemize}
 \end{frame}
%-------------------------------------------------------------------------------
 \section{Fourier}
 \begin{frame}{Transformada discreta de Fourier}
    \begin{enumerate}
       \item{Grafique una senoidal, una cuadrada y una triangular lado a lado con su respectivo espectro en frecuencias indicando los parámetros destacados como densidad espectral de potencia, Fs, N, B, etc.} 
       \item{Genere un LA-440 con python u otro instrumento, adquiéralo con la CIAA y grafique con pyplot o equivalente}
       \item{Sume al LA-440 un tono-442 y utilizando la técnica de zero padding para aumentar la resolución espectral intente discriminar ambas \\ Compare los resultados con una simulación}
       \item{Anime el trazado de la letra inicial de su nombre a partir de la IDFT de las coordenadas de sus vertices y explique su funcionamiento}
    \end{enumerate}
 \end{frame}
