%-------------------------------------------------------------------------------
\subtitle{Trabajo Practico}
\begin{frame}[c]
\maketitle
\begin{tikzpicture}[overlay,remember picture]
    \node[anchor=south east,xshift=-40pt,yshift=25pt]
      at (current page.south east) {
        \includegraphics[width=55mm]{1_clase/python_continuo_vs_discreto}
      };
  \end{tikzpicture}
\end{frame}
%-------------------------------------------------------------------------------
 \section{LTI}
 \subsection{Sistemas LTI}
 \begin{frame}{Sistemas LTI}
 Demuestre si los siguientes sistemas son LTI:\\
 \begin{align*}
    y(t) &= x(t)*\cos{(t)} \\
    y(t) &= \cos(x(t)) \\
    y(t) &= e^{x(t)} \\
    y(t) &= \frac{1}{2}x(t)
 \end{align*}
    \vfill
 \end{frame}
%-------------------------------------------------------------------------------
 \section{Ruido de cuantización}
 \begin{frame}{Ruido de cuantización}
    \begin{enumerate}
       \item{Calcule la relación señal a ruido de cuantización teórica máxima de un sistema con un ADC de 10 bits}
    \end{enumerate}
 \end{frame}
%-------------------------------------------------------------------------------
 \section{FAA}
 \begin{frame}{Filtro antialias y reconstrucción}
    \begin{enumerate}
       \item{Calcular el filtro antialias que utilizara para su practica y/o trabajo final y justifique su decision}
       \item{Si lo tuviera justifique el filtro de reconstrucción}
    \end{enumerate}
 \end{frame}
%-------------------------------------------------------------------------------
 \section{Python - Numpy}
 \begin{frame}{Generación y simulación}
    \begin{enumerate}
       \item{Genere un modulo o un paquete con al menos las siguientes funciones}
             \begin{itemize}
                \item{senoidal  (f muestreo[Hz], f[Hz], amp[0 a 1], muestras),fase [radianes]}
                \item{Cuadrada  (f muestreo[Hz], f[Hz], amp[0 a 1], muestras)}
                \item{Triangular(f muestreo[Hz], f[Hz], amp[0 a 1], muestras)}
             \end{itemize}
          \item{Realice los siguientes experimentos} \\
            Dado:
             \begin{itemize}
                \item{fs   = 1000}
                \item{N    = 1000}
                \item{fase = 0}
             \end{itemize}
             \begin{enumerate}
                \item{f0 = [0.1, 1.1]*fs}
                   Como podría diferenciar las senoidales?
                \item{f0 = [0.49, 0.51]*fs}
                   Como es la frecuencia y la fase?
             \end{enumerate}
    \end{enumerate}
 \end{frame}
%-------------------------------------------------------------------------------
 \section{CIAA}
 \begin{frame}{Adquisición y reconstrucción con la CIAA}
    \begin{enumerate}
       \item{Genere con un tono LA-440. Digitalice con 10, 8, 4 y 2 bits con el ADC, envíe los datos a la PC y grafique :}
          \begin{itemize}
             \item{Señal original}
             \item{Señal adquirida}
             \item{Señal error: Original-Adquirida}
          \end{itemize}
       \item{Realice el mismo experimento pero ahora el tono generado con el DAC}
       \item{Calcule las siguientes características de la señal original}
   \begin{itemize}
      \item{Valor medio}
      \item{RMS}
      \item{Energía}
      \item{Histograma del error}
   \end{itemize}
    \end{enumerate}
 \end{frame}
%-------------------------------------------------------------------------------
 \section{Fourier}
 \begin{frame}{Transformada discreta de Fourier}
    \begin{enumerate}
       \item{Grafique una senoidal, una cuadrada y una triangular lado a lado con su respectivo espectro en frecuencias indicando los parámetros destacados como densidad espectral de potencia, Fs, N, B, etc.}
       \item{Genere un LA-440 mas tono-442 y utilizando la técnica de zero padding para aumentar la resolución espectral intente discriminar ambas}
       \item{Anime el trazado de la letra inicial de su nombre a partir de la IDFT de las coordenadas de sus vertices y explique su funcionamiento}
    \end{enumerate}
 \end{frame}
