\subtitle{Clase 6 - Filtrado con DFT}

\begin{frame}[c]
\maketitle
\begin{tikzpicture}[overlay,remember picture]
    \node[anchor=south east,xshift=-30pt,yshift=45pt]
      at (current page.south east) {
        \includegraphics[width=0.4\textwidth]{6_clase/overlap_add1}
      };
  \end{tikzpicture}
     \note{
        \begin{itemize}
           \item{comentar lo de los cambios en los archivos del drive}
           \item{comentar lo del espacio para que suban los trabajos los alumnos}
        \end{itemize}
     }
\end{frame}
%-------------------------------------------------------------------------------
\begin{frame}[t]{Filtrado}{Definición}
   \videoicon{0h00m00s}
   \begin{columns}[t]
      \footnotesize
      \begin{column}{.4\textwidth}
         \begin{itemize}
            \item{Plantilla de diseño de un filtro}
            \item{En el ejemplo se aprecia un pasabajos pero se destacan las zonzas de interés y los niveles de la banda de paso y de rechazo}
            \item{Cuanto mas exigente se la plantilla del filtro, mas puntos tendrá nuestra h(n) y mas lenta y compleja su convolución}
            \item{El objetivo es llegar a un compromiso entre los requisitos y la performance }
         \end{itemize}
      \end{column}
      \hspace{2pt}
      \vrule
      \hspace{2pt}
      \begin{column}{.6\textwidth}
         \center\includegraphics[width=1.0\textwidth]{5_clase/pyfda3}
      \end{column}
      \hspace{2pt}
   \end{columns}
   \vfill
   \note{
      \begin{itemize}
         \item{explicar las zonas de los filtros, tipos de filtro}
         \item{relación de compromiso entre ripple y bandas, etc}
      \end{itemize}
   }
\end{frame}
%-------------------------------------------------------------------------------
\begin{frame}[t]{Filtrado}{PyFDA \href{/opt/anaconda3/bin/pyfdax}{/opt/anaconda3/bin/pyfdax}}
   \videoicon{0h00m00s}
   \begin{columns}[t]
      \footnotesize
      \begin{column}{.4\textwidth}
         \begin{itemize}
            \item{Uso de PyFDA como herramienta para diseño de filtros}
            \item{Inicialmente solo nos concentramos en la H(f) para visualizar de manera practica las zonas de paso y de rechazo}
         \end{itemize}
      \end{column}
      \hspace{2pt}
      \vrule
      \hspace{2pt}
      \begin{column}{.6\textwidth}
         \center\includegraphics[width=1.00\textwidth]{5_clase/pyfda1}
      \end{column}
      \hspace{2pt}
   \end{columns}
   \vfill
   \note{
      \begin{itemize}
         \item{explicar ahora el uso de la convolution en el filtrado}
         \item{a partir de 64 puntos de fir conviene FFT, por menos conviene convolution en tiempo}
      \end{itemize}
   }
\end{frame}
%-------------------------------------------------------------------------------
\begin{frame}[t]{Filtrado}{Pyfda \href{/opt/anaconda3/bin/pyfdax}{/opt/anaconda3/bin/pyfdax}}
   \videoicon{0h00m00s}
   \begin{columns}[t]
      \footnotesize
      \begin{column}{.4\textwidth}
         \begin{itemize}
            \item{Uso de PyFDA como herramienta para diseño de filtros}
            \item{Inicialmente solo nos concentramos en la H(f) para visualizar de manera practica las zonas de paso y de rechazo}
            \item{Notar la variedad de opciones disponibles y la respuesta en fase en esta imagen}
         \end{itemize}
      \end{column}
      \hspace{2pt}
      \vrule
      \hspace{2pt}
      \begin{column}{.6\textwidth}
         \center\includegraphics[width=1.00\textwidth]{5_clase/pyfda2}
      \end{column}
      \hspace{2pt}
   \end{columns}
   \vfill
   \note{
      \begin{itemize}
         \item{explicar ahora el uso de la convolution en el filtrado}
         \item{a partir de 64 puntos de fir conviene FFT, por menos conviene convolucino en tiempo}
      \end{itemize}
   }
\end{frame}
%-------------------------------------------------------------------------------
\begin{frame}[t]{Convolución}{Superponer y sumar en t}
   \videoicon{0h00m00s}
   \begin{columns}[t]
      \tiny
      \begin{column}{.4\textwidth}
         \begin{itemize}
            \item{Ejemplo de como particionar la x(n) en tramos y filtrar usando convolucion en tiempo }
            \item{Se convoliciona x(n) con h(n) ambas con zero padding hata que cada una tenga N+M-1 puntos}
            \item{Notar el efecto del inicio del filtrado en cada tramo como se corrige con el siguiente}
            \item{Esto es equivalente a tomar la x(n) completa y convolucionar con la h(n)}
            \item{Si x(n) es una señal muy larga o en tiempo real, convolucionar de una sola vez se vuelve impracticable}
            \item{Con este metodo podemos mostrar cada cierto tiempo como va evolucionando la salida}
         \end{itemize}
      \end{column}
      \hspace{2pt}
      \vrule
      \hspace{2pt}
      \begin{column}{.6\textwidth}
      \pythonpic{6\_clase/overlap\_add1.py}
         {https://drive.google.com/open?id=1xZT8fWAvQEmlsDpi0A-WIrAzRsSm2X4u}
         {1.0}
         {6_clase/overlap_add1}
      \end{column}
      \hspace{2pt}
   \end{columns}
   \vfill
   \note{
      \begin{itemize}
         \item{explicar el detalle de overlap para sumar}
      \end{itemize}
   }
\end{frame}
%-------------------------------------------------------------------------------
\begin{frame}[t]{Convolución con FFT}{Superponer y sumar en f}
   \videoicon{0h00m00s}
   \begin{columns}[t]
      \footnotesize
      \begin{column}{.4\textwidth}
         \begin{itemize}
            \item{Ejemplo de como particionar la x(n) en tramos y filtrar usando IFFT(fft(x-padd)*fft(h-padd))}
            \item{Se hace zero padding a x(n) y  h(n) hasta que ambas tengan N+M-1 datos}
            \item{Se denera elegir N de modo tal que N+M-1 sea potencia de 2 para que la eficiencia del algoritmo FFT sea maxima}
            \item{Se calcula DFT(x)=X y DFT(h)=H}
            \item{Se calcular  Y=X multiplicado H}
            \item{Se antitransforma Y y se obtiene y(n)}
         \end{itemize}
      \end{column}
      \hspace{2pt}
      \vrule
      \hspace{2pt}
      \begin{column}{.6\textwidth}
      \pythonpic{6\_clase/overlap\_add2.py}
         {https://drive.google.com/open?id=17wJBX7EVMFLRX1J8rUFCT6qLZQ29M9UF}
         {1.0}
         {6_clase/overlap_add2}
      \end{column}
      \hspace{2pt}
   \end{columns}
   \vfill
   \note{
      \begin{itemize}
         \item{explicar el detalle de overlap para sumar}
      \end{itemize}
   }
\end{frame}
%-------------------------------------------------------------------------------
\begin{frame}[t]{Promedio en magnitud de FFT}{Superponer y promediar en f}
   \videoicon{0h00m00s}
   \begin{columns}[t]
      \footnotesize
      \begin{column}{.4\textwidth}
         \begin{itemize}
            \item{Ejemplo de como particionar la x(n) en tramos luego calcular fft(x) y promediar}
            \item{Se hace un promedio de la lagnitud de los espectros de cada segmento}
            \item{Se logra promediar el ruido y destacar las componentes con informacion incluso sumergidas en ruido}
         \end{itemize}
      \end{column}
      \hspace{2pt}
      \vrule
      \hspace{2pt}
      \begin{column}{.6\textwidth}
      \pythonpic{6\_clase/overlap\_average.py}
         {https://drive.google.com/open?id=1DHg7owe-AppZx0_7SB_eyI6QvWMHtCIV}
         {1.0}
         {6_clase/overlap_average}
      \end{column}
      \hspace{2pt}
   \end{columns}
   \vfill
   \note{
      \begin{itemize}
         \item{explicar como se saca la senial dentro del ruido}
      \end{itemize}
   }
\end{frame}
%-------------------------------------------------------------------------------
\section{CIAA}
\begin{frame}[t]{Filtrado con CIAA}{Conversor PyFDA a fir.h para C}
   \handsonicon
   \videoicon{0h00m00s}
   \begin{columns}[t]
      \begin{column}{.4\textwidth}
         \begin{itemize}
            \item{Código en Python para convertir los coeficientes del fir extendidos en Q1.15 en C}
            \item{Util para pasar de PyFda a los codigos de C}
         \end{itemize}
      \end{column}
      \hspace{2pt}
      \vrule
      \hspace{2pt}
      \begin{column}{.6\textwidth}
      \pythonpic{6\_clase/fir\_to\_c.py}
         {https://drive.google.com/open?id=1aQp6UMQGfGUqixawvTdX19Ukrp1NHOx5}
         {0.8}
         {6_clase/fir_to_c}
      \end{column}
      \hspace{2pt}
   \end{columns}

   \vfill
   \note{
      \begin{itemize}
         \item{mostrar como pasar de pyfda a C}
         \item{lanzar psf1}
         \item{probar distintos filtros y ver resultado }
         \item{hacer notar el efecto del padding}
      \end{itemize}
   }
\end{frame}
%-------------------------------------------------------------------------------
\begin{frame}[t]{Filtrado con CIAA}{Convolución en tiempo con padding}
   \videoicon{0h00m00s}
   \protoboardicon
   \lstset{ basicstyle=\fontsize{ 5}{ 1}\selectfont\ttfamily,language=c,tabsize=4}
   \begin{columns}[c]
      \hspace{5pt}
      \begin{column}{.3\textwidth}
         \lstinputlisting[lastline=32]{6_clase/ciaa/psf1/src/psf.c}
      \end{column}
      \hspace{2pt}
      \vrule
      \hspace{2pt}
      \begin{column}{.3\textwidth}
         \lstinputlisting[firstline=34]{6_clase/ciaa/psf1/src/psf.c}
      \end{column}
      \hspace{2pt}
      \vrule
      \hspace{2pt}
      \begin{column}{.4\textwidth}
      \pythonpic{6\_clase/ciaa/psf1/src/conv1.c}
         {https://drive.google.com/open?id=1lHlxr3KEaLmgdWRp1LQ9AgpAOJdbaZQN}
         {1.0}
         {6_clase/ciaa/psf1/conv1}
      \end{column}
      \hspace{2pt}
   \end{columns}
   \vfill
   \note{
      \begin{itemize}
         \item{mostrar como pasar de pyfda a C}
         \item{lanzar psf1}
         \item{probar distintos filtros y ver resultado }
         \item{hacer notar el efecto del padding}
      \end{itemize}
   }
\end{frame}
%-------------------------------------------------------------------------------
\begin{frame}[t]{Filtrado con CIAA}{Multiplicacion en frecuencia con padding}
   \videoicon{0h00m00s}
   \protoboardicon
   \lstset{ basicstyle=\fontsize{ 4}{ 1}\selectfont\ttfamily,language=c,tabsize=4}
   \begin{columns}[c]
      \hspace{2pt}
      \begin{column}{.3\textwidth}
         \lstinputlisting[lastline=35]{6_clase/ciaa/psf2/src/psf.c}
      \end{column}
      \hspace{2pt}
      \vrule
      \hspace{2pt}
      \begin{column}{.3\textwidth}
         \lstinputlisting[firstline=36]{6_clase/ciaa/psf2/src/psf.c}
      \end{column}
      \hspace{2pt}
      \vrule
      \hspace{2pt}
      \begin{column}{.4\textwidth}
      \pythonpic{6\_clase/ciaa/psf2/src/conv1.c}
         {https://drive.google.com/open?id=1xDsfLidIBCHnY4XC34iDUsbcfQqT0awR}
         {1.0}
         {6_clase/ciaa/psf2/conv1}
      \end{column}
      \hspace{2pt}
   \end{columns}
   \vfill
   \note{
      \begin{itemize}
         \item{mostrar como pasar de pyfda a C}
         \item{lanzar psf1}
         \item{probar distintos filtros y ver resultado }
         \item{hacer notar el efecto del padding}
      \end{itemize}
   }
\end{frame}
%-------------------------------------------------------------------------------
 \begin{frame}{Enuestas}
    \framesubtitle{Encuesta anónima clase a clase}
    Propiciamos este espacio para compartir sus sugerencias, criticas constructivas, oportunidades de mejora y cualquier tipo de comentario relacionado a la clase.
    \begin{block}{Encuesta anónima}{
       \includegraphics[width=0.1\textwidth]{1_clase/click}
       \href{https://forms.gle/1j5dDTQ7qjVfRwYo8}{https://forms.gle/1j5dDTQ7qjVfRwYo8}
    }
       \end{block}
    \begin{block}{Link al material de la material}{
       \includegraphics[width=0.1\textwidth]{1_clase/click}
       \tiny{\href{https://drive.google.com/drive/u/1/folders/1TlR2cgDPchL\_4v7DxdpS7pZHtjKq38CK}{https://drive.google.com/drive/u/1/folders/1TlR2cgDPchL\_4v7DxdpS7pZHtjKq38CK}
    }
    }
       \end{block}
\end{frame}
%-------------------------------------------------------------------------------
\begin{frame}{Bibliografía}
   \framesubtitle{Libros, links y otro material}
   \begin{thebibliography}{9}
         \bibitem{cmsisdsp}
         \emph{ARM CMSIS DSP}. \\
         \href {https://arm-software.github.io/CMSIS_5/DSP/html/index.html}{https://arm-software.github.io/CMSIS\_5/DSP/html/index.html}
         \bibitem{dsp}
         Steven W. Smith.
         \emph{The Scientist and Engineer's Guide to Digital Signal Processing}.
         Second Edition, 1999.
         \bibitem{Teorema de la convolucion}
         \emph{Wikipedia}. \\
         \href {https://en.wikipedia.org/wiki/Convolution\_theorem}{https://en.wikipedia.org/wiki/Convolution\_theorem}
   \end{thebibliography}
\end{frame}
