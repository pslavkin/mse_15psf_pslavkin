%\documentclass[draft]{beamer}
\documentclass[aspectratio=169]{beamer}
%\documentclass[aspectratio=43]{beamer}
\usetheme[faculty=fi]{fibeamer}
\usepackage[utf8]{inputenc}
\usepackage[ main=spanish,english,activeacute]{babel}

%% These additional packages are used within the document:
\usepackage{pdfpages}
\usepackage{ragged2e}  % `\justifying` text
\usepackage{booktabs}  % Tables
\usepackage{graphicx}
\usepackage{tabularx}
\usepackage{tikz}      % Diagrams
\usetikzlibrary{calc, shapes, backgrounds}
\usepackage{amsmath, amssymb}
\usepackage{url}       % `\url`s
\usepackage{listings}  % Code listings
%\frenchspacing    %no se para que es
\usepackage{multiaudience} %para seleccionar que frames sacar en funcio de una variable
\usepackage{pgfpages} %para poner notas de ayuda con un parche para que no se salgan las notas fuera de margen... truchazo
\makeatletter
\defbeamertemplate{note page}{infolines}
{%
  {%
    \scriptsize
    \insertvrule{.25\paperheight}{white!90!black}
    \vskip-.25\paperheight
    \nointerlineskip
    \vbox{
      \hfill\insertslideintonotes{0.25}\hskip-\Gm@rmargin\hskip0pt%
      \vskip-0.25\paperheight%
      \nointerlineskip
      \begin{pgfpicture}{0cm}{0cm}{0cm}{0cm}
        \begin{pgflowlevelscope}{\pgftransformrotate{90}}
          {\pgftransformshift{\pgfpoint{-2cm}{0.2cm}}%
          \pgftext[base,left]{\footnotesize\the\year-\ifnum\month<10\relax0\fi\the\month-\ifnum\day<10\relax0\fi\the\day}}
        \end{pgflowlevelscope}
      \end{pgfpicture}}
    \nointerlineskip
    \vbox to .25\paperheight{\vskip0.5em
      \hbox{\insertshorttitle[width=8cm]}%
      \setbox\beamer@tempbox=\hbox{\insertsection}%
      \hbox{\ifdim\wd\beamer@tempbox>1pt{\hskip4pt\raise3pt\hbox{\vrule
            width0.4pt height7pt\vrule width 9pt
            height0.4pt}}\hskip1pt\hbox{\begin{minipage}[t]{7.5cm}\def\breakhere{}\insertsection\end{minipage}}\fi%
      }%
      \setbox\beamer@tempbox=\hbox{\insertsubsection}%
      \hbox{\ifdim\wd\beamer@tempbox>1pt{\hskip17.4pt\raise3pt\hbox{\vrule
            width0.4pt height7pt\vrule width 9pt
            height0.4pt}}\hskip1pt\hbox{\begin{minipage}[t]{7.5cm}\def\breakhere{}\insertsubsection\end{minipage}}\fi%
      }%
      \setbox\beamer@tempbox=\hbox{\insertshortframetitle}%
      \hbox{\ifdim\wd\beamer@tempbox>1pt{\hskip30.8pt\raise3pt\hbox{\vrule
            width0.4pt height7pt\vrule width 9pt
            height0.4pt}}\hskip1pt\hbox{\insertshortframetitle[width=7cm]}\fi%
      }%
      \vfil}%
  }%
  \vskip.25em
  \nointerlineskip
  \begin{minipage}{1.3\textwidth} % this is an addition
  \insertnote
  \end{minipage}               % this is an addition
}
\makeatother
\setbeamertemplate{note page}[infolines]
%-------------------------------------------------------------------------------
\newcommand{\asignature}{ PDF MSE2020 }
\newcommand{\handsonicon}{
  \begin{tikzpicture}[overlay,remember picture]
     \node[anchor=south east,xshift=-5pt,yshift=0.85\textheight]
     at (current page.south east) {
           \includegraphics[width=15mm]{resources/hands_on}
     };
  \end{tikzpicture}
  }
\newcommand{\protoboardicon}{
  \begin{tikzpicture}[overlay,remember picture]
     \node[anchor=south east,xshift=-5pt,yshift=0.81\textheight]
         at (current page.south east) {
           \includegraphics[width=15mm]{resources/protoboard}
         };
     \end{tikzpicture}
  }
\newtheorem{teorema}{Teorema} %teorema en spanish

%-------------------------------------------------------------------------------
\newcommand{\insertuniversity}{Maestría en sistemas embebidos \\
Universidad de Buenos Aires \\
MSE 5Co2020 }

\newcommand{\insertauthoremail}{\scriptsize{slavkin.pablo@gmail.com \\wapp:011-62433453}}

%% These macros specify information about the presentation
\title{Procesamiento de señales, fundamentos}
\author{Ing. Pablo Slavkin}

% Declare all possible audience groups
\SetNewAudience{clase1}
\SetNewAudience{clase2}
\SetNewAudience{clase3}
\SetNewAudience{clase4}
\SetNewAudience{tp1}
\SetNewAudience{tp2}
\SetNewAudience{python}
%-------------------------------------------------------------------------------
\abovedisplayskip= 0pt
\belowdisplayskip= 0pt
\abovedisplayshortskip=0pt
\belowdisplayshortskip=7pt

\begin{document}

\abovedisplayskip= 0pt
\belowdisplayskip= 0pt
\abovedisplayshortskip=0pt
\belowdisplayshortskip=7pt

%  \shorthandoff{-}
  \frame[c]{
  \maketitle
  \begin{tikzpicture}[overlay,remember picture]
       \node[anchor=south east,xshift=-30pt,yshift=45pt]
         at (current page.south east) {
           \includegraphics[width=45mm]{1_clase/python_continuo_vs_discreto}
         };
     \end{tikzpicture}%
  }

  \AtBeginSection[]{% Print an outline at the beginning of sections
  \begin{frame}<beamer>
      \frametitle{Resumen de seccion \thesection}
      \tableofcontents[currentsection]
    \end{frame}}



  \begin{darkframes}
    \section{Señales}
%-------------------------------------------------------------------------------
    \subsection{Plan de vuelo}
    \begin{frame}{Plan de vuelo}
      \framesubtitle{Ud. esta aqui}
      \center\includegraphics[width=0.8\textwidth]{1_clase/Esquema_MSE}
      \vfill
    \end{frame}

    \subsection{Porque digital?}
    \begin{frame}{Porque digital?}
      \framesubtitle{Digital vs analogico}
      \begin{columns}[onlytextwidth]
        \column{.5\textwidth}
            \begin{itemize}
               \item{Digital}
                  \begin{itemize}
                     \item{Reproducibilidad}
                     \item{Tolerancia de componentes}
                     \item{Partidas todas iguales}
                     \item{Componentes no envejecen}
                     \item{Facil de actualizar}
                     \item{Soluciones de un solo chip}
                  \end{itemize}
               \item{Analogico}
                  \begin{itemize}
                     \item{Alto ancho de banda}
                     \item{Alta potencia}
                     \item{Baja latencia}
                  \end{itemize}
            \end{itemize}
        \column{.5\textwidth}
               \includegraphics[width=30mm]{1_clase/fpga}
               \newline

               \includegraphics[width=35mm]{1_clase/transistor_amp}
      \end{columns}
    \end{frame}


%      \begin{tikzpicture}[overlay,remember picture]
%        \node[anchor=south east,xshift=-30pt,yshift=105pt]
%          at (current page.south east) {
%            \includegraphics[width=45mm]{1_clase/transistor_amp}
%            \includegraphics[width=45mm]{1_clase/fpga}
%            \includegraphics[width=45mm]{1_clase/digital_vs_analogico}
%          };
%      \end{tikzpicture}%
%-------------------------------------------------------------------------------
    \subsection{Señales}
    \begin{frame}{Señales y sistemas}
      \framesubtitle{Que son?}
      \begin{block}{Señal}
         Una señal, en función de una o más variables, puede definirse como un cambio observable en una entidad cuantificable
      \end{block}
      \begin{block}{Sistema}
         Un sistema es cualquier conjunto físico de componentes que actúan en una señal, tomando una o más señales de entrada, y produciendo una o más señales de salida.
      \end{block}
    \end{frame}
%-------------------------------------------------------------------------------
    \begin{frame}{Señales y sistemas}
      \framesubtitle{Tipos de señales}
      \begin{columns}[onlytextwidth]
        \column{.5\textwidth}
            \begin{itemize}
               \item{De tiempo continuo}
               \item{Pares}
               \item{Periódicas}
               \item{De energía}
               \item{Reales}
            \end{itemize}
        \column{.5\textwidth}
            \begin{itemize}
               \item{De tiempo discreto}
               \item{No deterministas}
               \item{Impares}
               \item{Aperiódicas}
               \item{De potencia}
               \item{Imaginarias}
            \end{itemize}
      \end{columns}
    \end{frame}
%-------------------------------------------------------------------------------
    \begin{frame}{Señales y sistemas}
      \framesubtitle{Tipos de señales}
      \begin{columns}[onlytextwidth]
        \column{.45\textwidth}
            \begin{itemize}
               \item{De tiempo continuo}
      \end{itemize}
            Tiene valores para todos los puntos en el tiempo en algún intervalo (posiblemente infinito)
        \column{.45\textwidth}
            \begin{itemize}
               \item{De tiempo discreto}
            \end{itemize}
         Tiene valores solo para puntos discretos en el tiempo
         \end{columns}
      \vfill
      \includegraphics[width=\textwidth]{1_clase/continuo_vs_discreto}
    \end{frame}

    \subsection{Generacion de señales en Python}
    \begin{frame}{Generacion de señales en Python}{Continuo? vs discreto}
      \lstset{ basicstyle=\fontsize{ 8}{ 2}\selectfont\ttfamily }
      \begin{columns}[onlytextwidth]
        \column{.6\textwidth}
      \lstinputlisting[language=Python,tabsize=4]{1_clase/sine.py}
        \column{.4\textwidth}
         \includegraphics[width=\textwidth]{1_clase/python_continuo_vs_discreto}
            \end{columns}
            Posrian pensarse como muestras de una señal de tiempo continuo $x[n] = x (nT)$ donde n es un número entero y \textbf{T} es el período de muestreo.
         \vfill
       \end{frame}
%-------------------------------------------------------------------------------
       \begin{frame}{Señales periodicas}
         \begin{block}{Continua periodica}
            si existe un $T_0>0$, tal que $x(t+T_0)=x(t)$, para todo $t$\\
            $T_0$ es el período de $x(t)$ medido en tiempo, y $f_0=1/T_0$ es la frecuencia fundamental de $x(t)$
      \end{block}
         \begin{block}{Continua discreta}
            si existe un entero $N_0>0$ tal que $x[n+N_0]=x[n]$ para
            todo $n$ \\
            $N_0$ es el período fundamental de $x[n]$ medido en espacio entre muestras
            y  $F_0=\Delta t/N_0$ es la frecuencia fundamental de $x[n]$
      \end{block}
         \center\includegraphics[width=0.5\textwidth]{1_clase/periodica}
      \vfill
      \end{frame}
%-------------------------------------------------------------------------------
      \subsection{Sistemas}
      \begin{frame}{Sistemas}
         \begin{block}{Sistema}
            Un sistema es cualquier conjunto físico de componentes que actúan en una señal, tomando una o más señales de entrada, y produciendo una o más señales de salida.
         \end{block}
            En términos de ingeniería, muy a menudo la entrada y la salida son señales eléctricas.
         \vfill
      \end{frame}
   %-------------------------------------------------------------------------------
      \begin{frame}{Sistemas}{Lineales}
         \begin{block}{Lineal}
            Un sistema es lineal cuando su salida depende linealmente de la entrada.
            Satisface el principio de superposicion, escalado y adicion
         \end{block}
         \center\includegraphics[width=0.5\textwidth]{1_clase/superposicion1}
         \center\includegraphics[width=0.5\textwidth]{1_clase/superposicion2}
         \center\includegraphics[width=0.5\textwidth]{1_clase/superposicion3}
         \vfill
      \end{frame}
   %-------------------------------------------------------------------------------
      \begin{frame}{Sistemas}{Invariantes en el tiempo}
         \begin{block}{Invariantes en el tiempo}
            Un sistema es invariante en el tiempo cuando la salida para una determinada entrada es la misma sin importar el tiempo en el cual se aplica la entrada
         \end{block}
         \center\includegraphics[width=1\textwidth]{1_clase/invariante_en_tiempo}
         \vfill
      \end{frame}
   %-------------------------------------------------------------------------------
      \begin{frame}{Sistemas}{Lineales invariantes en el tiempo}
         \begin{block}{LTI}
            Un sistema es LTI cuando satisface las 2 condiciones anteriores, de linealidad y de invariancia en el tiempo.
         \end{block}
         \center\includegraphics[width=1\textwidth]{1_clase/lti}
         \vfill
         \begin{alertblock}{*** LTI ***}
            En este curso, \alert{solo} estudiaremos sistemas lineales invariantes en el tiempo.
         \end{alertblock}
      \end{frame}
   %-------------------------------------------------------------------------------
       \section{ADC}
      \begin{frame}{ADC}{Bloque generico de procesamiento}
         \center\includegraphics[width=1\textwidth]{1_clase/adc_dac}
         \begin{alertblock} {Porque el filtro antialising?}
         \end{alertblock}
         \vfill
      \end{frame}

%-------------------------------------------------------------------------------
       \subsection{Aliasing}
      \begin{frame}{Aliasing}{Simulando en Python}
         \handsonicon
         Diferentes frecuencias de sampleo para capturar una señal de 50hz
         \lstset{ basicstyle=\fontsize{ 8}{ 2}\selectfont\ttfamily }
         \lstinputlisting[language=Python,tabsize=4]{1_clase/teorema_sampleo.py}
         \vfill
      \end{frame}
%-------------------------------------------------------------------------------
      \begin{frame}{Aliasing}{Simulando en Python}
         Diferentes frecuencias de sampleo para capturar una señal de 50hz
         \center\includegraphics[width=1.0\textwidth]{1_clase/teorema_sampleo}
         \vfill
      \end{frame}
      \begin{frame}{Aliasing}{Simulando en Python}
         Que pasa si se suma ruido de alta frecuencia?
         \center\includegraphics[width=1.0\textwidth]{1_clase/teorema_sampleo2}
         \vfill
      \end{frame}
%-------------------------------------------------------------------------------
      \subsection{Teorema de Shannon}
      \begin{frame}{Teorema de sampleo}{Teorema de Shannon}
         \begin{teorema}
          La reconstrucción exacta de una señal periódica continua en banda base a partir de sus muestras, es matemáticamente posible si la señal está limitada en banda y la tasa de muestreo es superior al doble de su ancho de banda
         \end{teorema}
         \begin{columns}[onlytextwidth]
            \column{.6\textwidth}
            \center\includegraphics[width=0.8\textwidth]{1_clase/shannon}
            \column{.4\textwidth}
            \center\includegraphics[width=0.6\textwidth]{1_clase/claude_shannon}
         \end{columns}
         \vfill
      \end{frame}
%-------------------------------------------------------------------------------
      \begin{frame}{Teorema de sampleo}{Teorema de Shannon}
         \handsonicon
         Sampleo e interpolado
         \lstset{ basicstyle=\fontsize{ 7}{ 1}\selectfont\ttfamily }
         \lstinputlisting[language=Python,tabsize=4]{1_clase/teorema_sampleo_interpolado.py}
         \vfill
      \end{frame}
%-------------------------------------------------------------------------------
      \begin{frame}{Teorema de sampleo}{Teorema de Shannon}
         \handsonicon
         Sampleo e interpolado
         \center\includegraphics[width=1.0\textwidth]{1_clase/teorema_sampleo_interpolado}
         \vfill
      \end{frame}
%-------------------------------------------------------------------------------
      \begin{frame}{Sampleo}{Filtro antialias}
         \begin{block}{FAA}
            Filtro \alert{analogico} pasabajos que elimina o al menos mitiga el efecto de aliasing
         \end{block}
         \center\includegraphics[width=0.8\textwidth]{1_clase/filtro_anti_aliasing}
         \vfill
      \end{frame}
%-------------------------------------------------------------------------------
      \begin{frame}{Sampleo}{Acondicionamiento de señal}
         Acondicionar la señal de salida del dispositivo de sonido (en PC ronda $\pm1V$) al rango del ADC del hardware. En el caso de la CIAA sera de 0-3.3V. \\ Se propone el siguiente circuito, que minimiza los componentes sacrificando calidad y agrega en filtro anti alias de 1er orden.
         \protoboardicon
         \center\includegraphics[width=0.8\textwidth]{1_clase/circuito}
         \vfill
      \end{frame}
%-------------------------------------------------------------------------------
      \section{Quantizacion}
      \subsection{Ejemplos}
      \begin{frame}{Ruido de cuantizacion}{Ejemplo de cuantizacion}
         Diferentes formas de onda cuantizadas
         \center\includegraphics[width=1\textwidth]{1_clase/noise_examples}
         \vfill
      \end{frame}
%-------------------------------------------------------------------------------
      \begin{frame}{Ruido de cuantizacion}{Cuantizacion en python}
         \handsonicon
         \lstset{ basicstyle=\fontsize{ 9}{ 2}\selectfont\ttfamily }
         \lstinputlisting[language=Python,tabsize=4]{1_clase/noise_model.py}
         \vfill
      \end{frame}
%-------------------------------------------------------------------------------
      \begin{frame}{Ruido de cuantizacion}{Histogramas}
         Histogramas de ruido para cada señal
         \center\includegraphics[width=1\textwidth]{1_clase/noise_histogram}
         \vfill
      \end{frame}
%-------------------------------------------------------------------------------
      \begin{frame}{Ruido de cuantizacion}{Histogramas}
         Histogramas en Python
         \handsonicon
         \lstset{ basicstyle=\fontsize{ 9}{ 2}\selectfont\ttfamily }
         \lstinputlisting[language=Python,tabsize=4]{1_clase/noise_histogram.py}
         \vfill
      \end{frame}
%-------------------------------------------------------------------------------
      \subsection{Modelo estadistico}
      \begin{frame}{Ruido de cuantizacion}{Modelo estadistico}
         En el caso de que se cumplan las siguientes premisas:
         \begin{itemize}
                \item La entrada se distancia de los diferentes niveles de cuantizacion con igual probabilidad
                \item El error de cuantizacion NO esta correlacionado con la entrada
                \item El cuantizador cuanta con un numero relativamente largo de niveles
                \item Los niveles de cuantizacion son uniformes
         \end{itemize}
            Se puede considerar la cuantizacion como un ruido aditivo a la señal segun el siguiente esquema:
      \center\includegraphics[width=0.5\textwidth]{1_clase/noise_model}
      \vfill
   \end{frame}
%-------------------------------------------------------------------------------
   \subsection{Modelo estadistico}
   \begin{frame}{Ruido de cuantizacion}{Funcion densidad de probabilidad}
      \begin{columns}[onlytextwidth]
         \column{.5\textwidth}
         \center\includegraphics[width=0.7\textwidth]{1_clase/noise_funcion_probabilidad}
         \column{.5\textwidth}
         \begin{align*}
            \int^\frac{lsb}{2}_{-\frac{lsb}{2}} p(e) de = 1 \\
         \end{align*}
      \end{columns}
      \vfill
   \end{frame}
%-------------------------------------------------------------------------------
   \begin{frame}{Ruido de cuantizacion}{Funcion densidad de probabilidad}
      \begin{columns}[onlytextwidth]
         \column{.5\textwidth}
         \begin{align*}
            P_q &= \int^\frac{lsb}{2}_{-\frac{lsb}{2}} e^2 p(e) de \\
            P_q &= \int^\frac{lsb}{2}_{-\frac{lsb}{2}} e^2 \frac{1}{lsb} de \\
            P_q &= \frac{1}{lsb} \frac{e^3}{3} \Big\rvert^{\frac{lsb}{2}}_{-\frac{lsb}{2}}
         \end{align*}
         \column{.5\textwidth}
         \begin{align*}
            P_q &= \frac{1}{lsb} \frac{(\frac{lsb}{2})^3}{3} - \frac{(\frac{-lsb}{2})^3}{3} \\
            P_q &= \frac{1}{lsb} \frac{lsb^3}{24} + \frac{lsb^3}{24} \\
         \end{align*}
      \end{columns}
      \begin{block}{Potencia de ruido de cuantizacion}
         \begin{equation}
            P_q = \frac{lsb^2}{12}
         \end{equation}
      \end{block}
      \vfill
   \end{frame}
%-------------------------------------------------------------------------------
   \subsection{Densidad espectral}
   \begin{frame}{Ruido de cuantizacion}{Densidad espectral de potencia de ruido}
      Si condideramos la potencia de ruido uniformemente distribuido en todo el espectro desde DC hasta la mitad de la frecuancia de sampleo, nos queda que:
      \begin{columns}[onlytextwidth]
         \column{.5\textwidth}
         \begin{align*}
            S_{medio\ espectro}(f) &= \frac{P_q}{\frac{Fs}{2}} \\
            S_{medio\ espectro}(f) &= \frac{lsb^2}{6*Fs}
         \end{align*}
         \column{.5\textwidth}
      \end{columns}
      \vfill
   \end{frame}
%-------------------------------------------------------------------------------
   \subsection{SNR}
   \begin{frame}{Ruido de cuantizacion}{Relacion señal a ruido}
      \begin{columns}[onlytextwidth]
         \column{.5\textwidth}
         \begin{align*}
         input&=\frac{Amp}{2}\sin(t) \\
         P_{input} &= \frac{1}{T} \int^T_0 \left(\frac{Amp}{2}\sin(t)\right)^2 \\
         P_{input} &= \frac{1}{T} \left(\frac{Amp}{2}\right)^2* \left( \frac{t}{2}-\frac{\sin(4t)}{4}\right)\Big\rvert^T_0 \\
         P_{input} &= \frac{Amp^2}{4T} \frac{T}{2}\\
         P_{input} &= \frac{Amp^2}{8} 
      \end{align*}
         \column{.5\textwidth}
         \begin{align*}
            lsb       &= \frac{Amp}{2^N} \\
            P_{ruido} &= \frac{lsb^2}{12}\\
            P_{ruido} &= \frac{\left(\frac{Amp}{2^N}\right)^2}{12}\\
            P_{ruido} &= \frac{Amp^2}{12*2^{2N}}\\
         \end{align*}
      \end{columns}
      \vfill
   \end{frame}
%-------------------------------------------------------------------------------
   \begin{frame}{Ruido de cuantizacion}{Relacion señal a ruido}
      \begin{columns}[onlytextwidth]
         \column{.5\textwidth}
         \begin{align*}
            SNR&=10 \log_{10} \left(\frac{P_{input}}{P_{ruido}} \right)\\
            SNR&=10 \log_{10}\left(\frac{\frac{Amp^2}{8}}{\frac{Amp^2}{12*2^{2N}}} \right) \\
         \end{align*}
         \column{.5\textwidth}
         \begin{align*}
            SNR&=10 \log_{10}\left(\frac{3*2^{2N}}{2} \right)\\
            SNR&=10\log_{10}\left(\frac{3}{2}\right)-10\log_{10}\left(2^{2N}\right)
         \end{align*}
      \end{columns}
      \begin{block}{SNR}
         \begin{equation}
            SNR = 1.76 + 6.02 * N
         \end{equation}
      \end{block}
      \begin{centering}
         \resizebox{0.2\textwidth}{!}{$SNR_{N=10} \approx 60dB$} \\
         \resizebox{0.2\textwidth}{!}{$SNR_{N=11} \approx 66dB$} \\
      \end{centering}
   \end{frame}
%-------------------------------------------------------------------------------
   \subsection{Densidad espectral}
   \begin{frame}{Ruido de cuantizacion}{Densidad espectral de potencia de ruido}
      Si condideramos la potencia de ruido uniformemente distribuido en todo el espectro desde $-Fs$ hasta $+Fs$, nos queda que:
      \begin{block}{Densidad espectral de potencia de ruido}
         \begin{align*}
            S_{espectral}(f) = \frac{P_q}{Fs}
         \end{align*}
      \end{block}
      Entonces como puedo mejorar la SNR de un sistema?
      \vfill
   \end{frame}



















































%    \begin{frame}[label=lists]{listas y columnas}
%      \framesubtitle{Que es una señal?}
%      \begin{columns}[onlytextwidth]
%        \column{.5\textwidth}
%          \begin{itemize}
%            \item {Una señal, en función de una o más variables, puede definirse como un cambio observable en una entidad cuantificable}
%            \begin{itemize}
%              \item Fusce id sodales dolor. Sed id metus dui.
%              \begin{itemize}
%                \item Cupio virtus licet mi vel feugiat.
%              \end{itemize}
%            \end{itemize}
%          \end{itemize}
%        \column{.5\textwidth}
%          \begin{enumerate}
%            \item Donec porta, risus porttitor egestas scelerisque video.
%            \begin{enumerate}
%              \item Nunc non ante fringilla, manus potentis cario.
%              \begin{enumerate}
%                \item Pellentesque servus morbi tristique.
%              \end{enumerate}
%            \end{enumerate}
%          \end{enumerate}
%      \end{columns}
%      \bigskip
%      \justifying
%
%      {\uselanguage{spanish}The quick, brown fox jumps over a lazy
%      dog. DJs flock by when MTV ax quiz prog. “Now fax quiz Jack!”}
%    \end{frame}
%
%    \subsection{Structuring Elements}
%    \begin{frame}[label=simmonshall]{Text blocks}
%      \framesubtitle{In plain, example, and \alert{alert} flavour}
%      \alert{This text} is highlighted.
%
%      \begin{block}{A plain block}
%        This is a plain block containing some \alert{highlighted text}.
%      \end{block}
%      \begin{exampleblock}{An example block}
%        This is an example block containing some \alert{highlighted text}.
%      \end{exampleblock}
%      \begin{alertblock}{An alert block}
%        This is an alert block containing some \alert{highlighted text}.
%      \end{alertblock}
%    \end{frame}
%
%    \begin{frame}[label=proof]{Definitions, theorems, and proofs}
%      \framesubtitle{All integers divide zero}
%      \begin{definition}
%        $\forall a,b\in\mathds{Z}: a\mid b\iff\exists c\in\mathds{Z}:a\cdot c=b$
%      \end{definition}
%      \begin{theorem}
%        $\forall a\in\mathds{Z}: a\mid 0$
%      \end{theorem}
%      \begin{proof}[Proof\nopunct]
%        $\forall a\in\mathds{Z}: a\cdot 0=0$
%      \end{proof}
%    \end{frame}
%
%    \subsection{Numerals and Mathematics}
%    \begin{frame}[label=math]{Numerals and Mathematics}
%      \framesubtitle{Formulae, equations, and expressions}
%      \begin{columns}[onlytextwidth]
%        \column{.20\textwidth}
%          1234567890
%        \column{.20\textwidth}
%          \oldstylenums{1234567890}
%        \column{.20\textwidth}
%          $\hat{x}$, $\check{x}$, $\tilde{a}$,
%          $\bar{a}$, $\dot{y}$, $\ddot{y}$
%        \column{.40\textwidth}
%          $\int \!\! \int f(x,y,z)\,\mathsf{d}x\mathsf{d}y\mathsf{d}z$
%      \end{columns}
%      \begin{columns}[onlytextwidth]
%        \column{.5\textwidth}
%          $$\frac{1}{\displaystyle 1+
%            \frac{1}{\displaystyle 2+
%            \frac{1}{\displaystyle 3+x}}} +
%            \frac{1}{1+\frac{1}{2+\frac{1}{3+x}}}$$
%        \column{.5\textwidth}
%          $$F:\left| \begin{array}{ccc}
%          F''_{xx} & F''_{xy} &  F'_x \\
%          F''_{yx} & F''_{yy} &  F'_y \\
%          F'_x     & F'_y     & 0
%          \end{array}\right| = 0$$
%      \end{columns}
%      \begin{columns}[onlytextwidth]
%        \column{.3\textwidth}
%          $$\mathop{\int \!\!\! \int}_{\mathbf{x} \in \mathds{R}^2}
%          \! \langle \mathbf{x},\mathbf{y}\rangle\,\mathsf{d}\mathbf{x}$$
%        \column{.33\textwidth}
%          $$\overline{\overline{a\alpha}^2+\underline{b\beta}
%           +\overline{\overline{d\delta}}}$$
%        \column{.37\textwidth}
%          $\left] 0,1\right[ + \lceil x \rfloor - \langle x,y\rangle$
%      \end{columns}
%      \begin{columns}[onlytextwidth]
%        \column{.4\textwidth}
%          \begin{eqnarray*}
%           e^x &\approx& 1+x+x^2/2! + \\
%             && {}+x^3/3! + x^4/4!
%          \end{eqnarray*}
%        \column{.6\textwidth}
%          $${n+1\choose k} = {n\choose k} + {n \choose k-1}$$
%      \end{columns}
%    \end{frame}
%
%    \subsection{Figures and Code Listings}
%    \begin{frame}[label=figs1]{Figures}
%      \framesubtitle{Tables, graphs, and images}
%      \begin{table}[!b]
%        {\carlitoTLF % Use monospaced lining figures
%        \begin{tabularx}{\textwidth}{Xrrr}
%          \textbf{Faculty} & \textbf{With \TeX} & \textbf{Total} &
%          \textbf{\%} \\
%          \toprule
%          Faculty of Informatics       & 1\,716  & 2\,904  &
%          59.09 \\% 1433
%          Faculty of Science           & 786     & 5\,275  &
%          14.90 \\% 1431
%          Faculty of $\genfrac{}{}{0pt}{}{\textsf{Economics and}}{%
%          \textsf{Administration}}$    & 64      & 4\,591  &
%          1.39  \\% 1456
%          Faculty of Arts              & 69      & 10\,000 &
%          0.69  \\% 1421
%          Faculty of Medicine          & 8       & 2\,014  &
%          0.40  \\% 1411
%          Faculty of Law               & 15      & 4\,824  &
%          0.31  \\% 1422
%          Faculty of Education         & 19      & 8\,219  &
%          0.23  \\% 1441
%          Faculty of Social Studies    & 12      & 5\,599  &
%          0.21  \\% 1423
%          Faculty of Sports Studies    & 3       & 2\,062  &
%          0.15  \\% 1451
%          \bottomrule
%        \end{tabularx}}
%        \caption{The distribution of theses written using \TeX\ during 2010--15 at MU}
%      \end{table}
%    \end{frame}
%
%    \begin{frame}[label=figs2]{Figures}
%      \framesubtitle{Tables, graphs, and images}
%      \begin{figure}[b]
%        \centering
%        % Flipping a coin
%        % Author: cis
%        \tikzset{
%          head/.style = {fill = none, label = center:\textsf{H}},
%          tail/.style = {fill = none, label = center:\textsf{T}}}
%        \scalebox{0.65}{\begin{tikzpicture}[
%            scale = 1.5, transform shape, thick,
%            every node/.style = {draw, circle, minimum size = 10mm},
%            grow = down,  % alignment of characters
%            level 1/.style = {sibling distance=3cm},
%            level 2/.style = {sibling distance=4cm},
%            level 3/.style = {sibling distance=2cm},
%            level distance = 1.25cm
%          ]
%          \node[shape = rectangle,
%            minimum width = 6cm, font = \sffamily] {Coin flipping}
%          child { node[shape = circle split, draw, line width = 1pt,
%                  minimum size = 10mm, inner sep = 0mm, rotate = 30] (Start)
%                  { \rotatebox{-30}{H} \nodepart{lower} \rotatebox{-30}{T}}
%           child {   node [head] (A) {}
%             child { node [head] (B) {}}
%             child { node [tail] (C) {}}
%           }
%           child {   node [tail] (D) {}
%             child { node [head] (E) {}}
%             child { node [tail] (F) {}}
%           }
%          };
%
%          % Filling the root (Start)
%          \begin{scope}[on background layer, rotate=30]
%            \fill[head] (Start.base) ([xshift = 0mm]Start.east) arc (0:180:5mm)
%              -- cycle;
%            \fill[tail] (Start.base) ([xshift = 0pt]Start.west) arc (180:360:5mm)
%              -- cycle;
%          \end{scope}
%
%          % Labels
%          \begin{scope}[nodes = {draw = none}]
%            \path (Start) -- (A) node [near start, left]  {$0.5$};
%            \path (A)     -- (B) node [near start, left]  {$0.5$};
%            \path (A)     -- (C) node [near start, right] {$0.5$};
%            \path (Start) -- (D) node [near start, right] {$0.5$};
%            \path (D)     -- (E) node [near start, left]  {$0.5$};
%            \path (D)     -- (F) node [near start, right] {$0.5$};
%            \begin{scope}[nodes = {below = 11pt}]
%              \node [name = X] at (B) {$0.25$};
%              \node            at (C) {$0.25$};
%              \node [name = Y] at (E) {$0.25$};
%              \node            at (F) {$0.25$};
%            \end{scope}
%          \end{scope}
%        \end{tikzpicture}}
%        \caption{Tree of probabilities -- Flipping a coin\footnote[frame]{%
%          A derivative of a diagram from \url{texample.net} by cis, CC BY 2.5 licensed}}
%      \end{figure}
%    \end{frame}
%
%    \defverbatim[colored]\sleepSort{
%      \begin{lstlisting}[language=C,tabsize=2]
%  #include <stdio.h>
%  #include <unistd.h>
%  #include <sys/types.h>
%  #include <sys/wait.h>
%
%  // This is a comment
%  int main(int argc, char **argv)
%  {
%          while (--c > 1 && !fork());
%          sleep(c = atoi(v[c]));
%          printf("%d\n", c);
%          wait(0);
%          return 0;
%  }
%    \end{lstlisting}}
%    \begin{frame}{Code listings}{An example source code in C}
%      \sleepSort
%    \end{frame}
%
%    \subsection{Citations and Bibliography}
%    \begin{frame}[label=citations]{Citations}
%      \framesubtitle{\TeX, \LaTeX, and Beamer}
%
%      \justifying\TeX\ is a programming language for the typesetting
%      of documents. It was created by Donald Erwin Knuth in the late
%      1970s and it is documented in \emph{The \TeX
%      book}~\cite{knuth84}.
%
%      In the early 1980s, Leslie Lamport created the initial version
%      of \LaTeX, a high-level language on top of \TeX, which is
%      documented in \emph{\LaTeX : A Document Preparation
%      System}~\cite{lamport94}. There exists a healthy ecosystem of
%      packages that extend the base functionality of \LaTeX;
%      \emph{The \LaTeX\ Companion}~\cite{MG94} acts as a guide
%      through the ecosystem.
%
%      In 2003, Till Tantau created the initial version of Beamer, a
%      \LaTeX\ package for the creation of presentations. Beamer is
%      documented in the \emph{User's Guide to the Beamer
%      Class}~\cite{tantau04}.
%    \end{frame}
%
%    \begin{frame}[label=bibliography]{Bibliography}
%      \framesubtitle{\TeX, \LaTeX, and Beamer}
%      \begin{thebibliography}{9}
%        \bibitem{knuth84}
%            Donald~E.~Knuth.
%            \emph{The \TeX book}.
%            Addison-Wesley, 1984.
%        \bibitem{lamport94}
%            Leslie~Lamport.
%            \emph{\LaTeX : A Document Preparation System}.
%            Addison-Wesley, 1986.
%        \bibitem{MG94}
%            M.~Goossens, F.~Mittelbach, and A.~Samarin.
%            \emph{The \LaTeX\ Companion}.
%            Addison-Wesley, 1994.
%        \bibitem{tantau04}
%            Till~Tantau.
%            \emph{User's Guide to the Beamer Class Version 3.01}.
%            Available at \url{http://latex-beamer.sourceforge.net}.
%        \bibitem{MS05}
%            A.~Mertz and W.~Slough.
%            Edited by B.~Beeton and K.~Berry.
%            \emph{Beamer by example} In TUGboat,
%              Vol. 26, No. 1., pp. 68-73.
%      \end{thebibliography}
%    \end{frame}

  \end{darkframes}

\end{document}
