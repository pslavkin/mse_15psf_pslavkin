\subtitle{Clase 3 - Euler | Fourier - DFT}

\begin{frame}[c]
\maketitle
\begin{tikzpicture}[overlay,remember picture]
    \node[anchor=south east,xshift=-30pt,yshift=45pt]
      at (current page.south east) {
        \includegraphics[width=0.2\textwidth]{3_clase/leonhard_euler}
        \includegraphics[width=0.2\textwidth]{3_clase/joseph_fourier}
      };
  \end{tikzpicture}
\end{frame}
%-------------------------------------------------------------------------------
\begin{frame}{2.7182818284590450907955982984276488423347473144}
   \begin{columns}[onlytextwidth]
      \column{.25\textwidth}
      \begin{itemize}
         \item $f(t) =t$
         \item $f(t) =t^2$
         \item $f(t) =sin(t)$
      \end{itemize}
      \column{.25\textwidth}
      \begin{itemize}
         \item $f'(t) = 1$
         \item $f'(t) = 2*t$
         \item $f'(t) = cos(t)$
      \end{itemize}
      \column{.4\textwidth}
      \includegraphics[width=0.8\textwidth]{3_clase/derivadas}
      %codigos en 3_clase/derivadas.py
   \end{columns}
   \begin{block}{La derivada es igual a la funcion}
      \begin{center}
         $f(t) =e^t    \implies f'(t) = e^t$\\
         $f(t) =e^{kt} \implies f'(t) =ke^{kt}$
      \end{center}
   \end{block}
   \vfill
\end{frame}
%-------------------------------------------------------------------------------
\begin{frame}{$e^{j2\pi ft}$}{Euler}
   Pero que pasa con $\LARGE\textcolor{green}{e^{jt}}$?
   \begin{columns}[onlytextwidth]
      \column{.5\textwidth}
   \begin{block}{La derivada es igual a la funcion}
      \begin{center}
         $f(t) =e^{jt} \implies f'(t) =je^{jt}$
      \end{center}
   \end{block}
      \begin{align*}
         e^{jt}   &= \cos(t) + j \sin(t)\\
         e^{j\pi} &= -1\\
         e^{\frac{j \pi}{2}} &= j\\
         e^{\frac{j3\pi}{2}} &= -j\\
      \end{align*}
      \column{.5\textwidth}
      \centering\includegraphics[width=0.8\textwidth]{3_clase/euler_senos_cosenos}
   \end{columns}
   \vfill
\end{frame}
%-------------------------------------------------------------------------------
\begin{frame}[t]{$e^{j2\pi ft}$}{$e^{j2\pi ft}$ animado }
   \handsonicon
   \begin{columns}[onlytextwidth]
      \column{.6\textwidth}
      \lstset{ basicstyle=\fontsize{ 7}{ 1}\selectfont\ttfamily }
      \lstinputlisting[language=Python,tabsize=4]{3_clase/euler1.py}
      \column{.4\textwidth}
      \centering\includegraphics[width=1.0\textwidth]{3_clase/euler1}
   \end{columns}
   \vfill
\end{frame}
%-------------------------------------------------------------------------------
\begin{frame}{$e^{j2\pi ft}$}{$e^{j2\pi ft}$ y $\sin(t)$ animados independientemente}
   \handsonicon
       \lstset{ basicstyle=\fontsize{ 5}{ 1}\selectfont\ttfamily,language=Python,tabsize=4}
       \begin{columns}[c]
          \hspace{2pt}
          \begin{column}{.3\textwidth}
             \lstinputlisting[lastline=27]{3_clase/euler2.py}
          \end{column}
          \hspace{2pt}
          \vrule
          \hspace{2pt}
          \begin{column}{.3\textwidth}
             \lstinputlisting[firstline=27]{3_clase/euler2.py}
          \end{column}
          \hspace{2pt}
          \vrule
          \hspace{2pt}
          \begin{column}{.4\textwidth}
             \centering\includegraphics[width=1.0\textwidth]{3_clase/euler2}
          \end{column}
          \hspace{2pt}
       \end{columns}
   \vfill
\end{frame}
%-------------------------------------------------------------------------------
\begin{frame}{$e^{j2\pi ft}$}{$e^{j2\pi ft}$ modulado por $\sin(t)$ y centro de masas }
   \handsonicon
   \lstset{ basicstyle=\fontsize{ 5}{ 1}\selectfont\ttfamily,language=Python,tabsize=4}
   \begin{columns}[c]
      \hspace{2pt}
      \begin{column}{.3\textwidth}
         \lstinputlisting[lastline=29]{3_clase/euler2.py}
      \end{column}
      \hspace{2pt}
      \vrule
      \hspace{2pt}
      \begin{column}{.3\textwidth}
         \lstinputlisting[firstline=29]{3_clase/euler3.py}
      \end{column}
      \hspace{2pt}
      \vrule
      \hspace{2pt}
      \begin{column}{.4\textwidth}
         \centering\includegraphics[width=1.0\textwidth]{3_clase/euler3}
      \end{column}
      \hspace{2pt}
   \end{columns}
   \vfill
\end{frame}
%-------------------------------------------------------------------------------
\begin{frame}{$e^{j2\pi ft}$}{$e^{j2\pi ft}$ modulado por $\sin(t)$ y centro de masas en f, DFT?}
   \handsonicon
   \lstset{ basicstyle=\fontsize{ 3}{ 1}\selectfont\ttfamily,language=Python,tabsize=4}
   \begin{columns}[c]
      \hspace{2pt}
      \begin{column}{.3\textwidth}
         \lstinputlisting[lastline=44]{3_clase/euler4.py}
      \end{column}
      \hspace{2pt}
      \vrule
      \hspace{2pt}
      \begin{column}{.3\textwidth}
         \lstinputlisting[firstline=45]{3_clase/euler4.py}
      \end{column}
      \hspace{2pt}
      \vrule
      \hspace{2pt}
      \begin{column}{.4\textwidth}
         \centering\includegraphics[width=1.0\textwidth]{3_clase/euler4}
      \end{column}
      \hspace{2pt}
   \end{columns}
   \vfill
\end{frame}
%-------------------------------------------------------------------------------
 \begin{frame}{Transformada de Fourier}{Diferentes tipos segun la señal}
   \begin{columns}[c]
      \begin{column}{.45\textwidth}
    \includegraphics[width=0.9\textwidth]{3_clase/fourier_types}
      \end{column}
      \begin{column}{.55\textwidth}
    \includegraphics[width=1.0\textwidth]{3_clase/fourier_table}
      \end{column}
   \end{columns}
    \vfill
 \end{frame}
%-------------------------------------------------------------------------------
 \begin{frame}{Analisi, Transformada discreta de Fourier}{DFT}
    \center\includegraphics[width=0.9\textwidth]{3_clase/dft_eq}
    \vfill
 \end{frame}

%-------------------------------------------------------------------------------
 \begin{frame}{DFT}{ Densidad de potencia espectral }
    \handsonicon
    \begin{columns}[c]
       \begin{column}{.2\textwidth}
          \begin{align*}
             P_{sin}&=\frac{A^2}{2}\\
             P_{sin}&=\frac{2.5^2}{2}\\
             P_{sin}&=3.125W\\
             P&=1.56+1.56\\
             P&=3.125W
          \end{align*}
       \end{column}
       \begin{column}{.8\textwidth}
          \centering\includegraphics[width=0.55\textwidth]{3_clase/densidad_potencia}
       \end{column}
    \end{columns}
    \vfill
 \end{frame}
%-------------------------------------------------------------------------------
 \begin{frame}{DFT}{Fuga espectral (Spectral leakage)}
   \handsonicon
   \begin{columns}[c]
      \begin{column}{.5\textwidth}
         10Hz
         \centering\includegraphics[width=1.0\textwidth]{3_clase/desparramo2}
      \end{column}
      \begin{column}{.5\textwidth}
         10.4hz
         \centering\includegraphics[width=1.0\textwidth]{3_clase/desparramo1}
      \end{column}
   \end{columns}
   \vfill
\end{frame}
%-------------------------------------------------------------------------------
\begin{frame}{DFT}{Resolucion espectral}
   \handsonicon
   \begin{columns}[t]
      \begin{column}{.5\textwidth}
         f1=2 \hspace{0.1cm}  f2=3 \hspace{0.1cm} fs = 30 \hspace{0.1cm} N  = 50\\
         \centering\includegraphics[width=1.0\textwidth]{3_clase/resolucion1}
      \end{column}
      \begin{column}{.5\textwidth}
         f1=2 \hspace{0.1cm}  f2=3 \hspace{0.1cm} fs = 50 \hspace{0.1cm} N  = 50\\
         \centering\includegraphics[width=1.0\textwidth]{3_clase/resolucion2}
      \end{column}
   \end{columns}
   \vfill
\end{frame}
%-------------------------------------------------------------------------------
\begin{frame}[c]{DFT}{DFT Zero padding}
   \handsonicon
   \begin{columns}[t]
      \begin{column}[t]{.2\textwidth}
         \begin{align*}
            f1 &= 2\\
            f2 &= 3\\
            fs &= 10\\
            N  &= 50\\
            Z  &= 950
         \end{align*}
      \end{column}
      \begin{column}[t]{.8\textwidth}
         \center\includegraphics[width=0.55\textwidth]{3_clase/zero_padding}
      \end{column}
   \end{columns}
   \vfill
\end{frame}
%-------------------------------------------------------------------------------
\begin{frame}{DFT}{Acelerada}
   \handsonicon
   \lstset{ basicstyle=\fontsize{ 3}{ 1}\selectfont\ttfamily,language=Python,tabsize=4}
   \begin{columns}[c]
      \hspace{2pt}
      \begin{column}{.3\textwidth}
         \lstinputlisting[lastline=49]{3_clase/euler5.py}
      \end{column}
      \hspace{2pt}
      \vrule
      \hspace{2pt}
      \begin{column}{.3\textwidth}
         \lstinputlisting[firstline=49]{3_clase/euler5.py}
      \end{column}
      \hspace{2pt}
      \vrule
      \hspace{2pt}
      \begin{column}{.4\textwidth}
         \centering\includegraphics[width=1.0\textwidth]{3_clase/euler5}
      \end{column}
      \hspace{2pt}
   \end{columns}
   \vfill
\end{frame}
%-------------------------------------------------------------------------------
\section{CIAA}
\subsection{CMSIS-DSP}
\begin{frame}{DFT para señales reales RDFT}
   \begin{columns}[c]
      \begin{column}{.5\textwidth}
         Sin
         \centering\includegraphics[width=1.0\textwidth]{3_clase/euler_reales_sin}
      \end{column}
      \begin{column}{.5\textwidth}
         Cos
         \centering\includegraphics[width=1.0\textwidth]{3_clase/euler_reales_cos}
      \end{column}
   \end{columns}
\end{frame}
%-------------------------------------------------------------------------------
\begin{frame}{DFT para señales reales RDFT}
   \protoboardicon
   \center\includegraphics[width=0.8\textwidth]{3_clase/real_dft_idft}
   \vfill
\end{frame}
%-------------------------------------------------------------------------------
\begin{frame}[c]{RDFT}{Análisis en la CIAA}
   \protoboardicon
   \lstset{ basicstyle=\fontsize{ 4}{ 0}\selectfont\ttfamily,language=c,tabsize=4}
   \begin{columns}[c]
      \hspace{2pt}
      \begin{column}{.5\textwidth}
         \lstinputlisting[lastline=30]{3_clase/ciaa/psf1/src/psf.c}
      \end{column}
      \hspace{2pt}
      \vrule
      \hspace{2pt}
      \begin{column}{.5\textwidth}
         \lstinputlisting[firstline=31,lastline=52]{3_clase/ciaa/psf1/src/psf.c}
      \end{column}
   \end{columns}
   \vfill
\end{frame}
%-------------------------------------------------------------------------------
\begin{frame}[c]{RDFT}{Análisis en la CIAA}
   \protoboardicon
   \lstset{ basicstyle=\fontsize{ 5}{ 1}\selectfont\ttfamily,language=c,tabsize=4}
   \begin{columns}[c]
      \begin{column}{.5\textwidth}
         \includegraphics[width=1.0\textwidth]{3_clase/dft_ciaa1}
      \end{column}
      \begin{column}{.5\textwidth}
         \includegraphics[width=1.0\textwidth]{3_clase/dft_ciaa2}
      \end{column}
   \end{columns}
   \vfill
\end{frame}
%-------------------------------------------------------------------------------
\begin{frame}{Bibliografía}
   \framesubtitle{Libros, links y otro material}
   \begin{thebibliography}{9}
         \bibitem{cmsisdsp}
         \emph{ARM CMSIS DSP}. \\
         \href {https://arm-software.github.io/CMSIS_5/DSP/html/index.html}{https://arm-software.github.io/CMSIS\_5/DSP/html/index.html}
         \bibitem{dsp}
         Steven W. Smith.
         \emph{The Scientist and Engineer's Guide to Digital Signal Processing}.
         Second Edition, 1999.
         \bibitem{grants Anderson}
         \emph{Grant Sanderson} \\
         \href{ https://youtu.be/spUNpyF58BY}{ https://youtu.be/spUNpyF58BY}
         \bibitem{interactivemath}
         \emph{Interactive Mathematics Site Info}. \\
         \href {https://www.intmath.com/fourier-series/fourier-intro.php}{https://www.intmath.com/fourier-series/fourier-intro.php}
   \end{thebibliography}
\end{frame}
%-------------------------------------------------------------------------------

