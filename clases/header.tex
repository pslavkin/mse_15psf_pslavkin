%\documentclass[draft]{beamer}
\documentclass[aspectratio=169]{beamer}
%\documentclass[aspectratio=43]{beamer}
\usetheme[faculty=fi]{fibeamer}
\usepackage[utf8]{inputenc}
\usepackage[ main=spanish,english,activeacute]{babel}

%% These additional packages are used within the document:
\usepackage{pdfpages}
\usepackage{ragged2e}  % `\justifying` text
\usepackage{booktabs}  % Tables
\usepackage{graphicx}
\usepackage{tabularx}
\usepackage{tikz}      % Diagrams
\usetikzlibrary{calc, shapes, backgrounds}
\usepackage{amsmath, amssymb}
\usepackage{url}       % `\url`s
\usepackage{listings}  % Code listings
%\frenchspacing    %no se para que es
\usepackage{multiaudience} %para seleccionar que frames sacar en funcio de una variable
\usepackage{pgfpages} %para poner notas de ayuda con un parche para que no se salgan las notas fuera de margen... truchazo
\makeatletter
\defbeamertemplate{note page}{infolines}
{%
  {%
    \scriptsize
    \insertvrule{.25\paperheight}{white!90!black}
    \vskip-.25\paperheight
    \nointerlineskip
    \vbox{
      \hfill\insertslideintonotes{0.25}\hskip-\Gm@rmargin\hskip0pt%
      \vskip-0.25\paperheight%
      \nointerlineskip
      \begin{pgfpicture}{0cm}{0cm}{0cm}{0cm}
        \begin{pgflowlevelscope}{\pgftransformrotate{90}}
          {\pgftransformshift{\pgfpoint{-2cm}{0.2cm}}%
          \pgftext[base,left]{\footnotesize\the\year-\ifnum\month<10\relax0\fi\the\month-\ifnum\day<10\relax0\fi\the\day}}
        \end{pgflowlevelscope}
      \end{pgfpicture}}
    \nointerlineskip
    \vbox to .25\paperheight{\vskip0.5em
      \hbox{\insertshorttitle[width=8cm]}%
      \setbox\beamer@tempbox=\hbox{\insertsection}%
      \hbox{\ifdim\wd\beamer@tempbox>1pt{\hskip4pt\raise3pt\hbox{\vrule
            width0.4pt height7pt\vrule width 9pt
            height0.4pt}}\hskip1pt\hbox{\begin{minipage}[t]{7.5cm}\def\breakhere{}\insertsection\end{minipage}}\fi%
      }%
      \setbox\beamer@tempbox=\hbox{\insertsubsection}%
      \hbox{\ifdim\wd\beamer@tempbox>1pt{\hskip17.4pt\raise3pt\hbox{\vrule
            width0.4pt height7pt\vrule width 9pt
            height0.4pt}}\hskip1pt\hbox{\begin{minipage}[t]{7.5cm}\def\breakhere{}\insertsubsection\end{minipage}}\fi%
      }%
      \setbox\beamer@tempbox=\hbox{\insertshortframetitle}%
      \hbox{\ifdim\wd\beamer@tempbox>1pt{\hskip30.8pt\raise3pt\hbox{\vrule
            width0.4pt height7pt\vrule width 9pt
            height0.4pt}}\hskip1pt\hbox{\insertshortframetitle[width=7cm]}\fi%
      }%
      \vfil}%
  }%
  \vskip.25em
  \nointerlineskip
  \begin{minipage}{1.3\textwidth} % this is an addition
  \insertnote
  \end{minipage}               % this is an addition
}
\makeatother
\setbeamertemplate{note page}[infolines]
%-------------------------------------------------------------------------------
\newcommand{\asignature}{ PDF MSE2020 }
\newcommand{\pythonicon}{
   \begin{tikzpicture}[overlay,remember picture]
      \node[anchor=south east,xshift=-78pt,yshift=0.90\textheight]
      at (current page.south east) {
         \includegraphics[width=7mm]{resources/python_icon}
      };
   \end{tikzpicture}
}
\newcommand{\pythonpic}[4]{
   \begin{center}
      \tiny{#1}\\
      \href{#2}{\includegraphics[width=#3\textwidth]{#4}}
   \end{center}
}

\newcommand{\videoslink}[1]{
   \let\videolink\undefined %primero lo borro porque si llamo con otro parametro en el mismo documento se queja
   \ifthenelse{\equal{#1}{1}}{\newcommand\videolink{https://drive.google.com/file/d/1c2TZLObWEAacB-EzCjFczHrTLzeaOhSI}}{}
   \ifthenelse{\equal{#1}{2}}{\newcommand\videolink{https://drive.google.com/file/d/1dguQYABcfy_BVhJNta_jARjZp1a48pqa}}{}
   \ifthenelse{\equal{#1}{3}}{\newcommand\videolink{https://drive.google.com/file/d/1ZenOaxlZaxevdkB5JrGS_1dV5ME9edIS}}{}
   \ifthenelse{\equal{#1}{4}}{\newcommand\videolink{https://drive.google.com/file/d/1TBlo1pg1OS6e8gbgBsHCAlncdCb4wPtr}}{}
   \ifthenelse{\equal{#1}{5}}{\newcommand\videolink{https://drive.google.com/file/d/1L23UbjTB2uUvOeMPF_yV5QS6gWkCxPnN}}{}
   \ifthenelse{\equal{#1}{6}}{\newcommand\videolink{https://drive.google.com/file/d/195_PoRe4Hd9CqPcAU0gmStt_dX00PJ62}}{}
}

\newcommand{\videoicon}[2]{
   \videoslink{#1}
      \begin{tikzpicture}[overlay,remember picture]
         \node[anchor=south east,xshift=-45pt,yshift=0.89\textheight]
         at (current page.south east) {
           \href{\videolink/view?t=#2}{\includegraphics[width=10mm]{resources/video_icon}}
         };
         \node[anchor=south east,xshift=-47pt,yshift=0.86\textheight]
         at (current page.south east) {
            \tiny{#2}
         };
      \end{tikzpicture}
}
\newcommand{\handsonicon}{
  \begin{tikzpicture}[overlay,remember picture]
     \node[anchor=south east,xshift=-5pt,yshift=0.85\textheight]
     at (current page.south east) {
           \includegraphics[width=12mm]{resources/hands_on}
     };
  \end{tikzpicture}
  }
\newcommand{\protoboardicon}{
  \begin{tikzpicture}[overlay,remember picture]
     \node[anchor=south east,xshift=-5pt,yshift=0.85\textheight]
         at (current page.south east) {
           \includegraphics[width=12mm]{resources/protoboard}
         };
     \end{tikzpicture}
  }
\newtheorem{teorema}{Teorema} %teorema en spanish

%-------------------------------------------------------------------------------
\newcommand{\insertuniversity}{Maestría en sistemas embebidos \\
Universidad de Buenos Aires \\
MSE 5Co2020 }

\newcommand{\insertauthoremail}{\scriptsize{slavkin.pablo@gmail.com \\wapp:011-62433453}}

%% These macros specify information about the presentation
\title{Procesamiento de señales, fundamentos}
\author{Ing. Pablo Slavkin}

% Declare all possible audience groups
\SetNewAudience{clase1}
\SetNewAudience{clase2}
\SetNewAudience{clase3}
\SetNewAudience{clase4}
\SetNewAudience{clase5}
\SetNewAudience{clase6}
\SetNewAudience{tp1}
\SetNewAudience{tp2}
\SetNewAudience{python}
%-------------------------------------------------------------------------------
\abovedisplayskip= 0pt
\belowdisplayskip= 0pt
\abovedisplayshortskip=0pt
\belowdisplayshortskip=7pt
