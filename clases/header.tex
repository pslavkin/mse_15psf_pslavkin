%\documentclass[draft]{beamer}
\documentclass{beamer}
\usetheme[faculty=fi]{fibeamer}
\usepackage[utf8]{inputenc}
\usepackage[ main=spanish,english,activeacute]{babel}

%% These additional packages are used within the document:
\usepackage{ragged2e}  % `\justifying` text
\usepackage{booktabs}  % Tables
\usepackage{graphicx}
\usepackage{tabularx}
\usepackage{tikz}      % Diagrams
\usetikzlibrary{calc, shapes, backgrounds}
\usepackage{amsmath, amssymb}
\usepackage{url}       % `\url`s
\usepackage{listings}  % Code listings
%\frenchspacing    %no se para que es
\usepackage{multiaudience} %para seleccionar que frames sacar en funcio de una variable
%-------------------------------------------------------------------------------
\newcommand{\asignature}{ PDF MSE2020 }
\newcommand{\handsonicon}{
  \begin{tikzpicture}[overlay,remember picture]
     \node[anchor=south east,xshift=-5pt,yshift=0.85\textheight]
     at (current page.south east) {
           \includegraphics[width=15mm]{resources/hands_on}
     };
  \end{tikzpicture}
  }
\newcommand{\protoboardicon}{
  \begin{tikzpicture}[overlay,remember picture]
     \node[anchor=south east,xshift=-5pt,yshift=0.81\textheight]
         at (current page.south east) {
           \includegraphics[width=15mm]{resources/protoboard}
         };
     \end{tikzpicture}
  }
\newtheorem{teorema}{Teorema} %teorema en spanish

%-------------------------------------------------------------------------------
\newcommand{\insertuniversity}{Maestría en sistemas embebidos \\
Universidad de Buenos Aires \\
MSE 5Co2020 }

\newcommand{\insertauthoremail}{<slavkin.pablo@gmail.com>}

%% These macros specify information about the presentation
\title{Procesamiento de señales, fundamentos}
\author{Ing. Pablo Slavkin}

% Declare all possible audience groups
\SetNewAudience{clase1}
\SetNewAudience{clase2}
\SetNewAudience{tp}
\SetNewAudience{python}
%-------------------------------------------------------------------------------
\abovedisplayskip= 0pt
\belowdisplayskip= 0pt
\abovedisplayshortskip=0pt
\belowdisplayshortskip=7pt
