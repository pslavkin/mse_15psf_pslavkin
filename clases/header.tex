%\documentclass[draft]{beamer}
\documentclass{beamer}
\usetheme[faculty=fi]{fibeamer}
\usepackage[utf8]{inputenc}
\usepackage[ main=spanish,english,activeacute]{babel}

%%
%% These macros specify information about the presentation
\title{Procesamiento de señnales, fundamentos}
\subtitle{Clase 1 - Introducción}
\author{Ing. Pablo Slavkin}
%% These additional packages are used within the document:
\usepackage{ragged2e}  % `\justifying` text
\usepackage{booktabs}  % Tables
\usepackage{tabularx}
\usepackage{tikz}      % Diagrams
\usetikzlibrary{calc, shapes, backgrounds}
\usepackage{amsmath, amssymb}
\usepackage{url}       % `\url`s
\usepackage{listings}  % Code listings
\frenchspacing

%%\includeonlyframes{Jabberwocky,lists,simmonshall}

\newcommand{\asignature}{ PDF MSE2020 }
\newcommand{\handsonicon}{
  \begin{tikzpicture}[overlay,remember picture]
     \node[anchor=south east,xshift=-5pt,yshift=0.85\textheight]
         at (current page.south east) {
           \includegraphics[width=15mm]{resources/hands_on}
         };
     \end{tikzpicture}
  }

\newcommand{\protoboardicon}{
  \begin{tikzpicture}[overlay,remember picture]
     \node[anchor=south east,xshift=-5pt,yshift=0.81\textheight]
         at (current page.south east) {
           \includegraphics[width=15mm]{resources/protoboard}
         };
     \end{tikzpicture}
  }
\newcommand{\insertuniversity}{Maestría en sistemas embebidos MSE2020\\
Universidad de Buenos Aires}

\newtheorem{teorema}{Teorema} %teorema en spanish
