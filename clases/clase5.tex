\subtitle{Clase 5 - Applicaciones de DFT}

\begin{frame}[c]
\maketitle
\begin{tikzpicture}[overlay,remember picture]
    \node[anchor=south east,xshift=-30pt,yshift=45pt]
      at (current page.south east) {
        \includegraphics[width=0.4\textwidth]{5_clase/dft_apps}
      };
  \end{tikzpicture}
\end{frame}


%-------------------------------------------------------------------------------
\begin{frame}{Repaso Convolución}{Descomposición delta}
   \begin{columns}[c]
      \begin{column}{1\textwidth}
         \centering\includegraphics[width=0.8\textwidth]{4_clase/descomposicion_delta}
      \end{column}
   \end{columns}
   \vfill
   \note{
      \begin{itemize}
         \item{lanzar conv3}
         \item{sumo delta a delta y voy acumulando..}
         \item{la salida es N+M-1}
      \end{itemize}
   }
\end{frame}
%-------------------------------------------------------------------------------
\begin{frame}{Repaso Convolución}{Descomposición delta}
   \begin{columns}[c]
      \begin{column}{1\textwidth}
         \centering\includegraphics[width=0.8\textwidth]{4_clase/descomposicion_delta}
      \end{column}
   \end{columns}
   \vfill
   \note{
      \begin{itemize}
         \item{lanzar conv3}
         \item{sumo delta a delta y voy acumulando..}
         \item{la salida es N+M-1}
      \end{itemize}
   }
\end{frame}
%-------------------------------------------------------------------------------
\begin{frame}{Repaso Funcion delta}{Respuesta al impulso}
   \centering\includegraphics[width=0.6\textwidth]{4_clase/entrada_conv_h}\\
   \centering\includegraphics[width=0.6\textwidth]{4_clase/convolucion_eq}
   \vfill
\end{frame}
%-------------------------------------------------------------------------------
\section{Repaso Convolución}
\begin{frame}{Convolución}{Respuesta al impulso}
   \lstset{ basicstyle=\fontsize{ 5}{ 2}\selectfont\ttfamily,language=python,tabsize=4}
   \begin{columns}[c]
      \hspace{5pt}
      \begin{column}{.3\textwidth}
         \lstinputlisting[lastline=27]{4_clase/conv1.py}
      \end{column}
      \hspace{2pt}
      \vrule
      \hspace{2pt}
      \begin{column}{.3\textwidth}
         \lstinputlisting[firstline=28]{4_clase/conv1.py}
      \end{column}
      \hspace{2pt}
      \vrule
      \hspace{2pt}
      \begin{column}{.38\textwidth}
         \centering\includegraphics[width=1.0\textwidth]{4_clase/conv1}
      \end{column}
      \hspace{2pt}
   \end{columns}
   \vfill
   \note{
      \begin{itemize}
         \item{lanzar conv1}
         \item{hacer las cuentas a mano}
      \end{itemize}
   }
\end{frame}
%-------------------------------------------------------------------------------
\section{Convolicion vs Multiplicacion}
\begin{frame}{Convolución vs Multiplicacion}{Tiempo vs Frecuencia}
   \lstset{ basicstyle=\fontsize{ 5}{ 2}\selectfont\ttfamily,language=python,tabsize=4}
   \begin{columns}[c]
      \hspace{5pt}
      \begin{column}{.4\textwidth}
      \end{column}
      \hspace{2pt}
      \vrule
      \hspace{2pt}
      \begin{column}{.6\textwidth}
         \centering\includegraphics[width=0.8\textwidth]{5_clase/conv_vs_dft1}
      \end{column}
      \hspace{2pt}
   \end{columns}
   \vfill
   \note{
      \begin{itemize}
         \item{explicar el teorema de la convolucion a mano}
      \end{itemize}
   }
\end{frame}
%-------------------------------------------------------------------------------
\begin{frame}{Bibliografía}
   \framesubtitle{Libros, links y otro material}
   \begin{thebibliography}{9}
         \bibitem{cmsisdsp}
         \emph{ARM CMSIS DSP}. \\
         \href {https://arm-software.github.io/CMSIS_5/DSP/html/index.html}{https://arm-software.github.io/CMSIS\_5/DSP/html/index.html}
         \bibitem{dsp}
         Steven W. Smith.
         \emph{The Scientist and Engineer's Guide to Digital Signal Processing}.
         Second Edition, 1999.
         \bibitem{grants Anderson}
         \emph{Grant Sanderson} \\
         \href{ https://youtu.be/spUNpyF58BY}{ https://youtu.be/spUNpyF58BY}
         \bibitem{interactivemath}
         \emph{Interactive Mathematics Site Info}. \\
         \href {https://www.intmath.com/fourier-series/fourier-intro.php}{https://www.intmath.com/fourier-series/fourier-intro.php}
   \end{thebibliography}
\end{frame}
%-------------------------------------------------------------------------------

