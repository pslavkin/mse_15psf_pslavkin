\subtitle{Clase 5 - Applicaciones de DFT}

\begin{frame}[c]
\maketitle
\begin{tikzpicture}[overlay,remember picture]
    \node[anchor=south east,xshift=-30pt,yshift=45pt]
      at (current page.south east) {
        \includegraphics[width=0.4\textwidth]{5_clase/dft_apps}
      };
  \end{tikzpicture}
   \note{
      \begin{itemize}
         \item{arrancar repaso de numeros Q con nueva clase 2}
         \item{arrancar comentando el repaso de convolucion con otro enfoque}
         \item{Recordar el tema de la encuesta}
         \item{Dejar espacio al final de la clase para ver numeros Q}
      \end{itemize}
   }
\end{frame}
%-------------------------------------------------------------------------------
 \begin{frame}{SAPI}
    \framesubtitle{Se aceptan pull request para la SAPI}
 \LARGE 
   SAPI DSP
\end{frame}
%%-------------------------------------------------------------------------------
 \begin{frame}{Enuestas}
    \framesubtitle{Encuesta anónima clase a clase}
    Propiciamos este espacio para compartir sus sugerencias, criticas constructivas, oportunidades de mejora y cualquier tipo de comentario relacionado a la clase.
    \begin{block}{Encuesta anónima}{
       \includegraphics[width=0.1\textwidth]{1_clase/click}
       \href{https://forms.gle/1j5dDTQ7qjVfRwYo8}{https://forms.gle/1j5dDTQ7qjVfRwYo8}
    }
       \end{block}
    \begin{block}{Link al material de la material}{
       \includegraphics[width=0.1\textwidth]{1_clase/click}
       \tiny{\href{https://drive.google.com/drive/u/1/folders/1TlR2cgDPchL\_4v7DxdpS7pZHtjKq38CK}{https://drive.google.com/drive/u/1/folders/1TlR2cgDPchL\_4v7DxdpS7pZHtjKq38CK}
    }
    }
       \end{block}
\end{frame}
%-------------------------------------------------------------------------------
\begin{frame}{Repaso Convolución}{Multiplicacion?!}
   Algoritmo de Multiplicacion de 2do grado
   \begin{align*}
      1\hspace{1cm} 2\\
      3\hspace{1cm} 4 \\
      \cline{1-2}
      4 \hspace{1cm} 8 \\
      3 \hspace{1cm} 6 \hspace{1cm}0 \\
      \cline{1-2}
      3 \hspace{1cm} 10 \hspace{1cm} 8\\
   \end{align*}
   \vfill
   \note{
      \begin{itemize}
         \item{no hay que lanzar nada}
         \item{explicar 3 manera de multiplicar un numero}
         \item{darle forma de respuesta al impulso y senial}
      \end{itemize}
   }
\end{frame}
%-------------------------------------------------------------------------------
\begin{frame}{Repaso Convolución}{Descomposición delta}
   \begin{columns}[c]
      \begin{column}{0.4\textwidth}
      \tiny
      SUma deltas desplazadas
            \begin{align*}
               \textcolor{cyan} {1} \hspace{1cm}\textcolor{cyan}{2} \hspace{1cm}0\\
               \textcolor{green}{3} \hspace{1cm}                 0  \hspace{1cm}0\\
               \cline{1-2}
               3\hspace{1cm} 6 \hspace{1cm}0\\[0.4cm]
               0\hspace{1cm}\textcolor{cyan} {1} \hspace{1cm}\textcolor{cyan}{2}\\
               0\hspace{1cm}\textcolor{green}{4} \hspace{1cm}                 0\\
               \cline{1-2}
               3\hspace{1cm} 6 \hspace{1cm}0\\
               0\hspace{1cm} 4 \hspace{1cm}8\\
               \cline{1-2}
               3 \hspace{1cm}10 \hspace{1cm} 8\\
            \end{align*}
      \end{column}
      \hspace{2pt}
      \vrule
      \hspace{2pt}
      \begin{column}{0.6\textwidth}
         \centering\includegraphics[width=0.9\textwidth]{5_clase/conv_as_multiply1}
      \end{column}
   \end{columns}
   \vfill
   \note{
      \begin{itemize}
      \item{lanzar conv\_as\_multiply1}
         \item{muestro la misma cuenta con señales}
      \end{itemize}
   }
\end{frame}
%-------------------------------------------------------------------------------
\begin{frame}{Repaso Convolución}{Convolucion formal}
   \begin{columns}[c]
      \begin{column}{0.4\textwidth}
      \tiny
      \tiny
      Convolucion
            \begin{align*}
                                2 \hspace{1cm} \textcolor{green}{1}\hspace{1cm}0 \hspace{1cm}0\\
               \textcolor{cyan}{0}\hspace{1cm} \textcolor{cyan} {3}\hspace{1cm}4 \hspace{1cm}0\\
               \cline{1-2}
               3\hspace{1cm} 0\hspace{1cm}0\\[0.4cm]
               0\hspace{1cm}                  2 \hspace{1cm}\textcolor{green}{1} \hspace{1cm}0\\
               0\hspace{1cm} \textcolor{cyan}{3}\hspace{1cm}\textcolor{cyan} {4} \hspace{1cm}0\\
               \cline{1-2}
               3\hspace{1cm} 10\hspace{1cm}0\\[0.4cm]
               0\hspace{1cm} 0\hspace{1cm}                 2 \hspace{1cm}\textcolor{green}{1}\\
               0\hspace{1cm} 3\hspace{1cm}\textcolor{cyan}{4}\hspace{1cm}\textcolor{cyan} {0}\\
               \cline{1-2}
               3 \hspace{1cm} 10 \hspace{1cm} 8\\
            \end{align*}
      \end{column}
      \hspace{2pt}
      \vrule
      \hspace{2pt}
      \begin{column}{0.6\textwidth}
         \centering\includegraphics[width=0.9\textwidth]{5_clase/conv_as_multiply2}
      \end{column}
   \end{columns}
   \vfill
   \note{
      \begin{itemize}
      \item{lanzar conv\_as\_multiply2}
         \item{muestro la misma cuenta con señales}
      \end{itemize}
   }
\end{frame}
%-------------------------------------------------------------------------------
\begin{frame}{Repaso Convolución}{Convolucion como producto de polinomios}
   \begin{align*}
   \large
      \left( 1 x10^1 + 2 x10^0 \right) * \left( 3 x10^1 + 4 x10^0 \right) &=  \\
      \left( 3 x10^2 + 4 x10^1 + 6 x10^1 + 8 x10^0\right) &=  \\
      \left( 3 x10^2 + 10 x10^1 + 8 x10^0\right) &=  \\
      \left( 300 + 100 + 8\right) &= 408
   \end{align*}
   \vfill
   \note{
      \begin{itemize}
         \item{comentar que tambien se puede ver como multiplicacion de polinomios}
         \item{en el caso de la convolucion, no se trata de 10\^x sino que queda expresado en ese orden cada termino}
      \end{itemize}
   }
\end{frame}
%-------------------------------------------------------------------------------
\begin{frame}{Repaso Convolución}{Multiplicacion?!}
   \begin{columns}[t]
      \hspace{5pt}
   \begin{column}{.3\textwidth}
      \tiny
      Algoritmo de Multiplicacion
          \begin{align*}
               1\hspace{1cm} 2\\
               3\hspace{1cm} 4 \\
               \cline{1-2}
               4 \hspace{1cm} 8 \\
               3 \hspace{1cm} 6 \hspace{1cm}0 \\
               \cline{1-2}
               3 \hspace{1cm} 10 \hspace{1cm} 8\\
            \end{align*}
      Multiplicacion de polinomios
            \begin{align*}
               \left( 1 x10^1 + 2 x10^0 \right) * \left( 3 x10^1 + 4 x10^0 \right) &=  \\
               \left( 3 x10^2 + 4 x10^1 + 6 x10^1 + 8 x10^0\right) &=  \\
               \left( 3 x10^2 + 10 x10^1 + 8 x10^0\right) &=  \\
               \left( 300 + 100 + 8\right) &= 408
            \end{align*}
      \end{column}
      \hspace{2pt}
      \vrule
      \hspace{2pt}
      \begin{column}{.3\textwidth}
      \tiny
      SUma deltas desplazadas
            \begin{align*}
               \textcolor{cyan} {1} \hspace{1cm}\textcolor{cyan}{2} \hspace{1cm}0\\
               \textcolor{green}{3} \hspace{1cm}                 0  \hspace{1cm}0\\
               \cline{1-2}
               3\hspace{1cm} 6 \hspace{1cm}0\\[0.4cm]
               0\hspace{1cm}\textcolor{cyan} {1} \hspace{1cm}\textcolor{cyan}{2}\\
               0\hspace{1cm}\textcolor{green}{4} \hspace{1cm}                 0\\
               \cline{1-2}
               3\hspace{1cm} 6 \hspace{1cm}0\\
               0\hspace{1cm} 4 \hspace{1cm}8\\
               \cline{1-2}
               3 \hspace{1cm}10 \hspace{1cm} 8\\
            \end{align*}
      \end{column}
      \hspace{2pt}
      \vrule
      \hspace{2pt}
      \begin{column}{.3\textwidth}
      \tiny
      Convolucion
            \begin{align*}
                                2 \hspace{1cm} \textcolor{green}{1}\hspace{1cm}0 \hspace{1cm}0\\
               \textcolor{cyan}{0}\hspace{1cm} \textcolor{cyan} {3}\hspace{1cm}4 \hspace{1cm}0\\
               \cline{1-2}
               3\hspace{1cm} 0\hspace{1cm}0\\[0.4cm]
               0\hspace{1cm}                  2 \hspace{1cm}\textcolor{green}{1} \hspace{1cm}0\\
               0\hspace{1cm} \textcolor{cyan}{3}\hspace{1cm}\textcolor{cyan} {4} \hspace{1cm}0\\
               \cline{1-2}
               3\hspace{1cm} 10\hspace{1cm}0\\[0.4cm]
               0\hspace{1cm} 0\hspace{1cm}                 2 \hspace{1cm}\textcolor{green}{1}\\
               0\hspace{1cm} 3\hspace{1cm}\textcolor{cyan}{4}\hspace{1cm}\textcolor{cyan} {0}\\
               \cline{1-2}
               3 \hspace{1cm} 10 \hspace{1cm} 8\\
            \end{align*}
      \end{column}
      \hspace{2pt}
   \end{columns}
   \vfill
   \note{
      \begin{itemize}
         \item{no hay que lanzar nada}
         \item{explicar 3 manera de multiplicar un numero}
         \item{darle forma de respuesta al impulso y senial}
      \end{itemize}
   }
\end{frame}
%-------------------------------------------------------------------------------
\begin{frame}{Convolucion en tiempo}{filtrado}
   \lstset{ basicstyle=\fontsize{ 5}{ 1}\selectfont\ttfamily,language=python,tabsize=4}
   \begin{columns}[c]
      \begin{column}{.5\textwidth}
         \centering\includegraphics[width=1.0\textwidth]{4_clase/conv1}
      \end{column}
      \hspace{2pt}
      \vrule
      \hspace{2pt}
      \begin{column}{.5\textwidth}
         \centering\includegraphics[width=1.0\textwidth]{4_clase/conv2}
      \end{column}
      \hspace{2pt}
   \end{columns}
   \vfill
   \note{
      \begin{itemize}
         \item{lanzar conv1 y luego conv2}
         \item{comentar lo que ya sabemos hacer con la convolucion en tiempo}
         \item{prepara la idea para F}
      \end{itemize}
   }
\end{frame}
%-------------------------------------------------------------------------------
\begin{frame}{Repaso Convolucion}{Propiedades}
   \begin{columns}[c]
      \hspace{5pt}
      \begin{column}{.2\textwidth}
         \begin{itemize}
            \item{Conmutativa}
            \item{Distributiva}
            \item{Asociativa}
         \end{itemize}
      \end{column}
      \hspace{2pt}
      \vrule
      \hspace{2pt}
      \begin{column}{.8\textwidth}
         \centering\includegraphics[width=0.7\textwidth]{4_clase/entrada_conv_h}\\
         \centering\includegraphics[width=0.7\textwidth]{4_clase/convolucion_eq}
      \end{column}
      \hspace{2pt}
   \end{columns}
   \vfill
\end{frame}
%-------------------------------------------------------------------------------
\begin{frame}{Repaso Multiplicacion}{Propiedad conmutativa}
   \begin{columns}[c]
      \hspace{5pt}
      \begin{column}{0.5\textwidth}
         \centering\includegraphics[width=0.8\textwidth]{5_clase/multi_conmutativa}
      \end{column}
      \hspace{2pt}
      \vrule
      \hspace{2pt}
      \begin{column}{0.5\textwidth}
         \centering\includegraphics[width=0.9\textwidth]{5_clase/conv_conmutativa}
      \end{column}
      \hspace{2pt}
   \end{columns}
   \vfill
\end{frame}
%-------------------------------------------------------------------------------
\begin{frame}{Repaso Multiplicacion}{Propiedad asociativa}
   \begin{columns}[c]
      \hspace{5pt}
      \begin{column}{0.5\textwidth}
         \centering\includegraphics[width=0.8\textwidth]{5_clase/multi_asociativa}
      \end{column}
      \hspace{2pt}
      \vrule
      \hspace{2pt}
      \begin{column}{0.5\textwidth}
         \centering\includegraphics[width=0.9\textwidth]{5_clase/conv_asociativa}
      \end{column}
      \hspace{2pt}
   \end{columns}
   \vfill
\end{frame}
%-------------------------------------------------------------------------------
\begin{frame}{Repaso Multiplicacion}{Propiedad distributiva}
   \begin{columns}[c]
      \hspace{5pt}
      \begin{column}{0.5\textwidth}
         \centering\includegraphics[width=0.8\textwidth]{5_clase/multi_distributiva}
      \end{column}
      \hspace{2pt}
      \vrule
      \hspace{2pt}
      \begin{column}{0.5\textwidth}
         \centering\includegraphics[width=0.9\textwidth]{5_clase/conv_distributiva}
      \end{column}
      \hspace{2pt}
   \end{columns}
   \vfill
\end{frame}
%-------------------------------------------------------------------------------
\section{Convolución vs Multiplicación}
\begin{frame}[t]{Convolución vs Multiplicación}{Teorema de la convolución}
   \center\includegraphics[width=0.8\textwidth]{5_clase/dft_apps}\\
   \vfill
   \note{
      \begin{itemize}
         \item{explicar que cuando la entrada en t es la delta estamos simulando barrido en f, porque la delta tiene todas las f}
         \item{luego en F la entrada son círculos rotando multiplicados por círculos rotando, amplitud se escala y fase(exponente) se suman, se corre la fase}
         \item{no hacemos la demostración del teorema, sino que lo probamos prácticamente}
         \item{explicar la conclusion y el teorema de la convolución}
      \end{itemize}
   }
\end{frame}
%-------------------------------------------------------------------------------
\begin{frame}{Multiplicación con DFT}{Tiempo vs Frecuencia}
   \centering\includegraphics[width=0.8\textwidth]{5_clase/conv_vs_dft1}
   \vfill
   \note{
      \begin{itemize}
         \item{lanzar conv\_vs\_dft1}
         \item{explicar multiplicacion usando DFT}
         \item{hacer notar que hay que estirar las cosas para que la salida tenga N+M-1}
         \item{explicar multiplicación usando DFT}
      \end{itemize}
   }
\end{frame}
%-------------------------------------------------------------------------------
\begin{frame}[t]{Convolución vs Multiplicación}{Teorema de la convolución}
   \center\includegraphics[width=0.4\textwidth]{5_clase/teorema_conv1}
   \vfill
   \note{
      \begin{itemize}
         \item{explicar la conclusion y el teorema de la convolución}
         \item{explicar que dado h r y(t) podemos dividir en frec y obtener x()}
      \end{itemize}
   }
\end{frame}
%-------------------------------------------------------------------------------
\begin{frame}[t]{Convolución vs Multiplicación}{Convolución circular}
   \center\includegraphics[width=0.4\textwidth]{5_clase/teorema_conv2}
   \vfill
   \note{
      \begin{itemize}
         \item{explicar el efecto de la convolución circular}
      \end{itemize}
   }
\end{frame}
%-------------------------------------------------------------------------------
\begin{frame}[t]{Convolución vs Multiplicación}{Teorema de la convolución}
   \center\includegraphics[width=0.8\textwidth]{5_clase/teorema_conv_eq}
   \vfill
   \note{
      \begin{itemize}
         \item{explicar el tema de la DTFT y la transformada circular}
      \end{itemize}
   }
\end{frame}
%-------------------------------------------------------------------------------
\begin{frame}[t]{Filtrado}{Pasabajos}
   \center\includegraphics[width=0.8\textwidth]{5_clase/filtrado1}
   \vfill
   \note{
      \begin{itemize}
         \item{explicar ahora el uso de la convolution en el filtrado}
         \item{a partir de 64 puntos de fir conviene FFT, por menos conviene convolution en tiempo}
      \end{itemize}
   }
\end{frame}
%-------------------------------------------------------------------------------
\begin{frame}[t]{Filtrado}{Pasaaltos}
   \center\includegraphics[width=0.8\textwidth]{5_clase/filtrado2}
   \vfill
   \note{
      \begin{itemize}
         \item{explicar ahora el uso de la convolution en el filtrado}
         \item{a partir de 64 puntos de fir conviene FFT, por menos conviene convolution en tiempo}
      \end{itemize}
   }
\end{frame}
%-------------------------------------------------------------------------------
\begin{frame}[t]{Filtrado}{Definición}
   \center\includegraphics[width=0.6\textwidth]{5_clase/pyfda3}
   \vfill
   \note{
      \begin{itemize}
         \item{explicar las zonas de los filtros, tipos de filtro}
         \item{relación de compromiso entre ripple y bandas, etc}
      \end{itemize}
   }
\end{frame}
%-------------------------------------------------------------------------------
\begin{frame}[t]{Filtrado}{PyFDA \href{/opt/anaconda3/bin/pyfdax}{/opt/anaconda3/bin/pyfdax}}
   \center\includegraphics[width=0.75\textwidth]{5_clase/pyfda1}
   \note{
      \begin{itemize}
         \item{explicar ahora el uso de la convolution en el filtrado}
         \item{a partir de 64 puntos de fir conviene FFT, por menos conviene convolution en tiempo}
      \end{itemize}
   }
\end{frame}
%-------------------------------------------------------------------------------
\begin{frame}[t]{Filtrado}{Pyfda \href{/opt/anaconda3/bin/pyfdax}{/opt/anaconda3/bin/pyfdax}}
   \center\includegraphics[width=0.75\textwidth]{5_clase/pyfda2}
   \vfill
   \note{
      \begin{itemize}
         \item{explicar ahora el uso de la convolution en el filtrado}
         \item{a partir de 64 puntos de fir conviene FFT, por menos conviene convolucino en tiempo}
      \end{itemize}
   }
\end{frame}
%-------------------------------------------------------------------------------
\begin{frame}[t]{Convolución}{Superponer y sumar}
   \center\includegraphics[width=0.8\textwidth]{5_clase/overlap_add1}
   \vfill
   \note{
      \begin{itemize}
         \item{explicar el detalle de overlap para sumar}
      \end{itemize}
   }
\end{frame}
%-------------------------------------------------------------------------------
\begin{frame}[t]{Convolución con FFT}{Superponer y sumar}
   \center\includegraphics[width=0.8\textwidth]{5_clase/overlap_add2}
   \vfill
   \note{
      \begin{itemize}
         \item{explicar el detalle de overlap para sumar}
      \end{itemize}
   }
\end{frame}
%-------------------------------------------------------------------------------
\section{CIAA}
\begin{frame}[t]{Filtrado con CIAA}{Conversor PyFDA a fir.h para C}
   \handsonicon
   Código en Python para convertir los coeficientes del fir extendidos en Q1.15 en C
   \lstset{ basicstyle=\fontsize{ 4.5}{ 2}\selectfont\ttfamily,language=python,tabsize=4}
   \begin{columns}[c]
      \hspace{2pt}
      \begin{column}{.3\textwidth}
         \lstinputlisting[lastline=27]{5_clase/fir_to_c.py}
      \end{column}
      \hspace{2pt}
      \vrule
      \hspace{2pt}
      \begin{column}{.3\textwidth}
         \lstinputlisting[firstline=28]{5_clase/fir_to_c.py}
      \end{column}
      \hspace{2pt}
      \vrule
      \hspace{2pt}
      \begin{column}{.4\textwidth}
         \includegraphics[width=0.8\textwidth]{5_clase/fir_to_c.png}
      \end{column}
      \hspace{2pt}
   \end{columns}
   \vfill
   \note{
      \begin{itemize}
         \item{mostrar como pasar de pyfda a C}
         \item{lanzar psf1}
         \item{probar distintos filtros y ver resultado }
         \item{hacer notar el efecto del padding}
      \end{itemize}
   }
\end{frame}
%-------------------------------------------------------------------------------
\begin{frame}[t]{Filtrado con CIAA}{Con padding y convolución}
   \protoboardicon
   Convolución en tiempo con padding en CIAA para filtrado
   \lstset{ basicstyle=\fontsize{ 4}{ 1}\selectfont\ttfamily,language=c,tabsize=4}
   \begin{columns}[c]
      \hspace{5pt}
      \begin{column}{.3\textwidth}
         \lstinputlisting[lastline=32]{5_clase/ciaa/psf1/src/psf.c}
      \end{column}
      \hspace{2pt}
      \vrule
      \hspace{2pt}
      \begin{column}{.3\textwidth}
         \lstinputlisting[firstline=34]{5_clase/ciaa/psf1/src/psf.c}
      \end{column}
      \hspace{2pt}
      \vrule
      \hspace{2pt}
      \begin{column}{.4\textwidth}
         \includegraphics[width=0.8\textwidth]{5_clase/ciaa/psf1/conv1.png}
      \end{column}
      \hspace{2pt}
   \end{columns}
   \vfill
   \note{
      \begin{itemize}
         \item{mostrar como pasar de pyfda a C}
         \item{lanzar psf1}
         \item{probar distintos filtros y ver resultado }
         \item{hacer notar el efecto del padding}
      \end{itemize}
   }
\end{frame}
%-------------------------------------------------------------------------------
\begin{frame}[t]{Filtrado con CIAA}{Con padding y FFT}
   \protoboardicon
   Convolución en tiempo con padding en CIAA para filtrado
   \lstset{ basicstyle=\fontsize{ 4}{ 1}\selectfont\ttfamily,language=c,tabsize=4}
   \begin{columns}[c]
      \hspace{2pt}
      \begin{column}{.3\textwidth}
         \lstinputlisting[lastline=35]{5_clase/ciaa/psf2/src/psf.c}
      \end{column}
      \hspace{2pt}
      \vrule
      \hspace{2pt}
      \begin{column}{.3\textwidth}
         \lstinputlisting[firstline=36]{5_clase/ciaa/psf2/src/psf.c}
      \end{column}
      \hspace{2pt}
      \vrule
      \hspace{2pt}
      \begin{column}{.4\textwidth}
         \includegraphics[width=0.9\textwidth]{5_clase/ciaa/psf2/conv1.png}
      \end{column}
      \hspace{2pt}
   \end{columns}
   \vfill
   \note{
      \begin{itemize}
         \item{mostrar como pasar de pyfda a C}
         \item{lanzar psf1}
         \item{probar distintos filtros y ver resultado }
         \item{hacer notar el efecto del padding}
      \end{itemize}
   }
\end{frame}
%-------------------------------------------------------------------------------
\begin{frame}{Bibliografía}
   \framesubtitle{Libros, links y otro material}
   \begin{thebibliography}{9}
         \bibitem{cmsisdsp}
         \emph{ARM CMSIS DSP}. \\
         \href {https://arm-software.github.io/CMSIS_5/DSP/html/index.html}{https://arm-software.github.io/CMSIS\_5/DSP/html/index.html}
         \bibitem{dsp}
         Steven W. Smith.
         \emph{The Scientist and Engineer's Guide to Digital Signal Processing}.
         Second Edition, 1999.
         \bibitem{Teorema de la convolucion}
         \emph{Wikipedia}. \\
         \href {https://en.wikipedia.org/wiki/Convolution\_theorem}{https://en.wikipedia.org/wiki/Convolution\_theorem}
   \end{thebibliography}
\end{frame}
%-------------------------------------------------------------------------------

