\subtitle{Clase 5 - Applicaciones de DFT}

\begin{frame}[c]
\maketitle
\begin{tikzpicture}[overlay,remember picture]
    \node[anchor=south east,xshift=-30pt,yshift=45pt]
      at (current page.south east) {
        \includegraphics[width=0.4\textwidth]{5_clase/dft_apps}
      };
  \end{tikzpicture}
\end{frame}


%-------------------------------------------------------------------------------
\begin{frame}{Repaso Convolución}{Multiplicacion?!}
   Algoritmo de Multiplicacion de 2do grado
   \begin{align*}
      1\hspace{1cm} 2\\
      3\hspace{1cm} 4 \\
      \cline{1-2}
      4 \hspace{1cm} 8 \\
      3 \hspace{1cm} 6 \hspace{1cm}0 \\
      \cline{1-2}
      3 \hspace{1cm} 10 \hspace{1cm} 8\\
   \end{align*}
   \vfill
   \note{
      \begin{itemize}
         \item{no hay que lanzar nada}
         \item{explicar 3 manera de multiplicar un numero}
         \item{darle forma de respuesta al impulso y senial}
      \end{itemize}
   }
\end{frame}
%-------------------------------------------------------------------------------
\begin{frame}{Repaso Convolución}{Descomposición delta}
   \begin{columns}[c]
      \begin{column}{0.4\textwidth}
      \tiny
      SUma deltas desplazadas
            \begin{align*}
               \textcolor{cyan} {1} \hspace{1cm}\textcolor{cyan}{2} \hspace{1cm}0\\
               \textcolor{green}{3} \hspace{1cm}                 0  \hspace{1cm}0\\
               \cline{1-2}
               3\hspace{1cm} 6 \hspace{1cm}0\\[0.4cm]
               0\hspace{1cm}\textcolor{cyan} {1} \hspace{1cm}\textcolor{cyan}{2}\\
               0\hspace{1cm}\textcolor{green}{4} \hspace{1cm}                 0\\
               \cline{1-2}
               3\hspace{1cm} 6 \hspace{1cm}0\\
               0\hspace{1cm} 4 \hspace{1cm}8\\
               \cline{1-2}
               3 \hspace{1cm}10 \hspace{1cm} 8\\
            \end{align*}
      \end{column}
      \hspace{2pt}
      \vrule
      \hspace{2pt}
      \begin{column}{0.6\textwidth}
         \centering\includegraphics[width=0.9\textwidth]{5_clase/conv_as_multiply1}
      \end{column}
   \end{columns}
   \vfill
   \note{
      \begin{itemize}
      \item{lanzar conv\_as\_multiply1}
         \item{muestro la misma cuenta con señales}
      \end{itemize}
   }
\end{frame}
%-------------------------------------------------------------------------------
\begin{frame}{Repaso Convolución}{Convolucion formal}
   \begin{columns}[c]
      \begin{column}{0.4\textwidth}
      \tiny
      \tiny
      Convolucion
            \begin{align*}
                                2 \hspace{1cm} \textcolor{green}{1}\hspace{1cm}0 \hspace{1cm}0\\
               \textcolor{cyan}{0}\hspace{1cm} \textcolor{cyan} {3}\hspace{1cm}4 \hspace{1cm}0\\
               \cline{1-2}
               3\hspace{1cm} 0\hspace{1cm}0\\[0.4cm]
               0\hspace{1cm}                  2 \hspace{1cm}\textcolor{green}{1} \hspace{1cm}0\\
               0\hspace{1cm} \textcolor{cyan}{3}\hspace{1cm}\textcolor{cyan} {4} \hspace{1cm}0\\
               \cline{1-2}
               3\hspace{1cm} 10\hspace{1cm}0\\[0.4cm]
               0\hspace{1cm} 0\hspace{1cm}                 2 \hspace{1cm}\textcolor{green}{1}\\
               0\hspace{1cm} 3\hspace{1cm}\textcolor{cyan}{4}\hspace{1cm}\textcolor{cyan} {0}\\
               \cline{1-2}
               3 \hspace{1cm} 10 \hspace{1cm} 8\\
            \end{align*}
      \end{column}
      \hspace{2pt}
      \vrule
      \hspace{2pt}
      \begin{column}{0.6\textwidth}
         \centering\includegraphics[width=0.9\textwidth]{5_clase/conv_as_multiply2}
      \end{column}
   \end{columns}
   \vfill
   \note{
      \begin{itemize}
      \item{lanzar conv\_as\_multiply2}
         \item{muestro la misma cuenta con señales}
      \end{itemize}
   }
\end{frame}
%-------------------------------------------------------------------------------
\begin{frame}{Repaso Convolución}{Multiplicacion?!}
   \begin{columns}[t]
      \hspace{5pt}
   \begin{column}{.3\textwidth}
      \tiny
      Algoritmo de Multiplicacion
          \begin{align*}
               1\hspace{1cm} 2\\
               3\hspace{1cm} 4 \\
               \cline{1-2}
               4 \hspace{1cm} 8 \\
               3 \hspace{1cm} 6 \hspace{1cm}0 \\
               \cline{1-2}
               3 \hspace{1cm} 10 \hspace{1cm} 8\\
            \end{align*}
      \end{column}
      \hspace{2pt}
      \vrule
      \hspace{2pt}
      \begin{column}{.3\textwidth}
      \tiny
      SUma deltas desplazadas
            \begin{align*}
               \textcolor{cyan} {1} \hspace{1cm}\textcolor{cyan}{2} \hspace{1cm}0\\
               \textcolor{green}{3} \hspace{1cm}                 0  \hspace{1cm}0\\
               \cline{1-2}
               3\hspace{1cm} 6 \hspace{1cm}0\\[0.4cm]
               0\hspace{1cm}\textcolor{cyan} {1} \hspace{1cm}\textcolor{cyan}{2}\\
               0\hspace{1cm}\textcolor{green}{4} \hspace{1cm}                 0\\
               \cline{1-2}
               3\hspace{1cm} 6 \hspace{1cm}0\\
               0\hspace{1cm} 4 \hspace{1cm}8\\
               \cline{1-2}
               3 \hspace{1cm}10 \hspace{1cm} 8\\
            \end{align*}
      \end{column}
      \hspace{2pt}
      \vrule
      \hspace{2pt}
      \begin{column}{.3\textwidth}
      \tiny
      Convolucion
            \begin{align*}
                                2 \hspace{1cm} \textcolor{green}{1}\hspace{1cm}0 \hspace{1cm}0\\
               \textcolor{cyan}{0}\hspace{1cm} \textcolor{cyan} {3}\hspace{1cm}4 \hspace{1cm}0\\
               \cline{1-2}
               3\hspace{1cm} 0\hspace{1cm}0\\[0.4cm]
               0\hspace{1cm}                  2 \hspace{1cm}\textcolor{green}{1} \hspace{1cm}0\\
               0\hspace{1cm} \textcolor{cyan}{3}\hspace{1cm}\textcolor{cyan} {4} \hspace{1cm}0\\
               \cline{1-2}
               3\hspace{1cm} 10\hspace{1cm}0\\[0.4cm]
               0\hspace{1cm} 0\hspace{1cm}                 2 \hspace{1cm}\textcolor{green}{1}\\
               0\hspace{1cm} 3\hspace{1cm}\textcolor{cyan}{4}\hspace{1cm}\textcolor{cyan} {0}\\
               \cline{1-2}
               3 \hspace{1cm} 10 \hspace{1cm} 8\\
            \end{align*}
      \end{column}
      \hspace{2pt}
   \end{columns}
   \vfill
   \note{
      \begin{itemize}
         \item{no hay que lanzar nada}
         \item{explicar 3 manera de multiplicar un numero}
         \item{darle forma de respuesta al impulso y senial}
      \end{itemize}
   }
\end{frame}
%-------------------------------------------------------------------------------
\begin{frame}{Repaso Convolucion}{Propiedades}
   \begin{columns}[c]
      \hspace{5pt}
      \begin{column}{.2\textwidth}
         \begin{itemize}
            \item{Conmutativa}
            \item{Distributiva}
            \item{Asociativa}
         \end{itemize}
      \end{column}
      \hspace{2pt}
      \vrule
      \hspace{2pt}
      \begin{column}{.8\textwidth}
         \centering\includegraphics[width=0.7\textwidth]{4_clase/entrada_conv_h}\\
         \centering\includegraphics[width=0.7\textwidth]{4_clase/convolucion_eq}
      \end{column}
      \hspace{2pt}
   \end{columns}
   \vfill
\end{frame}
%-------------------------------------------------------------------------------
\begin{frame}{Repaso Multiplicacion}{Propiedad conmutativa}
   \begin{columns}[c]
      \hspace{5pt}
      \begin{column}{0.5\textwidth}
         \centering\includegraphics[width=0.8\textwidth]{5_clase/multi_conmutativa}
      \end{column}
      \hspace{2pt}
      \vrule
      \hspace{2pt}
      \begin{column}{0.5\textwidth}
         \centering\includegraphics[width=0.9\textwidth]{5_clase/conv_conmutativa}
      \end{column}
      \hspace{2pt}
   \end{columns}
   \vfill
\end{frame}
%-------------------------------------------------------------------------------
\begin{frame}{Repaso Multiplicacion}{Propiedad asociativa}
   \begin{columns}[c]
      \hspace{5pt}
      \begin{column}{0.5\textwidth}
         \centering\includegraphics[width=0.8\textwidth]{5_clase/multi_asociativa}
      \end{column}
      \hspace{2pt}
      \vrule
      \hspace{2pt}
      \begin{column}{0.5\textwidth}
         \centering\includegraphics[width=0.9\textwidth]{5_clase/conv_asociativa}
      \end{column}
      \hspace{2pt}
   \end{columns}
   \vfill
\end{frame}
%-------------------------------------------------------------------------------
\begin{frame}{Repaso Multiplicacion}{Propiedad distributiva}
   \begin{columns}[c]
      \hspace{5pt}
      \begin{column}{0.5\textwidth}
         \centering\includegraphics[width=0.8\textwidth]{5_clase/multi_distributiva}
      \end{column}
      \hspace{2pt}
      \vrule
      \hspace{2pt}
      \begin{column}{0.5\textwidth}
         \centering\includegraphics[width=0.9\textwidth]{5_clase/conv_distributiva}
      \end{column}
      \hspace{2pt}
   \end{columns}
   \vfill
\end{frame}
%-------------------------------------------------------------------------------
\section{Convolucion vs Multiplicacion}
\begin{frame}[t]{Convolución vs Multiplicacion}{Teorema de la convolucion}
   \center\includegraphics[width=0.72\textwidth]{5_clase/dft_apps}
   \vfill
   \note{
      \begin{itemize}
         \item{explicar la conclusion y el teorema de la convolucion}
      \end{itemize}
   }
\end{frame}
%-------------------------------------------------------------------------------
\begin{frame}{Multiplicacion con DFT}{Tiempo vs Frecuencia}
   \centering\includegraphics[width=0.8\textwidth]{5_clase/conv_vs_dft1}
   \vfill
   \note{
      \begin{itemize}
         \item{explicar multiplicacion usando DFT}
      \end{itemize}
   }
\end{frame}
%-------------------------------------------------------------------------------
\begin{frame}[t]{Convolución vs Multiplicacion}{Teorema de la convolucion}
   \center\includegraphics[width=0.4\textwidth]{5_clase/teorema_conv1}
   \vfill
   \note{
      \begin{itemize}
         \item{explicar la conclusion y el teorema de la convolucion}
         \item{explicar que dado h r y(t) podemos dividir en frec y obtener x()}
      \end{itemize}
   }
\end{frame}
%-------------------------------------------------------------------------------
\begin{frame}[t]{Convolución vs Multiplicacion}{Convolucion circular}
   \center\includegraphics[width=0.4\textwidth]{5_clase/teorema_conv2}
   \vfill
   \note{
      \begin{itemize}
         \item{explicar el efecto de la convolucion circular}
      \end{itemize}
   }
\end{frame}
%-------------------------------------------------------------------------------
\begin{frame}[t]{Filtrado}{Pasabajos}
   \center\includegraphics[width=0.8\textwidth]{5_clase/filtrado1}
   \vfill
   \note{
      \begin{itemize}
         \item{explicar ahora el uso de la convolucino en el filtrado}
         \item{a partir de 64 puntos de fir conviene FFT, por menos conviene convolucino en tiempo}
      \end{itemize}
   }
\end{frame}
%-------------------------------------------------------------------------------
\begin{frame}[t]{Filtrado}{Pasaaltos}
   \center\includegraphics[width=0.8\textwidth]{5_clase/filtrado2}
   \vfill
   \note{
      \begin{itemize}
         \item{explicar ahora el uso de la convolucino en el filtrado}
         \item{a partir de 64 puntos de fir conviene FFT, por menos conviene convolucino en tiempo}
      \end{itemize}
   }
\end{frame}
%-------------------------------------------------------------------------------
\begin{frame}[t]{Filtrado}{Definicion}
   \center\includegraphics[width=0.6\textwidth]{5_clase/pyfda3}
   \vfill
   \note{
      \begin{itemize}
         \item{explicar las zonas de los filtros, tipos de filtro}
         \item{relacion de compromiso entre ripple y bandas, etc}
      \end{itemize}
   }
\end{frame}
%-------------------------------------------------------------------------------
\begin{frame}[t]{Filtrado}{Pyfda \href{/opt/anaconda3/bin/pyfdax}{/opt/anaconda3/bin/pyfdax}}
   \center\includegraphics[width=0.75\textwidth]{5_clase/pyfda1}
   \note{
      \begin{itemize}
         \item{explicar ahora el uso de la convolucino en el filtrado}
         \item{a partir de 64 puntos de fir conviene FFT, por menos conviene convolucino en tiempo}
      \end{itemize}
   }
\end{frame}
%-------------------------------------------------------------------------------
\begin{frame}[t]{Filtrado}{Pyfda \href{/opt/anaconda3/bin/pyfdax}{/opt/anaconda3/bin/pyfdax}}
   \center\includegraphics[width=0.75\textwidth]{5_clase/pyfda2}
   \vfill
   \note{
      \begin{itemize}
         \item{explicar ahora el uso de la convolucino en el filtrado}
         \item{a partir de 64 puntos de fir conviene FFT, por menos conviene convolucino en tiempo}
      \end{itemize}
   }
\end{frame}
%-------------------------------------------------------------------------------
\begin{frame}[t]{Convolucion}{Superponer y sumar}
   \center\includegraphics[width=0.8\textwidth]{5_clase/overlap_add1}
   \vfill
   \note{
      \begin{itemize}
         \item{explicar el detalle de overlap para sumar}
      \end{itemize}
   }
\end{frame}
%-------------------------------------------------------------------------------
\begin{frame}[t]{Convolucion con FFT}{Superponer y sumar}
   \center\includegraphics[width=0.8\textwidth]{5_clase/overlap_add2}
   \vfill
   \note{
      \begin{itemize}
         \item{explicar el detalle de overlap para sumar}
      \end{itemize}
   }
\end{frame}
%-------------------------------------------------------------------------------
\begin{frame}{Bibliografía}
   \framesubtitle{Libros, links y otro material}
   \begin{thebibliography}{9}
         \bibitem{cmsisdsp}
         \emph{ARM CMSIS DSP}. \\
         \href {https://arm-software.github.io/CMSIS_5/DSP/html/index.html}{https://arm-software.github.io/CMSIS\_5/DSP/html/index.html}
         \bibitem{dsp}
         Steven W. Smith.
         \emph{The Scientist and Engineer's Guide to Digital Signal Processing}.
         Second Edition, 1999.
         \bibitem{grants Anderson}
         \emph{Grant Sanderson} \\
         \href{ https://youtu.be/spUNpyF58BY}{ https://youtu.be/spUNpyF58BY}
         \bibitem{interactivemath}
         \emph{Interactive Mathematics Site Info}. \\
         \href {https://www.intmath.com/fourier-series/fourier-intro.php}{https://www.intmath.com/fourier-series/fourier-intro.php}
   \end{thebibliography}
\end{frame}
%-------------------------------------------------------------------------------

