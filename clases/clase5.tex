\subtitle{Clase 5 - Applicaciones de DFT}

\begin{frame}[c]
\maketitle
\begin{tikzpicture}[overlay,remember picture]
    \node[anchor=south east,xshift=-30pt,yshift=45pt]
      at (current page.south east) {
        \includegraphics[width=0.4\textwidth]{5_clase/dft_apps}
      };
  \end{tikzpicture}
   \note{
      \begin{itemize}
         \item{arrancar repaso de numeros Q con nueva clase 2}
         \item{arrancar comentando el repaso de convolucion con otro enfoque}
         \item{Recordar el tema de la encuesta}
         \item{Dejar espacio al final de la clase para ver numeros Q}
      \end{itemize}
   }
\end{frame}
%-------------------------------------------------------------------------------
 \begin{frame}{SAPI}
    \framesubtitle{Se aceptan pull request para la SAPI}
 \LARGE 
 \centering{SAPI DSP}
\end{frame}
%%-------------------------------------------------------------------------------
 \begin{frame}{Enuestas}
    \framesubtitle{Encuesta anónima clase a clase}
    Propiciamos este espacio para compartir sus sugerencias, criticas constructivas, oportunidades de mejora y cualquier tipo de comentario relacionado a la clase.
    \begin{block}{Encuesta anónima}{
       \includegraphics[width=0.1\textwidth]{1_clase/click}
       \href{https://forms.gle/1j5dDTQ7qjVfRwYo8}{https://forms.gle/1j5dDTQ7qjVfRwYo8}
    }
       \end{block}
    \begin{block}{Link al material de la material}{
       \includegraphics[width=0.1\textwidth]{1_clase/click}
       \tiny{\href{https://drive.google.com/drive/u/1/folders/1TlR2cgDPchL\_4v7DxdpS7pZHtjKq38CK}{https://drive.google.com/drive/u/1/folders/1TlR2cgDPchL\_4v7DxdpS7pZHtjKq38CK}
    }
    }
       \end{block}
\end{frame}
%-------------------------------------------------------------------------------
\begin{frame}{Repaso Convolución }{Multiplicacion?!}
   \videoicon{55m50s}
   Algoritmo de Multiplicacion de 2do grado
   \begin{columns}[c]
      \begin{column}{0.4\textwidth}
      \begin{itemize}
         \item{Recordar la tecnica de multiplicacion aprendida}
         \item{Considerar que 1 2 es h(n) y 3 4 x(n)}
         \item{Suponer que tenemos mas que 10 simbolos y que no es necesario hacer acarreo}
         \item{Entran 2 numeros de 2 cifras y sale 1 de 3}
      \end{itemize}
      \end{column}
      \hspace{2pt}
      \vrule
      \hspace{2pt}
      \begin{column}{0.6\textwidth}
         \begin{align*}
            1\hspace{1cm} 2\\
            3\hspace{1cm} 4 \\
            \cline{1-2}
            4 \hspace{1cm} 8 \\
            3 \hspace{1cm} 6 \hspace{1cm}0 \\
            \cline{1-2}
            3 \hspace{1cm} 10 \hspace{1cm} 8\\
   \end{align*}
      \end{column}
   \end{columns}
   \vfill
   \note{
      \begin{itemize}
         \item{no hay que lanzar nada}
         \item{explicar 3 manera de multiplicar un numero}
         \item{darle forma de respuesta al impulso y senial}
      \end{itemize}
   }
\end{frame}
%-------------------------------------------------------------------------------
\begin{frame}{Repaso Convolución }{Descomposición delta}
   \videoicon{56m57s}
   \begin{columns}[c]
      \begin{column}{0.4\textwidth}
         \tiny
         SUma deltas desplazadas
         \begin{itemize}
            \item{Considerar escalar y desplazar h(n) para hacer la convolucion}
            \item{2 movimientos para obtener el resultado}
            \item{Entran 2 vectores de 2 elementos y sale 1 de 3}
         \end{itemize}
         \begin{align*}
            \textcolor{cyan} {1} \hspace{1cm}\textcolor{cyan}{2} \hspace{1cm}0\\
            \textcolor{green}{3} \hspace{1cm}                 0  \hspace{1cm}0\\
            \cline{1-2}
            3\hspace{1cm} 6 \hspace{1cm}0\\[0.4cm]
            0\hspace{1cm}\textcolor{cyan} {1} \hspace{1cm}\textcolor{cyan}{2}\\
            0\hspace{1cm}\textcolor{green}{4} \hspace{1cm}                 0\\
            \cline{1-2}
            3\hspace{1cm} 6 \hspace{1cm}0\\
            0\hspace{1cm} 4 \hspace{1cm}8\\
            \cline{1-2}
            3 \hspace{1cm}10 \hspace{1cm} 8\\
         \end{align*}
      \end{column}
      \hspace{2pt}
      \vrule
      \hspace{2pt}
      \begin{column}{0.6\textwidth}
      \pythonpic{5\_clase/conv\_as\_multiply1.py}
         {https://drive.google.com/open?id=1lDBXrOKVYTdkmrzaXJroY8VZDN0ZGJFm}
         {0.7}
         {5_clase/conv_as_multiply1}
      \end{column}
   \end{columns}
   \vfill
   \note{
      \begin{itemize}
         \item{lanzar conv\_as\_multiply1}
         \item{muestro la misma cuenta con señales}
      \end{itemize}
   }
\end{frame}
%-------------------------------------------------------------------------------
\begin{frame}{Repaso Convolución}{Convolucion formal}
   \videoicon{1h3m12s}
   \begin{columns}[t]
      \begin{column}{0.4\textwidth}
      \tiny
      \centering{Convolucion}
            \begin{align*}
                                2 \hspace{1cm} \textcolor{green}{1}\hspace{1cm}0 \hspace{1cm}0\\
               \textcolor{cyan}{0}\hspace{1cm} \textcolor{cyan} {3}\hspace{1cm}4 \hspace{1cm}0\\
               \cline{1-2}
               3\hspace{1cm} 0\hspace{1cm}0\\[0.4cm]
               0\hspace{1cm}                  2 \hspace{1cm}\textcolor{green}{1} \hspace{1cm}0\\
               0\hspace{1cm} \textcolor{cyan}{3}\hspace{1cm}\textcolor{cyan} {4} \hspace{1cm}0\\
               \cline{1-2}
               3\hspace{1cm} 10\hspace{1cm}0\\[0.4cm]
               0\hspace{1cm} 0\hspace{1cm}                 2 \hspace{1cm}\textcolor{green}{1}\\
               0\hspace{1cm} 3\hspace{1cm}\textcolor{cyan}{4}\hspace{1cm}\textcolor{cyan} {0}\\
               \cline{1-2}
               3 \hspace{1cm} 10 \hspace{1cm} 8\\
            \end{align*}
      \end{column}
      \hspace{2pt}
      \vrule
      \hspace{2pt}
      \begin{column}{0.6\textwidth}
      \tiny
      \begin{itemize}
         \item{Invierto h(n) desplazo, multiplico y sumo}
         \item{3 movimientos para obtener el resultado}
         \item{Entran 2 vectores de 2 elementos y sale 1 de 3}
      \end{itemize}
      \pythonpic{5\_clase/conv\_as\_multiply2.py}
         {https://drive.google.com/open?id=1ir0AEgJr9dMKKGQ9xMInM3QkxmOsMxaN}
         {0.55}
         {5_clase/conv_as_multiply2}
      \end{column}
   \end{columns}
   \vfill
   \note{
      \begin{itemize}
      \item{lanzar conv\_as\_multiply2}
         \item{muestro la misma cuenta con señales}
      \end{itemize}
   }
\end{frame}
%-------------------------------------------------------------------------------
\begin{frame}{Repaso Convolución}{Convolucion como producto de polinomios}
   \videoicon{1h10m50s}
   \begin{itemize}
      \item{Considerar cada elemento de h(n) y x(n) como coeficientes de un polinomio}
      \item{Multiplicar los 2 polinomios respetando exponentes}
      \item{Entran 2 vectores de 2 elementos y sale 1 de 3}
   \end{itemize}
   \begin{align*}
      \large
      \left( 1 x10^1 + 2 x10^0 \right) * \left( 3 x10^1 + 4 x10^0 \right) &=  \\
      \left( 3 x10^2 + 4 x10^1 + 6 x10^1 + 8 x10^0\right) &=  \\
      \left( 3 x10^2 + 10 x10^1 + 8 x10^0\right)\\
   \end{align*}
   \vfill
   \note{
      \begin{itemize}
         \item{comentar que tambien se puede ver como multiplicacion de polinomios}
         \item{en el caso de la convolucion, no se trata de 10\^x sino que queda expresado en ese orden cada termino}
      \end{itemize}
   }
\end{frame}
%-------------------------------------------------------------------------------
\begin{frame}{Repaso Convolución}{Multiplicacion?!}
   \videoicon{1h20m49s}
   \begin{columns}[t]
      \hspace{5pt}
   \begin{column}{.3\textwidth}
      \tiny
      Algoritmo de Multiplicacion
          \begin{align*}
               1\hspace{1cm} 2\\
               3\hspace{1cm} 4 \\
               \cline{1-2}
               4 \hspace{1cm} 8 \\
               3 \hspace{1cm} 6 \hspace{1cm}0 \\
               \cline{1-2}
               3 \hspace{1cm} 10 \hspace{1cm} 8\\
            \end{align*}
      Multiplicacion de polinomios
            \begin{align*}
               \left( 1 x10^1 + 2 x10^0 \right) * \left( 3 x10^1 + 4 x10^0 \right) &=  \\
               \left( 3 x10^2 + 4 x10^1 + 6 x10^1 + 8 x10^0\right) &=  \\
               \left( 3 x10^2 + 10 x10^1 + 8 x10^0\right) \\
            \end{align*}
      \end{column}
      \hspace{2pt}
      \vrule
      \hspace{2pt}
      \begin{column}{.3\textwidth}
      \tiny
      SUma deltas desplazadas
            \begin{align*}
               \textcolor{cyan} {1} \hspace{1cm}\textcolor{cyan}{2} \hspace{1cm}0\\
               \textcolor{green}{3} \hspace{1cm}                 0  \hspace{1cm}0\\
               \cline{1-2}
               3\hspace{1cm} 6 \hspace{1cm}0\\[0.4cm]
               0\hspace{1cm}\textcolor{cyan} {1} \hspace{1cm}\textcolor{cyan}{2}\\
               0\hspace{1cm}\textcolor{green}{4} \hspace{1cm}                 0\\
               \cline{1-2}
               3\hspace{1cm} 6 \hspace{1cm}0\\
               0\hspace{1cm} 4 \hspace{1cm}8\\
               \cline{1-2}
               3 \hspace{1cm}10 \hspace{1cm} 8\\
            \end{align*}
      \end{column}
      \hspace{2pt}
      \vrule
      \hspace{2pt}
      \begin{column}{.3\textwidth}
      \tiny
      Convolucion
            \begin{align*}
                                2 \hspace{1cm} \textcolor{green}{1}\hspace{1cm}0 \hspace{1cm}0\\
               \textcolor{cyan}{0}\hspace{1cm} \textcolor{cyan} {3}\hspace{1cm}4 \hspace{1cm}0\\
               \cline{1-2}
               3\hspace{1cm} 0\hspace{1cm}0\\[0.4cm]
               0\hspace{1cm}                  2 \hspace{1cm}\textcolor{green}{1} \hspace{1cm}0\\
               0\hspace{1cm} \textcolor{cyan}{3}\hspace{1cm}\textcolor{cyan} {4} \hspace{1cm}0\\
               \cline{1-2}
               3\hspace{1cm} 10\hspace{1cm}0\\[0.4cm]
               0\hspace{1cm} 0\hspace{1cm}                 2 \hspace{1cm}\textcolor{green}{1}\\
               0\hspace{1cm} 3\hspace{1cm}\textcolor{cyan}{4}\hspace{1cm}\textcolor{cyan} {0}\\
               \cline{1-2}
               3 \hspace{1cm} 10 \hspace{1cm} 8\\
            \end{align*}
      \end{column}
      \hspace{2pt}
   \end{columns}
   \vfill
   \note{
      \begin{itemize}
         \item{no hay que lanzar nada}
         \item{explicar 3 manera de multiplicar un numero}
         \item{darle forma de respuesta al impulso y senial}
      \end{itemize}
   }
\end{frame}
%-------------------------------------------------------------------------------
\begin{frame}{Convolucion en tiempo}{filtrado}
   \videoicon{1h21m19s}
   \lstset{ basicstyle=\fontsize{ 5}{ 1}\selectfont\ttfamily,language=python,tabsize=4}
   \begin{columns}[c]
      \begin{column}{.5\textwidth}
         \pythonpic{4\_clase/conv1.py}
         {https://drive.google.com/open?id=1Bi0Lz0GcDI8teTLNse-S4px0tQTr1sVx}
         {0.9}
         {4_clase/conv1}
      \end{column}
      \hspace{2pt}
      \vrule
      \hspace{2pt}
      \begin{column}{.5\textwidth}
      \pythonpic{4\_clase/conv2.py}
                {https://drive.google.com/open?id=1SYwy_vsVlnUdArNaxykd93BW14uWSV39}
                {0.9}
                {4_clase/conv2}
      \end{column}
      \hspace{2pt}
   \end{columns}
   \vfill
   \note{
      \begin{itemize}
         \item{lanzar conv1 y luego conv2}
         \item{comentar lo que ya sabemos hacer con la convolucion en tiempo}
         \item{prepara la idea para F}
      \end{itemize}
   }
\end{frame}
%-------------------------------------------------------------------------------
\begin{frame}{Repaso Convolucion}{Propiedades}
   \videoicon{1h37m55s}
   \begin{columns}[c]
      \hspace{5pt}
      \begin{column}{.2\textwidth}
         \begin{itemize}
            \item{Conmutativa}
            \item{Distributiva}
            \item{Asociativa}
         \end{itemize}
      \end{column}
      \hspace{2pt}
      \vrule
      \hspace{2pt}
      \begin{column}{.8\textwidth}
         \centering\includegraphics[width=0.7\textwidth]{4_clase/entrada_conv_h}\\
         \centering\includegraphics[width=0.7\textwidth]{4_clase/convolucion_eq}
      \end{column}
      \hspace{2pt}
   \end{columns}
   \vfill
\end{frame}
%-------------------------------------------------------------------------------
\begin{frame}{Repaso Multiplicacion}{Propiedad conmutativa}
   \videoicon{1h39m58s}
   \begin{columns}[c]
      \hspace{5pt}
      \begin{column}{0.5\textwidth}
         \centering\includegraphics[width=0.8\textwidth]{5_clase/multi_conmutativa}
      \end{column}
      \hspace{2pt}
      \vrule
      \hspace{2pt}
      \begin{column}{0.5\textwidth}
         \centering\includegraphics[width=0.9\textwidth]{5_clase/conv_conmutativa}
      \end{column}
      \hspace{2pt}
   \end{columns}
   \vfill
\end{frame}
%-------------------------------------------------------------------------------
\begin{frame}{Repaso Multiplicacion}{Propiedad asociativa}
   \videoicon{1h40m49s}
   \begin{columns}[c]
      \hspace{5pt}
      \begin{column}{0.5\textwidth}
         \centering\includegraphics[width=0.8\textwidth]{5_clase/multi_asociativa}
      \end{column}
      \hspace{2pt}
      \vrule
      \hspace{2pt}
      \begin{column}{0.5\textwidth}
         \centering\includegraphics[width=0.9\textwidth]{5_clase/conv_asociativa}
      \end{column}
      \hspace{2pt}
   \end{columns}
   \vfill
\end{frame}
%-------------------------------------------------------------------------------
\begin{frame}{Repaso Multiplicacion}{Propiedad distributiva}
   \videoicon{1h42m40s}
   \begin{columns}[c]
      \hspace{5pt}
      \begin{column}{0.5\textwidth}
         \centering\includegraphics[width=0.8\textwidth]{5_clase/multi_distributiva}
      \end{column}
      \hspace{2pt}
      \vrule
      \hspace{2pt}
      \begin{column}{0.5\textwidth}
         \centering\includegraphics[width=0.9\textwidth]{5_clase/conv_distributiva}
      \end{column}
      \hspace{2pt}
   \end{columns}
   \vfill
\end{frame}
%-------------------------------------------------------------------------------
\section{Convolución vs Multiplicación}
\begin{frame}[t]{Convolución vs Multiplicación}{Teorema de la convolución}
   \videoicon{2h07m18s}
   \center\includegraphics[width=0.7\textwidth]{5_clase/dft_apps}\\
   \vfill
   \note{
      \begin{itemize}
         \item{explicar que cuando la entrada en t es la delta estamos simulando barrido en f, porque la delta tiene todas las f}
         \item{luego en F la entrada son círculos rotando multiplicados por círculos rotando, amplitud se escala y fase(exponente) se suman, se corre la fase}
         \item{no hacemos la demostración del teorema, sino que lo probamos prácticamente}
         \item{explicar la conclusion y el teorema de la convolución}
      \end{itemize}
   }
\end{frame}
%-------------------------------------------------------------------------------
\begin{frame}{Multiplicación con DFT}{Tiempo vs Frecuencia}
   \videoicon{2h18m24s}
   \begin{columns}[c]
      \begin{column}{.4\textwidth}
         \tiny
         \begin{itemize}
            \item{Ejemplo de calcular 1 2 conv 3 4 utilizando:}
               \begin{itemize}
                     \tiny
                  \item{Convolución x desplazamiento de h(n)}
                  \item{Convolución invirtiendo h(n)}
                  \item{Convolución usando iFFT(fft(h)*fft(x))}
               \end{itemize}
            \item{Notar que todos los resultados son iguales}
            \item{Notar que siempre la salida con N+M-1 datos}
            \item{En este ej. 2+2-1=3}
         \end{itemize}
      \end{column}
      \hspace{2pt}
      \vrule
      \hspace{2pt}
      \begin{column}{.6\textwidth}
         \pythonpic{5\_clase/conv\_vs\_dft1.py}
         {https://drive.google.com/open?id=1RpE_63Srvsdfrnsab7lH_73yaFsILF_o}
         {1.0}
         {5_clase/conv_vs_dft1}
      \end{column}
      \hspace{2pt}
   \end{columns}
   \vfill
   \note{
      \begin{itemize}
         \item{lanzar conv\_vs\_dft1}
         \item{explicar multiplicación usando DFT}
         \item{hacer notar que hay que estirar las cosas para que la salida tenga N+M-1}
         \item{explicar multiplicación usando DFT}
      \end{itemize}
   }
\end{frame}
%-------------------------------------------------------------------------------
\begin{frame}[t]{Convolución vs Multiplicación}{Teorema de la convolución}
   \videoicon{2h32m53s}
   \begin{columns}[t]
      \begin{column}{.5\textwidth}
         \begin{itemize}
            \item{Imagen en donde se destaca el camino de la ifft para calcular la convolución}
            \item{Notar la representación en parte real e imaginaria}
         \end{itemize}
      \end{column}
      \hspace{2pt}
      \vrule
      \hspace{2pt}
      \begin{column}{.5\textwidth}
         \center\includegraphics[width=0.7\textwidth]{5_clase/teorema_conv1}
      \end{column}
      \hspace{2pt}
   \end{columns}
   \vfill
   \note{
      \begin{itemize}
         \item{explicar la conclusion y el teorema de la convolución}
         \item{explicar que dado h r y(t) podemos dividir en frec y obtener x()}
      \end{itemize}
   }
\end{frame}
%-------------------------------------------------------------------------------
\begin{frame}[t]{Convolución vs Multiplicación}{Convolución circular}
   \videoicon{2h45m10s}
   \begin{columns}[t]
      \begin{column}{.5\textwidth}
         \begin{itemize}
            \item{Efecto de la convolución circular}
            \item{De la definición de la DFT se supone que x(n) es periódica}
            \item{Dado que en la practica cortamos en algún pinto x(n) para procesar, cuando involucionamos con otra señal debemos agregar ceros para evitar el efecto de solapamiento en la salida y asegurarnos que la salida tenga cuando menos N+M-1 valores}
         \end{itemize}
      \end{column}
      \hspace{2pt}
      \vrule
      \hspace{2pt}
      \begin{column}{.5\textwidth}
         \center\includegraphics[width=0.6\textwidth]{5_clase/teorema_conv2}
      \end{column}
      \hspace{2pt}
   \end{columns}
   \vfill
   \note{
      \begin{itemize}
         \item{explicar el Efecto de la convolución circular}
      \end{itemize}
   }
\end{frame}
%-------------------------------------------------------------------------------
\begin{frame}[t]{Convolución vs Multiplicación}{Teorema de la convolución}
   \videoicon{2h50m00s}
   \begin{columns}[t]
      \begin{column}{.5\textwidth}
         \begin{itemize}
            \item{Ecuación para obtener la salida de un sistema usando DFT}
            \item{El resultado es el mismo que convolucionar en el tiempo}
            \item{A partir de unos 64 puntos para h(n) la velocidad de calculo de la DFT es superior a la convolución en el tiempo}
         \end{itemize}
      \end{column}
      \hspace{2pt}
      \vrule
      \hspace{2pt}
      \begin{column}{.5\textwidth}
            \center\includegraphics[width=0.8\textwidth]{5_clase/teorema_conv_eq}
      \end{column}
      \hspace{2pt}
   \end{columns}
   \vfill
   \note{
      \begin{itemize}
         \item{explicar el tema de la DTFT y la transformada circular}
      \end{itemize}
   }
\end{frame}
%-------------------------------------------------------------------------------
\begin{frame}[t]{Filtrado}{Pasabajos}
   \videoicon{2h54m45s}
   \begin{columns}[t]
      \begin{column}{.4\textwidth}
         \begin{itemize}
            \item{Ejemplo de como tomar tramos de x, rellenar con ceros, tomar h, rellenar con ceros y luego convulocionar}
            \item{Al mismo tiempo, se calcula FFT(x-padd) FFT(h-padd), se multiplican entre si y luego se hace la IFFT para obtener el mismo resultado que la convolución}
         \end{itemize}
      \end{column}
      \hspace{2pt}
      \vrule
      \hspace{2pt}
      \begin{column}{.6\textwidth}
         \pythonpic{5\_clase/filtrado1.py}
         {https://drive.google.com/open?id=1V7WoFU8KfxnUAZnNDfRTsE12xJ7oq3cA}
         {1.0}
         {5_clase/filtrado1}
      \end{column}
      \hspace{2pt}
   \end{columns}
   \vfill
   \note{
      \begin{itemize}
         \item{explicar ahora el uso de la convolution en el filtrado}
         \item{a partir de 64 puntos de fir conviene FFT, por menos conviene convolution en tiempo}
      \end{itemize}
   }
\end{frame}
%-------------------------------------------------------------------------------
\begin{frame}[t]{Filtrado}{Pasaaltos}
   \videoicon{3h04m50s}
   \begin{columns}[t]
      \begin{column}{.4\textwidth}
         \begin{itemize}
            \item{Ejemplo de como tomar tramos de x, rellenar con ceros, tomar h, rellenar con ceros y luego convulocionar}
            \item{Al mismo tiempo, se calcula FFT(x-padd) FFT(h-padd), se multiplican entre si y luego se hace la IFFT para obtener el mismo resultado que la convolucion}
         \end{itemize}
      \end{column}
      \hspace{2pt}
      \vrule
      \hspace{2pt}
      \begin{column}{.6\textwidth}
         \pythonpic{5\_clase/filtrado2}
         {https://drive.google.com/open?id=1V7WoFU8KfxnUAZnNDfRTsE12xJ7oq3cA}
         {1.0}
         {5_clase/filtrado2}
      \end{column}
      \hspace{2pt}
   \end{columns}
   \vfill
   \note{
      \begin{itemize}
         \item{explicar ahora el uso de la convolution en el filtrado}
         \item{a partir de 64 puntos de fir conviene FFT, por menos conviene convolution en tiempo}
      \end{itemize}
   }
\end{frame}
%-------------------------------------------------------------------------------
\begin{frame}[t]{Filtrado}{Definición}
   \videoicon{3h09m58s}
   \begin{columns}[t]
      \footnotesize
      \begin{column}{.4\textwidth}
         \begin{itemize}
            \item{Plantilla de diseño de un filtro}
            \item{En el ejemplo se aprecia un pasabajos pero se destacan las zonzas de interés y los niveles de la banda de paso y de rechazo}
            \item{Cuanto mas exigente se la plantilla del filtro, mas puntos tendrá nuestra h(n) y mas lenta y compleja su convolución}
            \item{El objetivo es llegar a un compromiso entre los requisitos y la performance }
         \end{itemize}
      \end{column}
      \hspace{2pt}
      \vrule
      \hspace{2pt}
      \begin{column}{.6\textwidth}
         \center\includegraphics[width=0.8\textwidth]{5_clase/pyfda3}
      \end{column}
      \hspace{2pt}
   \end{columns}
   \vfill
   \note{
      \begin{itemize}
         \item{explicar las zonas de los filtros, tipos de filtro}
         \item{relación de compromiso entre ripple y bandas, etc}
      \end{itemize}
   }
\end{frame}
%-------------------------------------------------------------------------------
\begin{frame}[t]{Filtrado}{PyFDA \href{/opt/anaconda3/bin/pyfdax}{/opt/anaconda3/bin/pyfdax}}
   \videoicon{3h14m00s}
   \begin{columns}[t]
      \footnotesize
      \begin{column}{.4\textwidth}
         \begin{itemize}
            \item{Uso de PyFDA como herramienta para diseño de filtros}
            \item{Inicialmente solo nos concentramos en la H(f) para visualizar de manera practica las zonas de paso y de rechazo}
         \end{itemize}
      \end{column}
      \hspace{2pt}
      \vrule
      \hspace{2pt}
      \begin{column}{.6\textwidth}
         \center\includegraphics[width=0.80\textwidth]{5_clase/pyfda1}
      \end{column}
      \hspace{2pt}
   \end{columns}
   \vfill
   \note{
      \begin{itemize}
         \item{explicar ahora el uso de la convolution en el filtrado}
         \item{a partir de 64 puntos de fir conviene FFT, por menos conviene convolution en tiempo}
      \end{itemize}
   }
\end{frame}
%-------------------------------------------------------------------------------
\begin{frame}[t]{Filtrado}{Pyfda \href{/opt/anaconda3/bin/pyfdax}{/opt/anaconda3/bin/pyfdax}}
   \videoicon{3h14m00s}
   \begin{columns}[t]
      \footnotesize
      \begin{column}{.4\textwidth}
         \begin{itemize}
            \item{Uso de PyFDA como herramienta para diseño de filtros}
            \item{Inicialmente solo nos concentramos en la H(f) para visualizar de manera practica las zonas de paso y de rechazo}
            \item{Notar la variedad de opciones disponibles y la respuesta en fase en esta imagen}
         \end{itemize}
      \end{column}
      \hspace{2pt}
      \vrule
      \hspace{2pt}
      \begin{column}{.6\textwidth}
         \center\includegraphics[width=0.75\textwidth]{5_clase/pyfda2}
      \end{column}
      \hspace{2pt}
   \end{columns}
   \vfill
   \note{
      \begin{itemize}
         \item{explicar ahora el uso de la convolution en el filtrado}
         \item{a partir de 64 puntos de fir conviene FFT, por menos conviene convolucino en tiempo}
      \end{itemize}
   }
\end{frame}
%-------------------------------------------------------------------------------
\begin{frame}{Bibliografía}
   \framesubtitle{Libros, links y otro material}
   \begin{thebibliography}{9}
         \bibitem{cmsisdsp}
         \emph{ARM CMSIS DSP}. \\
         \href {https://arm-software.github.io/CMSIS_5/DSP/html/index.html}{https://arm-software.github.io/CMSIS\_5/DSP/html/index.html}
         \bibitem{dsp}
         Steven W. Smith.
         \emph{The Scientist and Engineer's Guide to Digital Signal Processing}.
         Second Edition, 1999.
         \bibitem{Teorema de la convolucion}
         \emph{Wikipedia}. \\
         \href {https://en.wikipedia.org/wiki/Convolution\_theorem}{https://en.wikipedia.org/wiki/Convolution\_theorem}
         \bibitem{PyFDA Doc}
         \emph{PyFDA doc}. \\
         \href {https://buildmedia.readthedocs.org/media/pdf/pyfda/latest/pyfda.pdf}{https://buildmedia.readthedocs.org/media/pdf/pyfda/latest/pyfda.pdf}
 \end{thebibliography}
\end{frame}
%-------------------------------------------------------------------------------

