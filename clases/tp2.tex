%-------------------------------------------------------------------------------
\subtitle{Trabajo Practico N\textsuperscript{o} 2}
\begin{frame}[c]
\maketitle
\begin{tikzpicture}[overlay,remember picture]
    \node[anchor=south east,xshift=-40pt,yshift=25pt]
      at (current page.south east) {
        \includegraphics[width=55mm]{4_clase/euler6}
      };
  \end{tikzpicture}
\end{frame}
%-------------------------------------------------------------------------------
 \section{Fourier}
 \begin{frame}{Transformada discreta de Fourier}
    \begin{enumerate}
       \item{Grafique las siguientes señales lado a lado con su respectivo espectro en frecuencias:}
    \begin{itemize}
       \item{senoidal}
       \item{cuadrada}
       \item{triangular}
       \item{delta en t=0}
    \end{itemize}
 Indicando en cada caso los parámetros destacados como:
    \begin{itemize}
       \item{frecuencia}
       \item{amplitud}
       \item{densidad espectral de potencia}
       \item{Fs}
       \item{N}
       \item{B}
    \end{itemize}
    \end{enumerate}
 \end{frame}
%-------------------------------------------------------------------------------
 \begin{frame}{Transformada discreta de Fourier}
    \begin{enumerate}
       \item{Dada la siguiente secuencia de números con N=100 y Fs=200, indique:}
    \begin{itemize}
       \item{Resolución espectral}
       \item{Obtenga el contenido espectral}
       \item{Que técnica conoce para mejorar la resolución en frecuencia?}
       \item{Aplique la técnica, grafique y comente los resultados}
    \end{itemize}

   \lstset{ basicstyle=\fontsize{ 6}{ 0}\selectfont\ttfamily,language=Python,tabsize=4}
   \begin{columns}[c]
      \hspace{2pt}
      \begin{column}{.6\textwidth}
         \lstinputlisting{tp2/zero_padding.txt}
      \end{column}
      \hspace{2pt}
      \vrule
      \begin{column}{.4\textwidth}
         \centering\includegraphics[width=1.0\textwidth]{tp2/zero_padding}
      \end{column}
      \hspace{2pt}
   \end{columns}
    \end{enumerate}
 \end{frame}
%-------------------------------------------------------------------------------
 \begin{frame}{Anti transformada discreta de Fourier}
  Dado el siguiente espectro extraído del archivo fft\_hjs.npy, indique:
   \begin{columns}[c]
      \hspace{2pt}
      \begin{column}{.6\textwidth}
    \begin{itemize}
       \item{Que cree que representa esta señal?  \footnotesize{tip: grafique en 2d la idft}}
       \item{Hasta que punto podría limitar el ancho de banda y que se siga interpretando su significado}
       \item{Grafique para mostrar los resultados}
    \end{itemize}
      \end{column}
      \hspace{2pt}
      \vrule
      \begin{column}{.4\textwidth}
         \centering\includegraphics[width=1.0\textwidth]{tp2/homer}
      \end{column}
      \hspace{2pt}
   \end{columns}
 \end{frame}
%-------------------------------------------------------------------------------
 \begin{frame}{Convolución}
   \begin{columns}[c]
      \hspace{2pt}
      \begin{column}{.6\textwidth}
    Dado el segmento de audio en el archivo chapu\_noise.npy con fs=8000 y sumergido en ruido de alta frecuencia resuelva:
    \begin{itemize}
       \item{Diseñe un filtro que mitigue el efecto del ruido}
       \item{Grafique el espectro antes y después del filtro}
       \item{Reproduzca el segmento antes y después del filtrado}
       \item{Comente los resultados obtenidos}
    \end{itemize}
      \end{column}
      \hspace{2pt}
      \vrule
      \begin{column}{.4\textwidth}
         \centering\includegraphics[width=1.0\textwidth]{tp2/chapu}
      \end{column}
      \hspace{2pt}
   \end{columns}
 \end{frame}


