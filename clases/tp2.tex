%-------------------------------------------------------------------------------
\subtitle{Trabajo Practico N\textsuperscript{o} 2}
\begin{frame}[c]
\maketitle
\begin{tikzpicture}[overlay,remember picture]
    \node[anchor=south east,xshift=-40pt,yshift=25pt]
      at (current page.south east) {
        \includegraphics[width=55mm]{3_clase/euler6}
      };
  \end{tikzpicture}
\end{frame}
%-------------------------------------------------------------------------------
 \section{Fourier}
 \begin{frame}{Transformada discreta de Fourier}
    \begin{enumerate}
       \item{Grafique una senoidal, una cuadrada y una triangular lado a lado con su respectivo espectro en frecuencias indicando los parámetros destacados como densidad espectral de potencia, Fs, N, B, etc.}
       \item{Genere un LA-440 mas tono-442 y utilizando la técnica de zero padding para aumentar la resolución espectral intente discriminar ambas}
       \item{Anime el trazado de la letra inicial de su nombre a partir de la IDFT de las coordenadas de sus vertices y explique su funcionamiento}
    \end{enumerate}
 \end{frame}
