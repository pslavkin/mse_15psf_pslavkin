%-------------------------------------------------------------------------------
\subtitle{Introducción a \\ Python y NumPy}
\begin{frame}[c]
\maketitle
\begin{tikzpicture}[overlay,remember picture]
    \node[anchor=south east,xshift=-30pt,yshift=45pt]
      at (current page.south east) {
        \includegraphics[width=25mm]{python/python_logo}
        \includegraphics[width=35mm]{python/numpy_logo}
      };
  \end{tikzpicture}
\end{frame}
%-------------------------------------------------------------------------------
 \begin{frame}{Anaconda}
       \center\includegraphics[width=1\textwidth]{python/anaconda}
    \vfill
 \end{frame}
%-------------------------------------------------------------------------------
 \begin{frame}{Anaconda}{jupyter-notebook}
       \center\includegraphics[width=1\textwidth]{python/jupyter}
    \vfill
 \end{frame}
%-------------------------------------------------------------------------------
 \begin{frame}{Anaconda}{Spyder}
       \center\includegraphics[width=1\textwidth]{python/spyder}
    \vfill
 \end{frame}
%-------------------------------------------------------------------------------
 \begin{frame}{Anaconda}{Ipython}
       \center\includegraphics[width=1\textwidth]{python/ipython}
    \vfill
 \end{frame}
%-------------------------------------------------------------------------------
 \begin{frame}{python}{python}
       \center\includegraphics[width=1\textwidth]{python/python3}
    \vfill
 \end{frame}
%-------------------------------------------------------------------------------
 \section{Introducción}
 \subsection{Python}
 \begin{frame}{Introducción}
 \begin{itemize}
    \item{Lenguaje Interpretado}
    \item{Tipado dinámico: El tipo se define en tiempo de ejecución}
    \item{Fuertemente tipado: Durante las operaciones se chequea el tipo}
 \end{itemize}
    \vfill
 \end{frame}
%-------------------------------------------------------------------------------
 \begin{frame}{Indentación}
    \begin{columns}[onlytextwidth]
       \column{.5\textwidth}
       \center\includegraphics[width=0.9\textwidth]{python/indentado}
       \column{.5\textwidth}
       \lstset{ basicstyle=\fontsize{12}{ 5}\selectfont\ttfamily }
       \lstinputlisting[language=Python,tabsize=4]{python/indentado.py}
    \end{columns}
    \vfill
 \end{frame}
%-------------------------------------------------------------------------------
 \begin{frame}{Números}
    \begin{columns}[onlytextwidth]
       \column{.5\textwidth}
       \begin{itemize}
          \item{Enteros: Con signo, sin límite}
          \item{Punto flotante}
          \item{Complejos}
          \item{Booleanos}
       \end{itemize}
       \column{.5\textwidth}
       \lstset{ basicstyle=\fontsize{10}{ 3}\selectfont\ttfamily }
       \lstinputlisting[language=Python,tabsize=4]{python/numeros.py}
    \end{columns}
    \vfill
 \end{frame}
%-------------------------------------------------------------------------------
 \begin{frame}{Cadenas}
    \begin{columns}
       \column{.5\textwidth}
       \begin{itemize}
          \item{Strings: se guardan codificadas. Ej. UTF\-8 y son inmutables}
          \item{Byte array: sin codificar, raw, son mutables}
       \end{itemize}
       \column{.5\textwidth}
       \lstset{ basicstyle=\fontsize{ 8}{3}\selectfont\ttfamily }
       \lstinputlisting[language=Python,tabsize=4]{python/cadenas.py}
    \end{columns}
    \vfill
 \end{frame}
%-------------------------------------------------------------------------------
 \begin{frame}{Listas}
    \begin{columns}
       \lstset{ basicstyle=\fontsize{10}{2}\selectfont\ttfamily }
       \column{.5\textwidth}
       \lstinputlisting[language=Python,tabsize=4,lastline=16]{python/listas.py}
       \column{.5\textwidth}
       \lstinputlisting[language=Python,tabsize=4,firstline=17]{python/listas.py}
    \end{columns}
    \vfill
 \end{frame}
%-------------------------------------------------------------------------------
 \begin{frame}{funciones}
       \lstset{ basicstyle=\fontsize{16}{5}\selectfont\ttfamily }
       \lstinputlisting[language=python,tabsize=4]{python/funciones.py}
    \vfill
 \end{frame}

 \section{numpy}
 \begin{frame}{NumPy}{arrays}
       \lstset{ basicstyle=\fontsize{16}{5}\selectfont\ttfamily }
       \lstinputlisting[language=python,tabsize=4]{python/arrays.py}
    \vfill
 \end{frame}
%-------------------------------------------------------------------------------
 \begin{frame}{NumPy}{linspace, arange}
       \lstset{ basicstyle=\fontsize{ 8}{3}\selectfont\ttfamily }
    \begin{columns}
       \column{.5\textwidth}
       \lstinputlisting[language=Python,tabsize=4,lastline=18]{python/linspace.py}
       \column{.5\textwidth}
       \lstinputlisting[language=Python,tabsize=4,firstline=19]{python/linspace.py}
    \end{columns}
    \vfill
 \end{frame}
%-------------------------------------------------------------------------------
 \begin{frame}{matplotlib}{pyplot}
       \lstset{ basicstyle=\fontsize{ 6}{2}\selectfont\ttfamily }
    \begin{columns}
       \column{.5\textwidth}
       \lstinputlisting[language=Python,tabsize=4]{python/plot.py}
       \column{.5\textwidth}
       \center\includegraphics[width=1.0\textwidth]{python/plot}
    \end{columns}
    \vfill
 \end{frame}
%-------------------------------------------------------------------------------
 \begin{frame}{FuncAnimation}{Animation}
       \lstset{ basicstyle=\fontsize{ 7}{2}\selectfont\ttfamily }
    \begin{columns}
       \column{.5\textwidth}
       \lstinputlisting[language=Python,tabsize=4]{python/animation.py}
       \column{.5\textwidth}
       \center\includegraphics[width=0.6\textwidth]{python/euler1}
       \center\includegraphics[width=0.6\textwidth]{python/visualize}
    \end{columns}
    \vfill
 \end{frame}
%-------------------------------------------------------------------------------
 \begin{frame}{Módulos y paquetes}
       \lstset{ basicstyle=\fontsize{ 10}{2}\selectfont\ttfamily }
    \begin{columns}
       \column{.5\textwidth}
       Módulos:
       \lstinputlisting[language=Python,tabsize=4]{python/modulos.py}
       \column{.5\textwidth}
       Paquetes:
       \lstinputlisting[language=Python,tabsize=4]{python/paquetes.py}
    \end{columns}
    \vfill
 \end{frame}



