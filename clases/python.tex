%-------------------------------------------------------------------------------
\subtitle{Introduccion a \\ Python y NumPy}
\begin{frame}[c]
\maketitle
\begin{tikzpicture}[overlay,remember picture]
    \node[anchor=south east,xshift=-30pt,yshift=45pt]
      at (current page.south east) {
        \includegraphics[width=25mm]{python/python_logo}
        \includegraphics[width=35mm]{python/numpy_logo}
      };
  \end{tikzpicture}
\end{frame}
%-------------------------------------------------------------------------------
 \section{Introduccion}
 \subsection{Python}
 \begin{frame}{Introduccion}
 \begin{itemize}
    \item{Lenguaje Interpretado}
    \item{Tipado dinámico: El tipo se define en tiempo de ejecución}
    \item{Fuertemente tipado: Durante las operaciones se chequea el tipo}
 \end{itemize}
    \vfill
 \end{frame}
%-------------------------------------------------------------------------------
 \begin{frame}{Indentacion}
    \begin{columns}[onlytextwidth]
       \column{.5\textwidth}
       \center\includegraphics[width=0.9\textwidth]{python/indentado}
       \column{.5\textwidth}
       \lstset{ basicstyle=\fontsize{12}{ 5}\selectfont\ttfamily }
       \lstinputlisting[language=Python,tabsize=4]{python/indentado.py}
    \end{columns}
    \vfill
 \end{frame}
%-------------------------------------------------------------------------------
 \begin{frame}{Numeros}
    \begin{columns}[onlytextwidth]
       \column{.5\textwidth}
       \begin{itemize}
          \item{Enteros: Con signo, sin límite}
          \item{Punto flotante}
          \item{Complejos}
          \item{Booleanos}
       \end{itemize}
       \column{.5\textwidth}
       \lstset{ basicstyle=\fontsize{12}{ 5}\selectfont\ttfamily }
       \lstinputlisting[language=Python,tabsize=4]{python/numeros.py}
    \end{columns}
    \vfill
 \end{frame}
%-------------------------------------------------------------------------------
 \begin{frame}{Cadenas}
    \begin{columns}
       \column{.5\textwidth}
       \begin{itemize}
          \item{Strings: se guardan en codificadas. Ej. UTF\-8}
          \item{Byte array: sin codificar, raw}
       \end{itemize}
       \column{.5\textwidth}
       \lstset{ basicstyle=\fontsize{ 8}{3}\selectfont\ttfamily }
       \lstinputlisting[language=Python,tabsize=4]{python/cadenas.py}
    \end{columns}
    \vfill
 \end{frame}
%-------------------------------------------------------------------------------
 \begin{frame}{Listas}
    \begin{columns}
       \lstset{ basicstyle=\fontsize{ 8}{3}\selectfont\ttfamily }
       \column{.5\textwidth}
       \lstinputlisting[language=Python,tabsize=4,lastline=16]{python/listas.py}
       \column{.5\textwidth}
       \lstinputlisting[language=Python,tabsize=4,firstline=17]{python/listas.py}
    \end{columns}
    \vfill
 \end{frame}
%-------------------------------------------------------------------------------
 \begin{frame}{funciones}
       \lstset{ basicstyle=\fontsize{ 8}{3}\selectfont\ttfamily }
       \lstinputlisting[language=python,tabsize=4]{python/funciones.py}
    \vfill
 \end{frame}
%-------------------------------------------------------------------------------
 \section{numpy}
 \begin{frame}{NumPy}{arrays}
       \lstset{ basicstyle=\fontsize{ 8}{3}\selectfont\ttfamily }
       \lstinputlisting[language=python,tabsize=4]{python/arrays.py}
    \vfill
 \end{frame}
%-------------------------------------------------------------------------------
 \begin{frame}{NumPy}{linspace}
       \lstset{ basicstyle=\fontsize{ 8}{3}\selectfont\ttfamily }
    \begin{columns}
       \column{.5\textwidth}
       \lstinputlisting[language=Python,tabsize=4,lastline=18]{python/linspace.py}
       \column{.5\textwidth}
       \lstinputlisting[language=Python,tabsize=4,firstline=19]{python/linspace.py}
    \end{columns}
    \vfill
 \end{frame}

