%-------------------------------------------------------------------------------
\subtitle{Trabajo Practico N\textsuperscript{o} 1}
\begin{frame}[c]
\maketitle
\begin{tikzpicture}[overlay,remember picture]
    \node[anchor=south east,xshift=-40pt,yshift=25pt]
      at (current page.south east) {
        \includegraphics[width=55mm]{1_clase/python_continuo_vs_discreto}
      };
  \end{tikzpicture}
\end{frame}
%-------------------------------------------------------------------------------
 \section{LTI}
 \subsection{Sistemas LTI}
 \begin{frame}{Sistemas LTI}
 Demuestre si los siguientes sistemas son LTI:\\
 \begin{align*}
    y(t) &= x(t)*\cos{(t)} \\
    y(t) &= \cos(x(t)) \\
    y(t) &= e^{x(t)} \\
    y(t) &= \frac{1}{2}x(t)
 \end{align*}
    \vfill
 \end{frame}
%-------------------------------------------------------------------------------
 \section{Ruido de cuantización}
 \begin{frame}{Ruido de cuantización}
    \begin{enumerate}
       \item{Calcule la relación señal a ruido de cuantización teórica máxima de un sistema con un ADC de:}
          \begin{itemize}
             \item{24 bits}
             \item{16 bits}
             \item{10 bits}
             \item{ 8 bits}
             \item{ 2 bits}
          \end{itemize}
       \item{Dado un sistema con un ADC de 10 bits, que técnica le permitiría aumentar la SNR? En que consiste?}
    \end{enumerate}
 \end{frame}
%-------------------------------------------------------------------------------
 \section{FAA}
 \begin{frame}{Filtro antialias y reconstrucción}
    \begin{enumerate}
       \item{Calcular el filtro antialias que utilizara para su practica y/o trabajo final y justifique su decision}
    \end{enumerate}
 \end{frame}
%-------------------------------------------------------------------------------
 \section{Python - Numpy}
 \begin{frame}{Generación y simulación}
    \begin{enumerate}
       \item{Genere un modulo o paquete con al menos las siguientes funciones}
             \begin{itemize}
                \item{senoidal  (fs[Hz], f0[Hz], amp[0 a 1], muestras),fase [radianes]}
                \item{Cuadrada  (fs[Hz], f0[Hz], amp[0 a 1], muestras)}
                \item{Triangular(fs[Hz], f0[Hz], amp[0 a 1], muestras)}
             \end{itemize}
          \item{Realice los siguientes experimentos}
             \begin{itemize}
                \item{fs   = 1000}
                \item{N    = 1000}
                \item{fase = 0}
                \item{amp = 1}
             \end{itemize}
             \begin{enumerate}
                \item{f0 = 0.1*fs y 1.1*fs}
                   Como podría diferenciar las senoidales?
                \item{f0 = 0.49*fs y 0.51*fs}
                    Como es la frecuencia y la fase entre ambas?
             \end{enumerate}
             \begin{block}{\tiny{tip: Grafique los casos superponiendo la misma señal pero sampleada 10 veces mas}}
             \end{block}
    \end{enumerate}




    \note{
   }
 \end{frame}
%-------------------------------------------------------------------------------
 \section{CIAA}
 \begin{frame}{Adquisición y reconstrucción con la CIAA}
    \begin{enumerate}
       \item{Genere con un tono de LA-440. Digitalice con 10, 8, 4 y 2 bits con el ADC, envíe los datos a la PC, grafique y comente los resultados}
          \begin{itemize}
             \item{Señal original con su máximo, mínimo y RMS }
             \item{Señal adquirida con su máximo, mínimo y RMS }
             \item{Señal error = Original-Adquirida}
             \item{Histograma del error}
          \end{itemize}
       \item{Realice el mismo experimento con una cuadrada y una triangular}
    \end{enumerate}
 \end{frame}
%-------------------------------------------------------------------------------
 \section{CMSIS}
 \begin{frame}{Sistema de números}
    \begin{enumerate}
       \item{Explique brevemente algunas de las diferencias entre la representación flotante de simple precision (32b) y el sistema de punto fijo Qn.m}
       \item{Escriba los bits de los siguientes números decimales (o el mas cercano) en float, Q1.15, Q2.14 }
          \begin{itemize}
             \item{0.5}
             \item{-0.5}
             \item{-1.25}
             \item{0.001}
             \item{-2.001}
             \item{204000000}
          \end{itemize}
    \end{enumerate}
   \note{
      respuestas:
      conversion a float desde la pagina: https://www.binaryconvert.com
      conversion a Q desde la pagina: https://www.rfwireless-world.com/calculators/floating-vs-fixed-point-converter.html
      \begin{itemize}
            0.5
        \item{float: 0x3F000000 = 00111111 00000000 00000000 00000000}
        \item{Q1.15: 0x40 00}
        \item{Q2.14: 0x20 00}
      \end{itemize}
      \begin{itemize}
            -0.5
        \item{float: 0xBF000000 = 10111111 00000000 00000000 00000000}
        \item{Q1.15: 0xC0 00}
        \item{Q2.14: 0xE0 00}
      \end{itemize}
      \begin{itemize}
            -1.25
        \item{float: 0xBFA00000 = 10111111 10100000 00000000 00000001}
        \item{Q1.15: no se puede}
        \item{Q2.14: 0xB0 00}
      \end{itemize}
      \begin{itemize}
           0.001
        \item{float: 0x3A83126F = 00111010 10000011 00010010 01101111}
        \item{Q1.15: 0x00 21}
        \item{Q2.14: 0x00 10}
      \end{itemize}
      \begin{itemize}
           -2.001
        \item{float: 0xC0001062 = 11000000 00000000 00010000 01100010}
        \item{Q1.15: no se puede}
        \item{Q2.14: no se puede}
      \end{itemize}
      \begin{itemize}
             204000000
        \item{float: 0x4D428CB0 = 01001101 01000010 10001100 10110000}
        \item{Q1.15: no se puede}
        \item{Q2.14: no se puede}
      \end{itemize}

   }
 \end{frame}
