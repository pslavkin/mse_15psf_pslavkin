\subtitle{Clase 7 - Ejemplo de presentacion - Repaso}

\begin{frame}[c]
\maketitle
\begin{tikzpicture}[overlay,remember picture]
    \node[anchor=south east,xshift=-30pt,yshift=45pt]
      at (current page.south east) {
        \includegraphics[width=0.2\textwidth]{7_clase/afinador_icon}
      };
  \end{tikzpicture}
     \note{
        \begin{itemize}
           \item{comentar lo de los cambios en los archivos del drive}
           \item{comentar lo del espacio para que suban los trabajos los alumnos}
        \end{itemize}
     }
\end{frame}
%-------------------------------------------------------------------------------
\begin{frame}[t]{Afinador de Guitarra}{Requisitos:Medir y comparar los tonos de las cuerdas al aire}
   \videoicon{7}{00m00s}
   \begin{columns}
      \begin{column}{.3\textwidth}
         \begin{itemize}
            \item{E=329.63}
            \item{B=246.94}
            \item{G=196.00}
            \item{D=146.83}
            \item{A=110.00}
            \item{E= 82.41}
         \end{itemize}
      \end{column}
      \begin{column}{.7\textwidth}
         \center\includegraphics[width=1.0\textwidth]{7_clase/chords_frec}
      \end{column}
   \end{columns}
   \vfill
   \note{
      \begin{itemize}
         \item{}
         \item{}
      \end{itemize}
   }
\end{frame}
%-------------------------------------------------------------------------------
\begin{frame}[t]{Afinador de Guitarra}{Estrategia}
   \videoicon{7}{00m00s}
   \footnotesize
   \begin{columns}
      \begin{column}{.5\textwidth}
         \begin{itemize}
            \item{Se define un rango de 60Hz a 380Hz}
            \item{Filtro antialiasing 1er orden en 350Hz. R=15k C=33nF}
            \item{Se samplea a FS=1000hz dado que 1000/2>380}
            \item{Resolucion en frecuencia deseada 0.1hz}
            \item{Refresco de muestras en grafico ~8hz => N=128 dado que a 1k son 128msegs ~8hz}
            \item{Resolucion en f=fs/N = 1000/128 ~8hz => Se hace padding hasta 2048. 1000/2048=0.5Hz}
            \item{Se realiza un average de 2 espectros consecutivos para aumentar la SNR}
            \item{Se utiliza centroide en f para mejorar la resolucion}
         \end{itemize}
      \end{column}
      \begin{column}{.5\textwidth}
         \center\includegraphics[width=0.8\textwidth]{7_clase/bloques}
      \end{column}
   \end{columns}
   \vfill
   \note{
      \begin{itemize}
         \item{}
         \item{}
      \end{itemize}
   }
\end{frame}
%-------------------------------------------------------------------------------
\begin{frame}[t]{Afinador de Guitarra}{Plantilla de filtro}
   \videoicon{7}{00m00s}
      \footnotesize
         \begin{itemize}
            \item{Pasabanda de 60 a 380}
            \item{Elimino los 50Hz}
            \item{h(n) con n~156}
         \end{itemize}
         \begin{columns}
            \begin{column}{.5\textwidth}
               \includegraphics[width=1.0\textwidth]{7_clase/filter_template}
            \end{column}
            \begin{column}{.5\textwidth}
               \includegraphics[width=0.8\textwidth]{7_clase/fir_to_c}
            \end{column}
         \end{columns}
   \vfill
   \note{
      \begin{itemize}
         \item{}
         \item{}
      \end{itemize}
   }
\end{frame}
%-------------------------------------------------------------------------------
\begin{frame}[t]{Afinador de Guitarra}{Max y Local Max}
   \videoicon{7}{00m00s}
   \footnotesize
   \lstset{ basicstyle=\fontsize{ 8}{ 1}\selectfont\ttfamily,language=c,tabsize=4}
   \begin{columns}
      \begin{column}{.5\textwidth}
         \begin{itemize}
            \item{Busco el máximo bin de la DFT}
            \item{Usando un threshold de 1/10 de ese máximo busco un máximo local}
         \end{itemize}
      \lstinputlisting[firstline=81,lastline=99]{7_clase/ciaa/psf3/src/psf.c}
      \end{column}
      \begin{column}{.5\textwidth}
         \includegraphics[width=0.9\textwidth]{7_clase/local_max}
      \end{column}
   \end{columns}
   \vfill
   \note{
      \begin{itemize}
         \item{}
         \item{}
      \end{itemize}
   }
\end{frame}
%-------------------------------------------------------------------------------
%-------------------------------------------------------------------------------
\begin{frame}[t]{Afinador de Guitarra}{Centroide y average}
   \videoicon{7}{00m00s}
   \footnotesize
   \lstset{ basicstyle=\fontsize{ 6}{ 1}\selectfont\ttfamily,language=c,tabsize=4}
   \begin{columns}
      \begin{column}{.5\textwidth}
         \begin{itemize}
            \item{Se toma el promedio del espectro actual con el promedio de todos los anteriores para aumentar la SNR y disminuir los saltos en amplitud en el espectro}
            \item{Calculo el centroide de 21 puntos para aumentar la resolucion de 0.5 a 0.025}
         \end{itemize}
      \lstinputlisting[firstline=100,lastline=112]{7_clase/ciaa/psf3/src/psf.c}
      \lstinputlisting[firstline=113,lastline=117]{7_clase/ciaa/psf3/src/psf.c}
      \end{column}
      \begin{column}{.5\textwidth}
         \includegraphics[width=1.0\textwidth]{7_clase/centroide}
      \end{column}
   \end{columns}
   \vfill
   \note{
      \begin{itemize}
         \item{}
         \item{}
      \end{itemize}
   }
\end{frame}
%-------------------------------------------------------------------------------
\begin{frame}[t]{Afinador de Guitarra}{Hardware Setup}
   \videoicon{7}{00m00s}
   \footnotesize
   \begin{columns}[t]
      \begin{column}{.3\textwidth}
         \begin{itemize}
            \item{Guitarra conectada a amplificado}
            \item{Salida auriculares al antialias}
            \item{Antialias a CIAA}
            \item{CIAA a PC}
         \end{itemize}
      \end{column}
      \begin{column}{.7\textwidth}
         \center\includegraphics[width=0.55\textwidth]{7_clase/bloques_setup}
      \end{column}
   \end{columns}
   \vfill
   \note{
      \begin{itemize}
         \item{}
         \item{}
      \end{itemize}
   }
\end{frame}
%-------------------------------------------------------------------------------
\begin{frame}[t]{Afinador de Guitarra}{GUI}
   \videoicon{7}{00m00s}
   \footnotesize
   \begin{columns}
      \begin{column}{.3\textwidth}
         \begin{itemize}
            \item{Señal analogica de 128 pts}
            \item{DFT de 128pts desde CIAA}
            \item{DFT de 2048pts en Python para referencia}
            \item{Indicador de máximo local}
            \item{Indicador de frecuencia estimada con centroide}
            \item{Zoom de frecuencia para cada cuerda}
         \end{itemize}
      \end{column}
      \begin{column}{.7\textwidth}
       \href{run:./7_clase/demo.mp4}{
         \includegraphics[width=1.0\textwidth]{7_clase/tuner2}
       }
      \end{column}
   \end{columns}
   \vfill
   \note{
      \begin{itemize}
         \item{}
         \item{}
      \end{itemize}
   }
\end{frame}
%-------------------------------------------------------------------------------
\begin{frame}{Bibliografía}
   \framesubtitle{Libros, links y otro material}
   \begin{thebibliography}{9}
         \bibitem{cmsisdsp}
         \emph{ARM CMSIS DSP}. \\
         \href {https://arm-software.github.io/CMSIS_5/DSP/html/index.html}{https://arm-software.github.io/CMSIS\_5/DSP/html/index.html}
         \bibitem{dsp}
         Steven W. Smith.
         \emph{The Scientist and Engineer's Guide to Digital Signal Processing}.
         Second Edition, 1999.
         \bibitem{Teorema de la convolucion}
         \emph{Wikipedia}. \\
         \href {https://en.wikipedia.org/wiki/Convolution\_theorem}{https://en.wikipedia.org/wiki/Convolution\_theorem}
   \end{thebibliography}
\end{frame}
